\documentclass[a4paper, 12pt]{article}
\usepackage[utf8]{inputenc}
\usepackage[T1]{fontenc}
\usepackage[italian]{babel}
\usepackage{a4wide}
\usepackage{graphicx}
\usepackage[table]{xcolor}
\usepackage{geometry}
\usepackage{array}
\usepackage[parfill]{parskip}
\geometry{a4paper,top=1.25cm,bottom=4cm,left=3cm,right=3cm}
\graphicspath{{../includes/pics/}}

%TITOLO
%%%%%%%%%%%%%%%%%%%%%%%%%%%%%%%%%%%%%%%%%%%%%%%%%%%%%%%%%%%%%%%%%%%%%
\newcommand{\thetitle}{Piano di lavoro}
%%%%%%%%%%%%%%%%%%%%%%%%%%%%%%%%%%%%%%%%%%%%%%%%%%%%%%%%%%%%%%%%%%%%%

%VERSIONE
%%%%%%%%%%%%%%%%%%%%%%%%%%%%%%%%%%%%%%%%%%%%%%%%%%%%%%%%%%%%%%%%%%%%%
\newcommand{\theversion}{ X.X.X }
%%%%%%%%%%%%%%%%%%%%%%%%%%%%%%%%%%%%%%%%%%%%%%%%%%%%%%%%%%%%%%%%%%%%%

%DATA
%%%%%%%%%%%%%%%%%%%%%%%%%%%%%%%%%%%%%%%%%%%%%%%%%%%%%%%%%%%%%%%%%%%%%
\newcommand{\thedate}{ G Mese Anno }
%%%%%%%%%%%%%%%%%%%%%%%%%%%%%%%%%%%%%%%%%%%%%%%%%%%%%%%%%%%%%%%%%%%%%

%DESCRIZIONE
%%%%%%%%%%%%%%%%%%%%%%%%%%%%%%%%%%%%%%%%%%%%%%%%%%%%%%%%%%%%%%%%%%%%%
\newcommand{\descript}{\input{../includes/description}}
%%%%%%%%%%%%%%%%%%%%%%%%%%%%%%%%%%%%%%%%%%%%%%%%%%%%%%%%%%%%%%%%%%%%%

\usepackage{ifthen}
\usepackage{ifpdf}
\ifpdf
\usepackage[pdftex]{hyperref}
\else
\usepackage{hyperref}
\fi
\usepackage{color}
\hypersetup{%
colorlinks=true,
linkcolor=black,
citecolor=black,
urlcolor=blue}
\usepackage[nonumberlist]{glossaries}
\usepackage{afterpage}
\setcounter{secnumdepth}{5}
\setcounter{tocdepth}{5}
\usepackage{subfig}
\usepackage{tikz}
\usepackage{tabularx}
\usetikzlibrary{shapes,arrows}
\usepackage{pgfplots}
\pgfplotsset{compat=newest}
\pgfplotsset{plot coordinates/math parser=false}
\newlength\figureheight
\newlength\figurewidth
\pgfkeys{/pgf/number format/.cd,
set decimal separator={,\!},
1000 sep={\,},
}
\renewcommand{\baselinestretch}{1.05}
\usepackage{multirow}
\usepackage{longtable}

%FANCYPAGESTYLE
%%%%%%%%%%%%%%%%%%%%%%%%%%%%%%%%%%%%%%%%%%%%%%%%%%%%%%%%%%%%%%%%%%%%%
\usepackage{fancyhdr}
\pagestyle{fancy}
\fancyfoot[R]{\thepage}
\fancyfoot[L]{\thetitle}
\fancyfoot[C]{}
\fancyhead[R]{\includegraphics[width=0.08\textwidth]{unipd.png}}
\fancyhead[L]{\bfseries\nouppercase{\leftmark}}
\setlength{\headheight}{50pt}
%%%%%%%%%%%%%%%%%%%%%%%%%%%%%%%%%%%%%%%%%%%%%%%%%%%%%%%%%%%%%%%%%%%%%

\let\headruleORIG\headrule
\renewcommand{\headrule}{\color{black} \headruleORIG}
\renewcommand{\headrulewidth}{1.0pt}
\usepackage{colortbl}
\arrayrulecolor{black}

\let\footruleORIG\footrule
\renewcommand{\footrule}{\color{black} \footruleORIG}
\renewcommand{\footrulewidth}{1.0pt}

\fancypagestyle{plain}{
    \fancyhead{}
    \fancyfoot[C]{\thepage}
    \renewcommand{\headrulewidth}{0pt}
}

\usepackage{amsthm}
\usepackage{amssymb,amsmath}
\usepackage{array}
\usepackage{bm}
\usepackage{multirow}
\usepackage[footnote]{acronym}

\newcommand\blankpage{%
\null
\newpage}

\parskip=5pt

\setlength{\arrayrulewidth}{0.2mm}
\setlength{\tabcolsep}{18pt}
\renewcommand{\arraystretch}{1.5}

\begin{document}
%PRIMA PAGINA
%%%%%%%%%%%%%%%%%%%%%%%%%%%%%%%%%%%%%%%%%%%%%%%%%%%%%%%%%%%%%%%%%%%%%
\begin{titlepage}
    \begin{center}
        \includegraphics[width=0.5\textwidth]{unipd.png}\\

        {\large Laurea in Informatica}\\[0.5cm]

        \rule{\linewidth}{0.5mm} \\[0.4cm]
        { \huge \bfseries Piano di lavoro \\[0.4cm] } %TITOLO

        \rule{\linewidth}{0.5mm} \\[1.cm]
        \noindent

        {\Large \textsc{Gianluca Bresolin} \\[0.2cm] \par}
        {\large \textit{2034316} \\[2cm] \par}

        {\large 24 Aprile 2024\\[1.5cm] \par}

        {\large Datasoil s.r.l. \\[1.5cm] \par}

        \vfill
        {\large \href{https://datasoil.it/}{https://datasoil.it/}} %BOTTOM
    \end{center}
\end{titlepage}

%%%%%%%%%%%%%%%%%%%%%%%%%%%%%%%%%%%%%%%%%%%%%%%%%%%%%%%%%%%%%%%%%%%%%
\clearpage %pagina nuova

\tableofcontents
\clearpage

\pagestyle{fancy}

\section{Contatti}

\subsection{Studente}
\begin{itemize}
    \item Gianluca Bresolin
    \item \href{mailto:gianluca.bresolin@studenti.unipd.it}{gianluca.bresolin@studenti.unipd.it}
    \item \href{mailto:gianbreso02@gmail.com}{gianbreso02@gmail.com}
    \item +39 391 30 47 314
\end{itemize}

\subsection{Tutor interno}
\begin{itemize}
    \item Paolo Baldan
    \item \href{mailto:baldan@math.unipd.it}{baldan@math.unipd.it}
\end{itemize}

\subsection{Tutor aziendale}
\begin{itemize}
    \item Pietro De Caro
    \item \href{mailto:pietro.decaro@datasoil.it}{pietro.decaro@datasoil.it}
\end{itemize}

\subsection{Azienda}
\begin{itemize}
    \item Datasoil s.r.l.
    \item Viale Codalunga, Padova (PD)
    \item \href{https://datasoil.it/}{https://datasoil.it/}
    \item P.I./C.F.: 05013950281
\end{itemize}

\clearpage

\section{Scopo dello stage}

\subsection{Informazioni sull'azienda}

%TODO: Aggiungere informazioni sull'azienda

\subsection{Informazioni sullo stage}

Lo scopo del progetto consiste nell'eseguire il refactor e l’ottimizzazione di una libreria ReactJS per la visualizzazione di dashboard parametriche. \\
L’SDK (\textit{Software Development Kit}), integrato nei prodotti Datasoil attraverso un micro servizio dedicato, rende immediata la composizione di dashboard dinamiche e personalizzate 
per la visualizzazione dei dati. Il progetto mira a ridurre la dimensione della codebase, aggiungere nuovi grafici e funzionalità ed aumentare le 
performance attraverso l’impiego di librerie di charting aggiornate e data model più performanti. \\
In quanto la componente è già live su diversi prodotti, la relativa analisi dei requisiti è già stata condotta e non è dunque oggetto del progetto.

\subsection{Contenuti formativi}
Le tecnologie coinvolte nello sviluppo dell'SDK sono:
\begin{itemize}
    \item Typescript;
    \item ReactJS;
    \item CSS;
    \item D3.js.
\end{itemize}

\newpage
\section{Interazione Studente - Tutor aziendale}

%TODO: Aggiungere informazioni sull'interazione

\newpage
\section{Pianificazione del lavoro}

La pianificazione, in termini di quantità di ore lavorative, sarà così distribuita:

\subsection{Distribuzione delle ore}

% \begin{longtable}[c]{|p{1cm}|p{1cm}|p{8cm}|}
%     \hline
%     \multicolumn{2}{|c|}{\textbf{Durata in ore}} & \textbf{Descrizione dell'attività} \\
%     \endfirsthead
%     %
%     \hline
%     \multicolumn{3}{|c|}%
%     {{\bfseries Distribuzione delle ore \thetable\ continua dalla pagina precedente}} \\
%     \multicolumn{2}{|c|}{\textbf{Durata in ore}} & \textbf{Descrizione dell'attività} \\
%     \endhead
%     \hline
%     %
%     \multicolumn{2}{|c|}{80} & Formazione:
%     \begin{itemize}
%         \item Studio data visualization in letteratura;
%         \item Studio infografiche in letteratura;
%         \item Approfondimento su criteri di valutazione per infografiche.
%     \end{itemize} \\ \hline
%     \multirow{3}{*}{98} &  & Progettazione e sviluppo \\ \cline{2-3}
%     & 2 & \begin{itemize}
%         \item Analisi di software gestionali esistenti
%         \item Analisi dell'impianto commerciale del gestionale
%         \item Analisi dell'integrazione del software gestionale con l'applicazione Android
%     \end{itemize} \\ \cline{2-3}
%     & 2 & \begin{itemize}
%         \item Analisi di software gestionali esistenti
%         \item Analisi dell'impianto commerciale del gestionale
%         \item Analisi dell'integrazione del software gestionale con l'applicazione Android
%     \end{itemize} \\ \hline
%     \multicolumn{2}{|c|}{80} & Verifica e documentazione:
%     \begin{itemize}
%         \item Sviluppo ed esecuzione di test sui sistemi prodotti;
%         \item Produzione documentazione dei sistemi prodotti.
%     \end{itemize} \\ \hline
%     1& 2 & 3 \\
%     1& 2 & 3 \\
%     1& 2 & 3 \\
%     1& 2 & 3 \\ \hline
% \end{longtable}

\begin{center}
    \begin{tabular}{ |p{3cm}|p{7cm}|  }
        \hline
        \textbf{Durata in ore} & \textbf{Descrizione dell'attività} \\
        \hline
        56                     & Formazione:

        \begin{itemize}
            \item Studio della libreria ReactJS;
            \item Studio della libreria D3.js;
            \item Studio del SDK Datasoil esistente;
            \item Infografiche e criteri di valutazione;
        \end{itemize}

        \\ \hline
        40                     & Analisi:

        %\begin{itemize}
            %TODO: Aggiungere attività
        %\end{itemize}

        \\ \hline
        172                    & Progettazione e Sviluppo:

        %\begin{itemize}
            %TODO: Aggiungere attività
        %\end{itemize}

        \\ \hline
        40                     & Verifica e stime:

        \begin{itemize}
            \item Esecuzione di test dell'SDK prodotto;
            \item Analisi e stima di nuove richieste dei clienti.
        \end{itemize}

        \\ \hline
        \multicolumn{2}{|c|}{\textbf{308 ore totali}}               \\ \hline
    \end{tabular}
\end{center}

\clearpage

\section{Obiettivi}

Si farà riferimento ai requisiti secondo le seguenti notazioni:

\begin{itemize}
    \item \textbf{OB} per i requisiti obbligatori, vincolanti in quanto obiettivo primario richiesto dal committente;
    \item \textbf{DE} per i requisiti desiderabili, non vincolanti o strettamente necessari, ma dal riconoscibile valore aggiunto;
    \item \textbf{OP} per i requisiti opzionali, rappresentanti valore aggiunto non strettamente competitivo.
\end{itemize}

{Le sigle precedentemente indicate saranno seguite da un numero, identificativo del requisito. Si prevede lo svolgimento dei seguenti obiettivi: \\[0.5cm] \par}

%\begin{center}
%    \begin{tabular}{ |p{3cm}|p{7cm}|  }
%        \hline
%        \multicolumn{2}{|c|}{\textbf{Obbligatorio}}                                                                                \\ \hline
%        \textbf{OB1} & Progettazione del software gestionale ed integrazione con dispositivi mobili, quali telefoni e tablet       \\ \hline
%        \textbf{OB2} & Progettazione database SQL                                                                                  \\ \hline
%        \textbf{OB3} & Realizzazione software Delphi con il framework Firemonkey                                                   \\ \hline
%        \textbf{OB4} & Realizzazione applicazione Android con Java                                                                 \\ \hline
%        \multicolumn{2}{|c|}{\textbf{Desiderabile}}                                                                                \\ \hline
%        \textbf{DE1} & Autonomia della gestione con il cliente per raccogliere nuove richieste                                     \\ \hline
%        \textbf{DE2} & Realizzazione di un thread-pool efficiente per gestire operazioni ricorrenti in modo efficace ed efficiente \\ \hline
%        \multicolumn{2}{|c|}{\textbf{Opzionale}}                                                                                   \\ \hline
%        \textbf{OP1} & Realizzazione di un componente Android Studio per l'interazione ad hoc con il gestionale                    \\ \hline
%    \end{tabular}
%\end{center}

\newpage
\section{Approvazione}
 {Il presente piano di lavoro è stato approvato dai seguenti:}

\vspace*{2cm}
\begin{tabular}{p{4cm}p{4cm}}
                   &                 \\ \hline
    Pietro De Caro & Tutor aziendale
\end{tabular}

\vspace*{2cm}
\begin{tabular}{p{4cm}p{4cm}}
                      &          \\ \hline
    Gianluca Bresolin & Stagista
\end{tabular}

\vspace*{2cm}
\begin{tabular}{p{4cm}p{4cm}}
                 &               \\ \hline
    Paolo Baldan & Tutor interno
\end{tabular}

\end{document}