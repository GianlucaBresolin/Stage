\documentclass[
    table,
    12pt,
    twoside,
    a4paper,
    italian
]{book}

\input{config/packages}
\input{config/thesis_config}

\date{}

\hypersetup{pdfstartview=}
\begin{document}
\frontmatter
\input{preface/1_title_page}
\increasepagenumbering % increase the page numbrering by 1, in order to cout the frontispiece
\input{preface/2_copyright}
\cleardoublepage
\phantomsection
\pdfbookmark{Ringraziamenti}{Ringraziamenti}

\begin{flushright}{
        \slshape
        'L'importante non è vincere, è pensare in modo vincente. La vita è fatta per il 10\% da quello che succede e per il 90\% da come lo affrontiamo.'} \\
    \medskip
    Gianluca Vialli.
\end{flushright}

\begingroup
\let\clearpage\relax
\let\cleardoublepage\relax
\let\cleardoublepage\relax

\chapter*{Ringraziamenti}

\noindent Desidero esprimere la mia gratitudine al professor \myProf, mio relatore, per l'aiuto e il sostegno che mi ha dato durante la stesura dell'elaborato.

\vspace{0.35cm}

\noindent Vorrei esprimere il mio più affettuoso ringraziamento ai miei genitori, Marta e Riccardo, e a mia sorella Emma, per il costante sostegno e
per i valori che mi hanno trasmesso, i quali mi hanno permesso di diventare la persona che sono oggi e di raggiungere i miei traguardi.

\vspace{0.35cm}

\noindent Desidero esprimere un ringraziamento speciale alla mia amata, Rebecca, per il suo supporto, per il suo amore e per la sua costante presenza che mi ha permesso di
affrontare con serenità e tranquillità questo percorso di studi, non facendomi mai sentire solo.

\vspace{0.35cm}

\noindent Ringrazio infine i miei amici Nicola, Manuel, Leonardo, Matteo e Andrea per la loro compagnia, per i momenti di divertimento e per avermi sempre mantenuto in
contatto con la vita e il mondo esterno.

\vspace{0.75cm}

\noindent{\myLocation, \myTime}
\hfill \textit{\myName}

\endgroup

\cleardoublepage
\phantomsection
\blankpage
\pdfbookmark{Compendio}{Compendio}
\begingroup
\let\clearpage\relax
\let\cleardoublepage\relax
\chapter*{Sommario}

Il presente documento descrive il lavoro svolto durante il periodo di stage, della durata di 304 ore, dal laureando Gianluca Bresolin
presso l'azienda \textit{Datasoil S.r.l.} Lo stage ha previsto la realizzazione di una libreria ReactJS attraverso il refactor
e l'ottimizzazione di una libreria utilizzata per la visualizzazione di dashboard parametriche, volta ad ampliare soluzioni e servizi
nel contesto di applicazioni SPA e nella realizzazione di interfacce utente, ponendo attenzione ai requisiti di scalabilità,
usabilità e performance. \newline
La prima attività condotta è stata l'analisi del codice sorgente della libreria esistente, in modo da poter comprendere il funzionamento e le opportunità
di ottimizzazione. Successivamente, sono state individuate le funzionalità da mantenere, quelle da migliorare e quelle da introdurre,
per poi procedere con la progettazione e l'implementazione della nuova libreria. \newline
Infine, la libreria realizzata è stata integrata nei prodotti \textit{Datasoil} attraverso un micro servizio dedicato, rendendo disponibile un SDK
compatibile con le risposte delle API Datasoil, senza richiedere modifiche lato server ai prodotti esistenti. \newline
Il risultato ottenuto al termine del periodo di stage consiste un SDK per la composizione dinamica di dashboard, dalle dimensioni ridotte rispetto
al boundle precedentemente utilizzato dall'azienda e dotato di una maggior robustezza e consistenza grazie ad un uso più opportuno di \textit{TypeScript},
permettendo inoltre di raggiungere una migliore aderenza a quelle che sono le esigenze dei clienti, grazie alla rimozione di grafici inutilizzati e
l'introduzione di nuove funzionalità.


\endgroup
\vfill

\input{preface/5_table_of_contents}
\printglossary[type=\acronymtype, title=Acronimi e abbreviazioni, toctitle=Acronimi e abbreviazioni]
\printglossary[type=main, title=Glossario, toctitle=Glossario]
\blankpage % example of a blank page that also increases the page couter by 1

\mainmatter
\chapter{Introduzione}
\label{chap:introduzione}

\section{L'azienda}
Lo stage è stato svolto presso l'azienda Datasoil S.r.l. situata nei dintorni della stazione ferroviaria di Padova.
Fondata nel 2016, Datasoil S.r.l. è un'azienda di prodotto che si dedica allo sviluppo di piattaforme per l'industria 4.0,
smart building e smart city.
L'obiettivo dell'azienda è integrare informazioni ed eventi provenienti dai vari livelli aziendali per creare insight proattivi
in tempo reale, garantendo che le informazioni corrette raggiungano le persone giuste al momento più opportuno.
Il punto di partenza è la spazialità aziendale e tutti gli asset che risiedono all'interno di essa, da cui
provengono tutti i dati che vengono analizzati trasversalmente grazie a sempre più persone e dispositivi connessi.
Da qui il nome dell'azienda, Datasoil: fare le informazioni da questo suolo fertile di dati da cui siamo circondati.

\begin{figure}[H]
      \centering
      \includegraphics[alt={Logo Datasoil S.r.l.}, width=0.25\columnwidth]{img/datasoil_logo.jpg}
      \caption{Logo Datasoil S.r.l.}
      \label{fig:datasoil}
\end{figure}

\section{L'idea}
\subsection{Il contesto applicativo}
L'azienda Datasoil S.r.l., essendo un'azienda di prodotto, nasce con l'idea di proporre servizi di monitoraggio e controllo
di impianti industriali, smart building e smart city, offrendo come principale prodotto la piattaforma SYN.

\begin{figure}[H]
      \centering
      \includegraphics[alt={Logo SYN}, width=0.25\columnwidth]{img/syn_logo.jpg}
      \caption{Logo SYN}
      \label{fig:syn}
\end{figure}

SYN è una piattaforma di monitoraggio e controllo di impianti industriali che raccoglie dati da sensori e dispositivi
distribuiti all'interno di un impianto o su più stabilimenti, permettendo di visualizzare in tempo reale lo stato di
funzionamento dei vari asset, di analizzare i dati raccolti e di attuare azioni di controllo o segnalazione tramite
ticketing. All'interno della piattaforma, a seconda dei piani attivi del cliente, è possibile visualizzare dashboard dinamiche
in merito ad informazioni filtrate e aggiornate live, permettendo di raggiungere una panoramica completa dello stato dei vari
asset monitorati in modo rapido, intuitivo ed efficace, grazie all'utilizzo di molteplici grafici e widget.

\subsection{Il progetto}
Il progetto svolto durante lo stage consiste nello sviluppo di una libreria TypeScript di componenti per la creazione di dashboard dinamiche.
Questo è stato realizzato tramite il refactoring e l'ottimizzazione di un tool grafico preesistente, integrato nei vari prodotti di Datasoil S.r.l.,
tra cui \textit{SYN}. La libreria è stata implementata a partire da un modulo open-source utilizzato in \textit{Redash},
una piattaforma per la creazione di dashboard dinamiche tramite interrogazioni sulle fonti di dati configurate all'interno del servizio. \newline
L'esigenza di tale progetto nasce dalla necessità da parte di Datasoil S.r.l. di avere una libreria grafica aggiornata alle versioni
correnti delle dipendenze utilizzate, raggiungendo una maggior performance e una maggiore manutenibilità del codice rispetto alla versione
preesistente, riducendo dove possibile le dipendenze esterne e introducendo migliorie grafiche e funzionali. \newline
La libreria sviluppata costituisce, insieme alla già esistente libreria \textit{dashboard} (la quale ha subito anch'essa in parte un processo di refactoring
per permettere l'integrazione con la nuova libreria e l'introduzione di nuove funzionalità), il nuovo SDK utilizzato all'interno dei prodotti Datasoil S.r.l.
per la realizzazione di dashboard dinamiche.

\section{Principali problematiche}
Durante l'analisi iniziale della libreria preesistente in uso all'interno dei prodotti di Datasoil S.r.l., sono emerse alcune problematiche
legate alla manutenibilità e alla performance del tool grafico a seguito di una attenta revisione del codice sorgente. \newline
Di seguito vengono presentate le principali problematiche riscontrate.

\subsection{Dipendenze obsolete}
La libreria preesistente utilizzava versioni obsolete delle dipendenze esterne, con conseguente degradazione delle prestazioni in termini
di utilizzo di spazio e di tempo di caricamente delle risorse, data l'assenza di ottimizzazioni e di aggiornamenti del codice sorgente. \newline
Essendo inoltre la libreria preesistente basata su una versione del modulo di visualizzazioni utilizzato nella piattaforma \textit{Redash},
le tecnologie utilizzate il più delle volte rappresentavano alternative a quelle già utilizzate nei prodotti Datasoil S.r.l., creando dipendenze
non necessarie, con conseguente aumento delle dimensioni del SDK utilizzato per la generazione delle dashboard.

\subsection{Manutenibilità del codice}
Il codice sorgente della libreria preesistente, sviluppato in tempistiche rapide a fronte di una esigenza specifica dell'azienda Datasoil S.r.l.,
risultava essere poco manutenibile e assente di documentazione. \newline
Inoltre, la libreria di visualizzazioni utilizzata da \textit{Redash} su cui si basa l'SDK utilizzato dall'azienda, inizialmente implementata in JavaScript,
ha subito solo successivamente un refactoring in TypeScript: questo refactoring non è però avvenuto con l'introduzione di interfacce e tipi, bensì con la semplice aggiunta
di tipi \textit{any} per le variabili e i parametri delle funzioni, introducendo numerosi \textit{@ts-ignore} per ignorare gli errori di compilazione, rendendo il codice
poco leggibile e difficile da mantenere.

\subsection{Componenti inutilizzate}
In quanto la libreria preesistente fosse realizzata a partire dal modulo utilizzato dalla piattaforma \textit{Redash}, non tutti i componenti presenti
in essa trovavano utilizzo all'interno dei prodotti realizzati da Datasoil S.r.l.: questo in gran parte era dovuto da un'offerta di grafici inopportuni
per quello che è il contesto applicativo dell'azienda.

\subsection{Assenza di funzionalità}
La libreria preesistente non soddisfaceva pienamente le esigenze dei clienti di Datasoil S.r.l., mancando di alcune funzionalità fondamentali.
Questa limitazione ha reso necessario un intervento di refactoring per colmare le lacune e migliorare le prestazioni complessive.

\section{Soluzione scelte}
Per risolvere le problematiche emerse durante l'analisi iniziale della libreria preesistente, si è optato per il refactoring e l'ottimizzazione
del codice sorgente, individuando le soluzioni presentate di seguito.

\subsection{Aggiornamento delle dipendenze}
Per risolvere il problema delle dipendenze obsolete, è stata effettuata un'analisi delle versioni correnti delle dipendenze utilizzate all'interno della libreria,
verificando che le funzionalità utilizzate non fossero deprecate o rimosse, procedendo con l'eventuale refactoring ad un codice compatibile che tenesse inoltre conto
delle nuove funzionalità introdotte nelle versioni più recenti. \newline
Grazie ad uno studio accurato delle librerie utilizzate nei prodotti Datasoil S.r.l., è stato possibile sostituire con librerie equivalenti dipendenze preesistenti,
riducendo così l'onere di utilizzo dell'SDK all'interno dei prodotti aziendali.

\subsection{Introduzione di interfacce e tipi}
Al fine di migliorare la manutenibilità e la comprensione del codice sorgente, è stato svolto un lavoro di introduzione di interfacce e tipi per le variabili e i parametri
delle funzioni, in modo da rendere il codice più leggibile e permettere controlli statici sul codice sorgente. \newline
Questo lavoro ha permesso di ridurre la presenza di tipi \textit{any} all'interno del codice sorgente, garantendo una maggiore sicurezza e affidabilità del prodotto finale.

\subsection{Refactoring del codice}
Per migliorare la comprensione e la manutenibilità del codice sorgente, è stato intrapreso un lavoro di refactoring su alcune porzioni della libreria preesistente.
Questo intervento ha comportato la riscrittura di diverse funzioni e la rimozione di componenti minori utilizzate per eseguire funzionalità semplici, attraverso un codice
più leggibile e immediato, eliminando complessità superflue e mantenendo l'efficacia nell'implementazione di operazioni elementari.

\subsection{Rimozione di componenti inutilizzate}
In merito alla presenza di componenti inutilizzate all'interno della libreria preesistente, è stata effettuata un'accurata analisi delle componenti presenti, selezionando
quelle che non trovavano un diretto utilizzo all'interno dei prodotti Datasoil S.r.l. e procedendo con la loro rimozione. \newline
Questo lavoro ha permesso di ridurre le dimensioni del codice sorgente e di migliorare le prestazioni complessive della libreria, diminuendo inoltre il numero di dipendenze
esterne richiesto per il corretto funzionamento dell'SDK prodotto.

\subsection{Implementazione di nuove funzionalità}
Per colmare le lacune riscontrate nella libreria preesistente in merito all'assenza di alcune funzionalità richieste dai clienti di Datasoil S.r.l. all'interno dei prodotti forniti,
è stato condotto un lavoro di implementazione di nuove funzionalità. \newline
In questo processo, si è cercato di utilizzare dipendenze esterne preferibilmente già esistenti o che fossero compatibili con il contesto applicativo dell'azienda. Qualora ciò non
fosse stato possibile, sono state introdotte nuove dipendenze esterne selezionate a seguito di un'analisi dettagliata che tenesse conto delle implicazioni in termini di dimensioni
finali dell'SDK prodotto.

\section{Strumenti Utilizzati}
La seguente sezione fornisce gli strumenti utilizzati durante lo svolgimento dello stage per la realizzazione della libreria grafica.
Gli strumenti verranno presentati in ordine alfabetico secondo la seguente struttura:
\begin{itemize}
      \item \textbf{Nome strumento}: nome dello strumento utilizzato;
      \item \textbf{Versione}: versione utilizzata nel progetto durante lo stage;
      \item \textbf{Link}: link di riferimento per ulteriori informazioni sullo strumento.
      \item \textbf{Descrizione}: breve descrizione dello strumento e delle sue funzionalità.
      \item \textbf{Vantaggi}: principali vantaggi derivanti dall'utilizzo dello strumento. \textit{(Opzionale)}
      \item \textbf{Svantaggi}: principali svantaggi derivanti dall'utilizzo dello strumento. \textit{(Opzionale)}
      \item \textbf{Alternative Esaminate}: alternative studiate e prese in considerazione durante la scelta dello strumento, con una breve considerazione
            su di esse e la motivazione per la quale non sono state selezionate. \textit{(Opzionale)}
\end{itemize}

\subsection{Tecnologie Frontend}

\subsubsection{Day.js}
\begin{itemize}
      \item \textbf{Nome strumento}: Dayjs
      \item \textbf{Versione}: 1.11.7
      \item \textbf{Link}: \href{https://day.js.org/}{day.js.org}
      \item \textbf{Descrizione}: Day.js è una libreria JavaScript open-source utilizzata per la manipolazione delle date e degli orari.
      \item \textbf{Vantaggi}:
            \begin{itemize}
                  \item Day.js è una libreria molto leggera, con un'ampia serie di funzionalità per la manipolazione delle date e degli orari;
                  \item Day.js viene fornito con dichiarazioni ufficiali di tipo per TypeScript
                  \item Day.js è tutt'ora supportato da una vasta e attiva community.
            \end{itemize}
      \item \textbf{Alternative Esaminate}:
            \begin{itemize}
                  \item \textit{Moment.js}: Moment.js è una libreria JavaScript open-source utilizzata per la manipolazione delle date e degli orari, ma risulta essere più pesante
                        di Day.js. Moment.js è inoltre considerato \textit{deprecated} a favore di Day.js, che offre una maggiore leggerezza e una maggiore efficienza.
            \end{itemize}
\end{itemize}

\subsubsection{Lodash}
\begin{itemize}
      \item \textbf{Nome strumento}: Lodash
      \item \textbf{Versione}: 4.14.0
      \item \textbf{Link}: \href{https://lodash.com/}{lodash.com}
      \item \textbf{Descrizione}: Lodash è una libreria JavaScript open-source utilizzata per la manipolazione di oggetti e array.
      \item \textbf{Vantaggi}:
            \begin{itemize}
                  \item Lodash offre un'ampia serie di funzionalità implementate con efficienza per la manipolazione di oggetti e array,
                        quali la ricerca, la modifica e la rimozione di elementi;
                  \item Lodash è supportato da una vasta e attiva community.
            \end{itemize}
      \item \textbf{Svantaggi}
            \begin{itemize}
                  \item Lodash è una libreria molto pesante, con un'ampia serie di funzionalità che possono non essere utilizzate all'interno del progetto:
                        per questo motivo in questo progetto sono eseguiti degli import sui moduli specifici, riducendo così le dimensioni del package finale.
            \end{itemize}
\end{itemize}

\subsubsection{React}
\begin{itemize}
      \item \textbf{Nome strumento}: React
      \item \textbf{Versione}: 18.0.0
      \item \textbf{Link}: \href{https://reactjs.org/}{reactjs.org}
      \item \textbf{Descrizione}: React è una libreria JavaScript open-source per la creazione di interfacce utente, sviluppata da Facebook.
      \item \textbf{Vantaggi}:
            \begin{itemize}
                  \item React offre un'ampia serie di funzionalità per la creazione di interfacce utente, quali il \textit{Virtual DOM}, il \textit{JSX} e \textit{Hooks};
                  \item React è supportato da una vasta e attiva community;
                  \item React è una libreria molto leggera, permettendo di creare interfacce utente performanti e veloci.
            \end{itemize}
\end{itemize}

\subsubsection{Plotly.js}
\begin{itemize}
      \item \textbf{Nome strumento}: Plotly.js
      \item \textbf{Versione}: custom-bundle: 2.33.0
      \item \textbf{Link}: \href{https://plotly.com/javascript/}{plotly.com/javascript}
      \item \textbf{Descrizione}: Plotly.js è una libreria JavaScript open-source, con supporto per TypeScript, utilizzata per la visualizazione di dati mediante grafici interattivi.
            Costruito sopra \textit{D3.js}, Plotly.js offre un vasto panorama di grafici dinamici in formato SVG, altamente personalizzabili.
      \item \textbf{Vantaggi}:
            \begin{itemize}
                  \item Plotly.js offre funzionalità di interattività avanzate, quali zoom, pan, selezione e salvataggio dei grafici;
                  \item Le disponibilità di grafici offerte da Plotly.js sono molto ampie, permettendo di soddisfare la maggior parte delle esigenze
                        all'interno della libreria grafica;
                  \item Plotly.js è supportato da una vasta e attiva community, con ampie serie di esempi disponibili su \textit{CodePen} (una piattaforma di condivisione di codice
                        online che permette di visualizzare e modificare codice HTML, CSS e JavaScript);
                  \item Plotly.js offre la possibilità di generare custom-bundle personalizzati, permettendo di registrare le sole \textit{traces} che si vogliono utilizzare, riducendo
                        così le dimensioni del pacchetto finale;
                  \item Plotly.js era già precedentemente utilizzato all'interno della libreria preesistente, non necessitando così di modifiche lato backend negli editor
                        utilizzati per la generazione delle risposte JSON da parte dei server Datasoil S.r.l. per la generazione delle dashboard;
                  \item Plotly.js offre una funzionalità, \textit{Plotly.react}, che permette di aggiornare i grafici in modo efficiente, riducendo il tempo di rendering
                        e migliorando le prestazioni complessive della libreria.
            \end{itemize}
      \item \textbf{Svantaggi}:
            \begin{itemize}
                  \item Plotly.js è una libreria molto pesante, dovuto anche dal fatto che è implementata sopra un wrapper proprietario di \textit{D3.js},
                        impedendo l'ottimizzazione di alcune dipendenze non necessarie in quanto non utilizzate;
                  \item La documentazione ufficiale di Plotly.js è vaga e poco esaustiva;
                  \item La versione ufficiale di Plotly.js presenta degli errori durante la registrazine delle \textit{traces}, impedendo di importare correttamente
                        \tetxit{Plotly} all'interno dei moduli TypeScript, motivo per il quale è stata utilizzata una versione custom-bundle di Plotly.js.
            \end{itemize}
      \item \textbf{Alternative Esaminate}:
            \begin{itemize}
                  \item \textit{D3.js}: D3.js è una libreria JavaScript open-source utilizzata per la generazione di grafici dinamici e manipolazione dati, la quale dalla sua parte risulta però
                        essere più complessa rispetto a Plotly.js, richiedendo una maggiore curva di apprendimento e una maggiore quantità di codice per la generazione di grafici.
                  \item \textit{Chart.js}: Chart.js è una libreria JavaScript open-source utilizzata per la generazione di grafici dinamici, la quale risulta essere più leggera di Plotly.js,
                        ma offre una minore quantità di grafici disponibili; il suo utilizzo avrebbe inoltre comportato la richiesta di modifiche lato backend.
            \end{itemize}
\end{itemize}

\subsubsection{TypeScript}
\begin{itemize}
      \item \textbf{Nome strumento}: TypeScript
      \item \textbf{Versione}: 5.1.6
      \item \textbf{Link}: \href{https://www.typescriptlang.org/}{typescriptlang.org}
      \item \textbf{Descrizione}: TypeScript è un linguaggio di programmazione open-source sviluppato da Microsoft, che fa da super-set a JavaScript.
      \item \textbf{Vantaggi}:
            \begin{itemize}
                  \item TypeScript permette di definire tipi ed interfacce per le variabili e i parametri delle funzioni, garantendo una maggiore sicurezza e affidabilità del codice;
                  \item TypeScript permette di effettuare controlli statici sul codice sorgente, riducendo il numero di errori a runtime.
            \end{itemize}
\end{itemize}

\subsection{Strumenti per lo Sviluppo}

\subsubsection{NVM}
\begin{itemize}
      \item \textbf{Nome strumento}: NVM
      \item \textbf{Versione}: 1.1.12
      \item \textbf{Link}: \href{https://github.com/nvm-sh/nvm}{github.com/nvm-sh/nvm}
      \item \textbf{Descrizione}: NVM (Node Version Manager) è uno strumento open-source utilizzato per la gestione delle versioni di Node.js.
      \item \textbf{Vantaggi}:
            \begin{itemize}
                  \item NVM permette di installare e gestire più versioni di Node.js all'interno del sistema, permettendo di selezionare la versione corretta
                        per il progetto in corso;
                  \item NVM permette di gestire le versioni di Node.js in modo semplice e veloce, permettendo di passare da una versione all'altra con un solo comando.
            \end{itemize}
\end{itemize}

\subsubsection{Rollup}
\begin{itemize}
      \item \textbf{Nome strumento}: Rollup
      \item \textbf{Versione}: 4.18.0
      \item \textbf{Link}: \href{https://rollupjs.org/}{rollupjs.org}
      \item \textbf{Descrizione}: Rollup è un bundler di moduli JavaScript che permette di risolvere le dipendenze tra i moduli, generando un unico file di output.
      \item \textbf{Vantaggi}:
            \begin{itemize}
                  \item Rollup effettua il tree-shaking, rimuovendo le dipendenze non utilizzate all'interno del codice sorgente;
                  \item Permette di generare bundle in diversi formati, quali \textit{CommonJS} e \textit{ESM};
                  \item Rollup è noto per la sua velocità ed efficienza nelle build, specialmente per progetti più piccoli o librerie.
                        Questo può risultare in tempi di build più rapidi e in una migliore esperienza di sviluppo;
                  \item Rollup ha un sistema di plug-in molto potente e flessibile che permette di estendere le sue funzionalità;
                  \item Rollup permette di utilizzare TypeScript all'interno del progetto, integrando la configurazione dichiarata nel file \textit{tsconfig.json};
                  \item Rollup permette di generare file di dichiarazione TypeScript.
            \end{itemize}
\end{itemize}

\subsubsection{Yarn}
\begin{itemize}
      \item \textbf{Nome strumento}: Yarn
      \item \textbf{Versione}: 1.22.22
      \item \textbf{Link}: \href{https://yarnpkg.com/}{yarnpkg.com}
      \item \textbf{Descrizione}: Yarn è un package manager per JavaScript, sviluppato da Facebook, Google e Tilde.
      \item \textbf{Vantaggi}:
            \begin{itemize}
                  \item Yarn è più veloce di npm, ha un sistema di cache più efficiente e permette di installare pacchetti in parallelo.
            \end{itemize}
      \item \textbf{Svantaggi}:
            \begin{itemize}
                  \item Yarn presenta un registry di dimensione minore rispetto a quello di \textit{npm}.
            \end{itemize}
      \item \textbf{Alternative Esaminate}:
            \begin{itemize}
                  \item \textit{npm}: npm è il package manager di default per Node.js, ma è più lento di Yarn e ha un sistema di cache meno efficiente,
                        offrendo inoltre un output meno comprensibile.
            \end{itemize}
\end{itemize}

\subsection{Strumenti per la Collaborazione e la Gestione del Progetto}
Nella seguente sezione vengono presentati gli strumenti utilizzati per la collaborazione e la gestione del progetto durante lo svolgimento dello stage.

\subsubsection{Confluence}
\begin{itemize}
      \item \textbf{Nome strumento}: Confluence
      \item \textbf{Versione}: Cloud
      \item \textbf{Link}: \href{https://www.atlassian.com/software/confluence}{atlassian.com/software/confluence}
      \item \textbf{Descrizione}: Confluence è un software di collaborazione sviluppato da Atlassian, utilizzato per la creazione e la gestione della documentazione
            all'interno di un'azienda.
\end{itemize}

\subsubsection{Slack}
\begin{itemize}
      \item \textbf{Nome strumento}: Slack
      \item \textbf{Versione}: Cloud
      \item \textbf{Link}: \href{https://slack.com/}{slack.com}
      \item \textbf{Descrizione}: Slack è un software di collaborazione sviluppato da Slack Technologies, utilizzato per la comunicazione quotidiana e la collaborazione
            con il team.
\end{itemize}

\section{Descrizione del prodotto ottenuto}
Il prodotto conseguito al termine dello stage consiste in una libreria TypeScript di componenti utilizzati nel processo di creazione di dashboard dinamiche,
ottenuto a partire dal refactor e l'ottimizzazione della libreria open-source utilizzata all'interno della piattaforma \textit{Redash}.
La libreria implementata, sotto il nome di \textit{viz-lib}, è stata infatti integrata assieme alla già esistente libreria \textit{dashboard} all'interno del SDK
\textit{dsdashboard2}, utilizzato all'interno dei prodotti Datasoil S.r.l. per la generazione di dashboard dinamiche.
La libreria \textit{viz-lib} offre una vasta gamma di grafici, quali:
\begin{itemize}
      \item Contatori;
      \item Grafici ad area;
      \item Grafici a barre;
      \item Grafici a bolle;
      \item Grafici a dispersione;
      \item Grafici a istogrammi;
      \item Grafici a linee;
      \item Grafici a torta;
      \item Tabelle.
\end{itemize}
Grazie al refactor e all'ottimizzazione della libreria preesistente, l'SDK prodotto risulta essere più performante, più manutenibile e soprattutto più leggero rispetto
alla versione preesistente, con un bundle di dimensioni 1.26 MB → 418.03 kB (gzip) rispetto ai 4.11 MB → 1.203 MB (gzip) iniziali.

\section{Organizzazione del testo}
Nel presente capitolo viene presentata l'introduzione della tesi, fornendo una panoramica sull'azienda, sul contesto applicativo,
sul progetto, sugli strumenti utilizzati e sul prodotto portato a termine durante lo svolgimento dello stage. \newline
In seguito il documento presenterà la seguente organizzazione:

\begin{description}
      \item[{\hyperref[chap:analisi-requisiti]{Analisi dei requisiti}}] descrive la fase di analisi dei requisiti che è stata
            svolta dall'azienda in fase antecedente all'inizio dell'attività di stage, in modo da permettere una comprensione più profonda
            di quelle che sono le necessità da soddisfare e gli obiettivi da raggiungere all'interno di questo progetto;

      \item[{\hyperref[chap:progettazione]{Progettazione}}] illustra l'attività di progettazione che è stata svolta in vista dell'implementazione
            dei grafici prodotti, definendo e individuando le soluzioni implementative che sono state attuate durante la successiva attività di codifica;

      \item[{\hyperref[chap:realizzazione-testing]{Realizzazione e Testing}}] approfondisce l'attività di codifica delle componenti grafiche presentate
            all'interno della libreria implementata e dei relativi test effettuati per garantire la qualità e la funzionalità del prodotto finale;

      \item[{\hyperref[chap:rilascio]{Rilascio}}] descrive l'attività di rilascio dell'SDK realizzato, presentando le modalità di integrazione
            all'interno di un prodotto Datasoil S.r.l.;

      \item[{\hyperref[chap:conclusioni]{Conclusioni}}] presentano un epilogo del progetto svolto, includendo una valutazione oggettiva
            degli strumenti utilizzati e una riflessione sui possibili punti di insoddisfazione e dei relativi miglioramenti applicabili al prodotto realizzato.
            La sezione infine si conclude con proposte per future estensioni e sviluppi del progetto.
\end{description}
In merito alla stesura del testo, all'interno del presente documento sono state adottate le seguenti convenzioni tipografiche:
\begin{itemize}
      \item gli acronimi, le abbreviazioni e i termini ambigui o di uso non comune menzionati vengono definiti nel glossario, situato alla fine del presente documento;
      \item per la prima occorrenza dei termini riportati nel glossario viene utilizzata la seguente nomenclatura: \textit{parola}\glox\gloxspacing;
      \item i termini in lingua straniera o facenti parti del gergo tecnico sono evidenziati con il carattere \textit{corsivo}.
\end{itemize}


\newpage
\chapter{Analisi dei requisiti}
\label{chap:analisi-requisiti}
Nel seguente capito verrà descritta l'attività di analisi dei requisiti inerente al progetto di stage. \newline
In quanto l'SDK in questione è uno strumento live sui diversi prodotti Datasoil S.r.l., l'analisi dei requisiti è stata svolta
in fase antecedente all'inizio dello stage da parte del team di sviluppo frontend dell'azienda. \newline
A fronte di tale contesto, nella seguente sezione verranno riportati i requisiti individuati in modo da poter fornire il contesto
necessario alla comprensione del progetto, senza approfondire l'attività di analisi svolta.

\section{Classificazione dei requisiti}
I requisiti presentati all'interno del documento vengono classificati in tre categorie, in base alle esigenze che definiscono, venendo inoltre
associati ad un livello di priorità che ne determina l'obbligatorietà o l'opzionalità. \newline

\newline
Le categorie di classificazione definite all'interno del progetto di stage sono:

\begin{itemize}
    \item[\textbf{F}:] \textit{Funzionale}: il requisito definisce le funzionalità che il sistema deve offrire;
    \item[\textbf{A}:] \textit{Di attributo}: il requisito definisce le caratteristiche che il sistema deve avere;
    \item[\textbf{V}:] \textit{Di vincolo}: il requisito definisce i vincoli che il sistema deve rispettare.
\end{itemize}

\newline
I livelli di priorità definiti all'interno del progetto di stage sono:

\begin{itemize}
    \item[\textbf{O}:] \textit{Obbligatorio}: requisito essenziale al fine di concludere il progetto di stage con successo;
    \item[\textbf{OP}:] \textit{Opzionale}: requisito non essenziale al fine di concludere il progetto di stage con successo, ma che aggiunge valore al prodotto finale.
\end{itemize}

\newline
La nomeclatura utilizzata per la classificazione dei requisiti all'interno del progetto di stage è la seguente:
\begin{itemize}
    \item \textbf{R}: dove R sta per requisito;
    \item \textbf{X.}: dove X indica la categoria di classificazione del requisito;
    \item \textbf{Y.}: dove Y indica il livello di priorità del requisito;
    \item \textbf{Z}: dove Z indica il numero progressivo del requisito. Nel caso di sotto-requisiti, il numero progressivo è composto da due cifre, separate da un punto,
          dove la prima cifra rappresenta il requisito padre e la seconda cifra rappresenta il sotto-requisito.
\end{itemize}

\newpage
\section{Requisiti}

\subsection{Requisiti funzionali}
Nella seguente sezione viene presentata la lista dei requisiti funzionali.

\begin{center}
    \rowcolors{1}{white}{tableGray}
    \begin{longtable}{|p{2.5cm}|p{10cm}|}
        \hline
        \rowcolor{gray!30}
        \textbf{Requisito} & \multicolumn{1}{c|}{\textbf{Descrizione}}                                                          \\
        \hline
        \endfirsthead
        \hline
        \multicolumn{2}{|c|}{{\tablename\ \thetable{} -- continua dalla pagina precedente}}                                     \\
        \hline
        \rowcolor{gray!30}
        \textbf{Requisito} & \multicolumn{1}{c|}{\textbf{Descrizione}}                                                          \\
        \endhead
        \hline
        \multicolumn{2}{|r|}{{Continua nella prossima pagina...}}                                                               \\
        \hline
        \endfoot
        \hline
        \endlastfoot
        R.F.O.1            & La libreria deve permettere la visualizzazione di grafici.                                         \\
        \hline
        R.F.O.1.1          & La libreria deve permettere la visualizzazione di grafici a barre (\textit{bar or columns}).       \\
        \hline
        R.F.O.1.2          & La libreria deve permettere la visualizzazione di grafici a torta (\textit{pie}).                  \\
        \hline
        R.F.O.1.3          & La libreria deve permettere la visualizzazione di grafici a scatola (\textit{box-plot}).           \\
        \hline
        R.F.O.1.4          & La libreria deve permettere la visualizzazione di grafici a bolle (\textit{bubbles}).              \\
        \hline
        R.F.O.1.5          & La libreria deve permettere la visualizzazione di grafici a dispersione (\textit{scatter}).        \\
        \hline
        R.F.O.1.6          & La libreria deve permettere la visualizzazione di grafici a mappe di calore (\textit{heatmap}).    \\
        \hline
        R.F.O.2            & La libreria deve permettere la visualizzazione di contatori.                                       \\
        \hline
        R.F.O.2.1          & La libreria deve permettere la visualizzazione di contatori con target.                            \\
        \hline
        R.F.O.2.2          & La libreria deve permettere la visualizzazione di contatori senza target.                          \\
        \hline
        R.F.O.3            & La libreria deve permettere la visualizzazione di tabelle.                                         \\
        \hline
        R.F.O.4            & La libreria deve permettere di zoomare i vari grafici.                                             \\
        \hline
        R.F.O.5            & La libreria deve permettere di effettuare il download in \textit{.png} dei grafici renderizzati.   \\
        \hline
        R.F.O.6            & La libreria deve permettere di effettuare il download in \textit{.csv} delle tabelle visualizzate. \\
        \hline
        R.F.O.7            & I widget renderizzati devono adattarsi alla dimensione del contenitore in modo responsivo.         \\
    \end{longtable}
    \captionof{table}{Tabella del tracciamento dei requisiti funzionali.}
    \label{tab:requisiti_funzionali}
\end{center}

\subsection{Requisiti qualitativi}
Nella seguente sezione viene presentata la lista dei requisiti qualitativi.

\begin{center}
    \rowcolors{1}{white}{tableGray}
    \begin{longtable}{|p{2.5cm}|p{10cm}|}
        \hline
        \rowcolor{gray!30}
        \textbf{Requisito} & \multicolumn{1}{c|}{\textbf{Descrizione}}                                                                                    \\
        \hline
        \endfirsthead
        \hline
        \multicolumn{2}{|c|}{{\tablename\ \thetable{} -- continua dalla pagina precedente}}                                                               \\
        \hline
        \rowcolor{gray!30}
        \textbf{Requisito} & \multicolumn{1}{c|}{\textbf{Descrizione}}                                                                                    \\
        \endhead
        \hline
        \multicolumn{2}{|r|}{{Continua nella prossima pagina...}}                                                                                         \\
        \hline
        \endfoot
        \hline
        \endlastfoot
        \hline
        R.A.O.1            & La libreria deve essere facilmente manutenibile, rispettando le convenzioni di codifica
        attualmente in atto all'interno dell'azienda Datasoil S.r.l.                                                                                      \\
        \hline
        R.A.O.2            & La libreria deve essere performante in termini di velocità di rendering dei widget forniti.                                  \\
        \hline
        R.A.O.3            & La libreria deve essere ottimizzata per richiedere il minor numero di rendering possibili durante l'interazione dell'utente,
        eseguendoli in modo efficiente per minimizzare l'uso delle risorse.                                                                               \\
        \hline
        R.A.O.3            & La libreria deve essere leggera in termini di dimensione di memoria occupata.                                                \\
        \hline
        R.A.O.3.1          & Le dipendenze esterne devono essere ridotte al minimo indispensabile.                                                        \\
        \hline
        R.A.0.3.2          & La libreria deve essere composta esclusivamente da codice effettivamente utilizzato per le funzionalità e i widget resi
        disponibili. Ciò comporta che qualsiasi componente di codice inutilizzata, superflua o non necessaria deve essere identificata e rimossa.         \\
    \end{longtable}
    \captionof{table}{Tabella del tracciamento dei requisiti qualitativi.}
    \label{tab:requisiti_qualitativi}
\end{center}

\subsection{Requisiti di vincolo}
Nella seguente sezione viene presentata la lista dei requisiti di vincolo.

\begin{center}
\rowcolors{1}{white}{tableGray}
\begin{longtable}{|p{2.5cm}|p{10cm}|}
    \hline
    \rowcolor{gray!30}
    \textbf{Requisito} & \multicolumn{1}{c|}{\textbf{Descrizione}}                                                 \\
    \hline
    \endfirsthead
    \hline
    \multicolumn{2}{|c|}{{\tablename\ \thetable{} -- continua dalla pagina precedente}}                            \\
    \hline
    \rowcolor{gray!30}
    \textbf{Requisito} & \multicolumn{1}{c|}{\textbf{Descrizione}}                                                 \\
    \endhead
    \hline
    \multicolumn{2}{|r|}{{Continua nella prossima pagina...}}                                                      \\
    \hline
    \endfoot
    \hline
    \endlastfoot
    \hline
    R.V.O.1            & La libreria deve essere sviluppata in \textit{TypeScript}.                                \\
    \hline
    R.V.O.2            & La libreria deve essere sviluppata con \textit{React}.                                    \\
    \hline
    R.V.O.3            & La libreria deve essere sviluppata con l'utilizzo di componenti di \textit{PrimeReact}.   \\
    \hline
    R.V.O.4            & La libreria deve essere sviluppata con l'utilizzo di \textit{PrimeFlex}.                  \\
    \hline
    R.V.O.5            & La libreria deve essere sviluppata con l'utilizzo di \textit{PrimeIcons}.                 \\
    \hline
    R.V.O.6            & Il codice della libreria deve essere formattato mediante l'utilizzo di \textit{Prettier}. \\
\end{longtable}
\captionof{table}{Tabella del tracciamento dei requisiti di vincolo.}
\label{tab:requisiti_vincolo}

\chapter{Progettazione}
\label{chap:progettazione}
Nel presente capitolo verrà presentata l'attività di progettazione relativa al progetto di stage focalizzato sul refactoring e
sull'ottimizzazione dello SDK impiegato dall'azienda \textit{Datasoil S.r.l.} per lo sviluppo di dashboard dinamiche.
Verrà analizzato l'approccio adottato per individuare le soluzioni volte al soddisfacimento dei requisiti individuati durante
la fase di analisi dei requisiti, proseguendo con la descrizione dell'architettura progettata per la realizzazione della libreria grafica \textit{viz-lib}
appartenente al \textit{development kit}, concludendo con la definizione dei tipi e delle interfacce e la progettazione delle singole componenti individuate.

\section{Metodologia di progettazione}
La metodologia di progettazione adottata durante lo svolgimento del progetto di stage ha seguito inizialmente un approccio
\textit{top-down}, che permettesse, attraverso uno studio generale della libreria preesistente e della sua architettura, di individuare
il flusso di dati tra le varie componenti, fornendo una visione globale delle interazioni e delle dipendenze esistenti.
Successivamente, è stato adottato un approccio \textit{bottom-up}, analizzando le singole componenti e le loro funzionalità,
in modo da identificare le possibili ottimizzazioni e le modifiche necessarie per il miglioramento delle prestazioni, della leggibilità e
della manutenibilità del codice.\newline
L'attività di progettazione è stata condotta in collaborazione stretta con il tutor aziendale, il quale ha fornito supporto e orientamento
riguardo alle decisioni progettuali individuate.

\section{Design dell'architettura}
L'architettura progettata per la realizzazione della libreria grafica si basa su un pattern comunemente utilizzato nello sviluppo di \textit{packages}
di componenti \textit{frontend} per \textit{React}, ovvero il pattern \textit{Component Library Architecture}: questa architettura si basa sulla creazione di una libreria di
componenti riutilizzabili, modulari e indipendenti tra loro. \newline
Tale pattern permette di creare l'architettura ideale per librerie che offrono collezioni di componenti riutilizzabili, permettendo e agevolando
la condivisione e il riutilizzo di codice all'interno del progetto, garantendo una maggior manutenibilità del codice. \newline
Il modello proposto prevede la seguente struttura, la quale si riflette nella definizione delle \textit{macro-directory} all'interno del progetto:
\begin{itemize}
      \item \textbf{components}: \textit{directory} contenente i componenti della libreria grafica;
      \item \textbf{hooks}: \textit{directory} contenente i \textit{custom hooks} utilizzati all'interno della libreria grafica;
      \item \textbf{utils}: \textit{directory} contenente le funzioni di utilità utilizzate all'interno della libreria grafica;
      \item \textbf{models}: \textit{directory} contenente le definizioni dei tipi e delle interfacce utilizzate all'interno della libreria grafica;
      \item \textbf{index.ts}: file principale della libreria grafica, contenente l'esportazione di tutti i componenti e le funzioni utilizzate;
      \item \textbf{style.css}: file contenente i fogli di stile globali utilizzati all'interno della libreria grafica.
\end{itemize}
L'\textit{entry point} della libreria grafica è il file \textit{index.ts}, il quale si occupa di esportare il \texttt{Renderer}, il componente principale
della liberia che si occupa di renderizzare il corretto componente a seconda del tipo di visualizzazione richiesto: tali informazioni necessarie
vengono passate come \textit{props} dalle componenti del modulo \textit{dashboard} che utilizzano il \texttt{Renderer}, riducendo così la dipendenza tra le
componenti di librerie differenti, agevolando la manutenibilità e la scalabilità del codice.


\section{Progettazione dei tipi e delle interfacce}
Nella seguente sezione vengono presentati i tipi e le interfacce definiti durante l'attività di progettazione, associati ad una descrizione dettagliata
delle informazioni che rappresentano. \\
La definizione dei tipi e delle interfacce ha costituito un ruolo fondamentale durante l'attività di progettazione, consentendo di raggiungere una maggior
leggibilità e manutenibilità del codice, permettendo di usufruire a pieno dei vantaggi offerti da \textit{TypeScript}. \\
I tipi e le interfacce verranno presentati in ordine alfabetico.

\subsection{ChartBaseVisualizationOptions}
Interfaccia che definisce i tipi dei dati relativi alle opzioni di visualizzazione comuni a tutti i \texttt{Chart}.
\begin{minted}{typescript}
export interface ChartBaseVisualizationOptions {
      globalSeriesType: string; 
      sortX: boolean;
      sortY?: boolean;
      legend: {
      enabled: boolean;
      placement: string;
      traceorder: 'grouped'|'normal'|'reversed'|'reversed+grouped';
      };
      xAxis: {
      labels: {
            enabled: boolean;
      };
      type: string;
      title?: {
            text: string;
      };
      };
      yAxis: [
      {
            type: string;
            rangeMax?: number;
            rangeMin?: number;
            title?: {
            text?: string;
            };
      },
      {
            opposite: boolean;
            type: string;
      },
      ];
      alignYAxesAtZero: boolean;
      error_y: {
      type: string;
      visible: boolean;
      };
      series: {
      error_y: {
            type: string;
            visible: boolean;
      };
      stacking: string;
      percentValues?: boolean;
      };
      seriesOptions: {
      [x: string]: any;
      };
      valuesOptions: {
      [x: string]: any;
      };
      columnMapping: { [x: string]: string };
      direction: {
      type: string;
      };
      sizemode: string;
      coefficient: number;
      showDataLabels: boolean;
      numberFormat: string;
      dateTimeFormat: string;
      percentFormat: string;
      textFormat: string;
      missingValuesAsZero: boolean;
      onHover?: () => void;
      onUnHover?: () => void;
      reverseX?: boolean;
      reverseY?: boolean;
      showpoints?: boolean;
      heatMinColor?: string;
      heatMaxColor?: string;
      colorScheme: string | (string | number)[][];
}
\end{minted}
\begin{listing}[H]
      \caption{Definizione dell'interfaccia \texttt{ChartBaseVisualizationOptions}}
      \label{listing:chartBaseVisualizationOptions}
\end{listing}
\begin{itemize}
      \item \textbf{globalSeriesType}: stringa che definisce il tipo di \textit{chart} da renderizzare. I valori possibili sono i seguenti:
            \textit{area, column, box, bubble, heatmap, line, pie} e \textit{scatter};
      \item \textbf{sortX}: booleano che definisce se ordinare o meno le serie sull'asse delle ascisse;
      \item \textbf{sortY}: booleano che definisce se ordinare o meno le serie sull'asse delle ordinate;
      \item \textbf{legend}: oggetto che definisce le opzioni della legenda del grafico:
            \begin{itemize}
                  \item \textbf{enabled}: booleano che definisce se abilitare o meno la legenda;
                  \item \textbf{placement}: stringa che definisce la posizione della legenda all'interno del grafico;
                  \item \textbf{traceorder}: stringa che definisce l'ordine delle tracce all'interno del grafico.
            \end{itemize}
      \item \textbf{xAxis}: oggetto che definisce le opzioni dell'asse delle ascisse.
            \begin{itemize}
                  \item \textbf{labels}: oggetto che definisce le opzioni delle label dell'asse delle ascisse;
                  \item \textbf{type}: stringa che definisce il tipo di dato visualizzato sull'asse delle ascisse;
                  \item \textbf{title}: oggetto che definisce il titolo dell'asse delle ascisse.
            \end{itemize}
      \item \textbf{yAxis}: array che definisce le opzioni dell'asse delle ordinate. Ogni elemento dell'array è un oggetto che definisce le opzioni per una serie di dati. \newline
            Il primo oggetto definisce i valori per la serie principale:
            \begin{itemize}
                  \item \textbf{type}: stringa che definisce il tipo di dato visualizzato sull'asse delle ordinate;
                  \item \textbf{rangeMax}: numero che definisce il valore massimo dell'asse delle ordinate;
                  \item \textbf{rangeMin}: numero che definisce il valore minimo dell'asse delle ordinate;
                  \item \textbf{title}: oggetto che definisce il titolo dell'asse delle ordinate.
            \end{itemize}
            Il secondo oggetto definisce i valori per la serie secondaria:
            \begin{itemize}
                  \item \textbf{opposite}: booleano che definisce se la serie secondaria è opposta alla serie principale;
                  \item \textbf{type}: stringa che definisce il tipo di dato visualizzato sull'asse delle ordinate.
            \end{itemize}
      \item \textbf{alignYAxesAtZero}: booleano che definisce se allineare le ordinate a zero;
      \item \textbf{error\_y}: oggetto che definisce le opzioni degli errori sull'asse delle ordinate.
            \begin{itemize}
                  \item \textbf{type}: stringa che definisce il tipo di errore;
                  \item \textbf{visible}: booleano che definisce se visualizzare o meno gli errori.
            \end{itemize}
      \item \textbf{series}: oggetto che definisce le opzioni delle serie.
            \begin{itemize}
                  \item \textbf{error\_y}: oggetto che definisce le opzioni degli errori sull'asse delle ordinate;
                  \item \textbf{stacking}: stringa che definisce se le serie sono sovrapposte o impilate;
                  \item \textbf{percentValues}: booleano che definisce se visualizzare i valori in percentuale.
            \end{itemize}
      \item \textbf{seriesOptions}: oggetto che definisce le opzioni delle serie;
      \item \textbf{valuesOptions}: oggetto che definisce le opzioni dei valori;
      \item \textbf{columnMapping}: oggetto che definisce il \textit{mapping} delle colonne;
      \item \textbf{direction}: oggetto che definisce tramite \textit{type} la direzione del grafico;
      \item \textbf{sizemode}: stringa che definisce la modalità di dimensionamento;
      \item \textbf{coefficient}: numero che definisce il coefficiente di dimensionamento;
      \item \textbf{showDataLabels}: booleano che definisce se visualizzare o meno le label dei dati;
      \item \textbf{numberFormat}: stringa che definisce il formato dei numeri;
      \item \textbf{dateTimeFormat}: stringa che definisce il formato della data;
      \item \textbf{percentFormat}: stringa che definisce il formato percentuale;
      \item \textbf{textFormat}: stringa che definisce il formato del testo;
      \item \textbf{missingValuesAsZero}: booleano che definisce se i valori mancanti devono essere considerati come zero;
      \item \textbf{onHover}: funzione che definisce l'azione da eseguire al passaggio del mouse;
      \item \textbf{onUnHover}: funzione che definisce l'azione da eseguire al passaggio del mouse;
      \item \textbf{reverseX}: booleano che definisce se invertire l'asse delle ascisse;
      \item \textbf{reverseY}: booleano che definisce se invertire l'asse delle ordinate;
      \item \textbf{showpoints}: booleano che definisce se visualizzare o meno i punti;
      \item \textbf{heatMinColor}: stringa che definisce il colore minimo del grafico \textit{heatmap};
      \item \textbf{heatMaxColor}: stringa che definisce il colore massimo del grafico \textit{heatmap};
      \item \textbf{colorScheme}: stringa o array che definisce lo schema dei colori del grafico \textit{heatmap}.
\end{itemize}

\subsection{Column}
Interfaccia che definisce i tipi dei dati relativi alle colonne della tabella.

\begin{minted}{typescript}
export interface Column {
      title: string;
      name: string;
      type: string;
      visible: boolean;
      displayAs: string;
      header?: any;
      alignContent?: string;
      allowHTML?: boolean; 
      allowSearch?: boolean;
      booleanValues?: string[];
      highlightLinks?: boolean;
      imageTitleTemplate?: string;
      imageUrlTemplate?: string;
      imageWidth?: string;
      imageHeight?: string;
      linkOpenInNewTab?: boolean;
      linkTextTemplate?: string;
      linkTitleTemplate?: string;
      linkUrlTemplate?: string;
      numberFormat?: string;
      dateTimeFormat?: string;
      order?: number;
}
\end{minted}
\begin{listing}[H]
      \caption{Definizione dell'interfaccia \texttt{Column}}
      \label{listing:column}
\end{listing}
\begin{itemize}
      \item \textbf{title}: stringa che definisce il titolo della colonna;
      \item \textbf{name}: stringa che definisce il nome della colonna;
      \item \textbf{type}: stringa che definisce il tipo di dato della colonna;
      \item \textbf{visible}: booleano che definisce se la colonna è visibile o meno;
      \item \textbf{displayAs}: stringa che definisce il tipo di visualizzazione della colonna;
      \item \textbf{header}: oggetto che definisce l'header della colonna;
      \item \textbf{alignContent}: stringa che definisce l'allineamento del contenuto della colonna;
      \item \textbf{allowHTML}: booleano che definisce se permettere o meno l'utilizzo di HTML;
      \item \textbf{allowSearch}: booleano che definisce se permettere o meno la ricerca;
      \item \textbf{booleanValues}: array di stringhe che definisce i valori booleani;
      \item \textbf{highlightLinks}: booleano che definisce se evidenziare i link;
      \item \textbf{imageTitleTemplate}: stringa che definisce il template del titolo dell'immagine;
      \item \textbf{imageUrlTemplate}: stringa che definisce il template dell'URL dell'immagine;
      \item \textbf{imageWidth}: stringa che definisce la larghezza dell'immagine;
      \item \textbf{imageHeight}: stringa che definisce l'altezza dell'immagine;
      \item \textbf{linkOpenInNewTab}: booleano che definisce se aprire il link in una nuova scheda;
      \item \textbf{linkTextTemplate}: stringa che definisce il template del testo del link;
      \item \textbf{linkTitleTemplate}: stringa che definisce il template del titolo del link;
      \item \textbf{linkUrlTemplate}: stringa che definisce il template dell'URL del link;
      \item \textbf{numberFormat}: stringa che definisce il formato numerico;
      \item \textbf{dateTimeFormat}: stringa che definisce il formato della data;
      \item \textbf{order}: numero che definisce l'ordine della colonna.
\end{itemize}

\subsection{CounterVisualizationOptions}
Interfaccia che definisce i tipi dei dati relativi alle opzioni di visualizzazione del componente \texttt{Counter}.
\begin{listing}[H]
      \begin{minted}{typescript}
      export interface CounterBaseVisualizationOptions {
            counterLabel: string;
            counterColName: string;
            targetColName?: string;
            rowNumber: number;
            targetRowNumber: number;
            countRows?: any;
            stringDecimal: number;
            stringDecChar: string;
            stringThouSep: string;
            tooltipFormat: string;
            stringPrefix?: string;
            stringSuffix?: string;
            formatTargetValue?: boolean;
      }
      \end{minted}
      \caption{Definizione dell'interfaccia \texttt{CounterVisualizationOptions}}
      \label{listing:counterVisualizationOptions}
\end{listing}
\begin{itemize}
      \item \textbf{counterLabel}: stringa che definisce l'etichetta del contatore;
      \item \textbf{counterColName}: stringa che definisce il nome della colonna contenente i valori del contatore;
      \item \textbf{targetColName}: stringa che definisce il nome della colonna contenente il valore target del contatore;
      \item \textbf{rowNumber}: numero che definisce il numero di righe da considerare per il calcolo del contatore;
      \item \textbf{targetRowNumber}: numero che definisce il numero di righe da considerare per il calcolo del valore target;
      \item \textbf{countRows}: oggetto che definisce le righe da considerare per il calcolo del contatore;
      \item \textbf{stringDecimal}: numero che definisce il numero di cifre decimali da visualizzare;
      \item \textbf{stringDecChar}: stringa che definisce il separatore decimale;
      \item \textbf{stringThouSep}: stringa che definisce il separatore delle migliaia;
      \item \textbf{tooltipFormat}: stringa che definisce il formato del \textit{tooltip};
      \item \textbf{stringPrefix}: stringa che definisce il prefisso del contatore;
      \item \textbf{stringSuffix}: stringa che definisce il suffisso del contatore;
      \item \textbf{formatTargetValue}: booleano che definisce se formattare o meno il valore target.
\end{itemize}

\subsection{DataLabels}
Interfaccia che definisce i tipi dei dati relativi alle label dei dati utilizzati negli \textit{heatmaps}.

\begin{minted}{typescript}
export interface DataLabels {
  x: any[];
  y: any[];
  mode: string;
  hoverinfo: string;
  showlegend: boolean;
  text: string[];
  textfont: {
    color: string[];
  };
}
\end{minted}
\begin{listing}[H]
      \caption{Definizione dell'interfaccia \texttt{DataLabels}}
      \label{listing:dataLabels}
\end{listing}
\begin{itemize}
      \item \textbf{x}: array che definisce i valori sull'asse delle ascisse;
      \item \textbf{y}: array che definisce i valori sull'asse delle ordinate;
      \item \textbf{mode}: stringa che definisce la modalità;
      \item \textbf{hoverinfo}: stringa che definisce le informazioni del \textit{tooltip};
      \item \textbf{showlegend}: booleano che definisce se visualizzare o meno la legenda;
      \item \textbf{text}: array di stringhe che definisce il testo;
      \item \textbf{textfont}: oggetto che definisce le opzioni del font del testo:
            \begin{itemize}
                  \item \textbf{color}: array di stringhe che definisce i colori del font.
            \end{itemize}
\end{itemize}

\subsection{PlotlyOptions}
Interfaccia che definisce i tipi dei dati relativi alle opzioni di visualizzazione di \textit{Plotly.js}.
\begin{minted}{typescript}
export interface PlotlyOptions {
  showLink: boolean;
  displaylogo: boolean;
  displayModeBar?: boolean;
  responsive?: boolean;
  autosize?: boolean;
}
\end{minted}
\begin{listing}[H]
      \caption{Definizione dell'interfaccia \texttt{PlotlyOptions}}
      \label{listing:plotlyOptions}
\end{listing}
\begin{itemize}
      \item \textbf{showLink}: booleano che definisce se visualizzare o meno il link;
      \item \textbf{displaylogo}: booleano che definisce se visualizzare o meno il logo;
      \item \textbf{displayModeBar}: booleano che definisce se visualizzare o meno la barra di \textit{Plotly};
      \item \textbf{responsive}: booleano che definisce se rendere o meno il grafico responsivo;
      \item \textbf{autosize}: booleano che definisce se adattare o meno la grandezza del grafico.
\end{itemize}

\subsection{RendererProps}
Interfaccia che definisce i tipi dei dati passati alla componenti \texttt{Renderer, Counter, Chart} e \texttt{Table}.
\begin{listing}[H]
      \begin{minted}{typescript}
      export interface RendererProps {
            type: string;
            visualizationName: string;
            data: {
                  [x: string]: any;
            };
            options: VisualizationOptions;
      }
      \end{minted}
      \caption{Definizione dell'interfaccia \texttt{RendererProps}}
      \label{listing:rendererProps}
\end{listing}
\begin{itemize}
      \item \textbf{type}: stringa che definisce il tipo componente da renderizzazre. Può assumere i seguenti valori: \textit{COUNTER}, \textit{CHART} e \textit{TABLE};
      \item \textbf{visualizationName}: stringa che definisce il titolo di visualizzazione della componente;
      \item \textbf{data}: oggetto che contiene i dati da visualizzare all'interno della componente;
      \item \textbf{options}: oggetto che contiene le opzioni di visualizzazione della componente di tipo \textit{VisualizationOptions}.
\end{itemize}

\subsection{Series}
Interfaccia che definisce i tipi dei dati relativi alle serie rappresentata nel grafico.
\begin{minted}{typescript}
export interface Series {
  name: string;
  type: string;
  data?: any[];
  visible?: boolean;
  values?: number[];
  labels?: string[];
  hole?: number;
  marker?: {
    colors?: string[];
    color?: string; 
    line?: {
      color?: string;
      width?: number;
    };
    size?: number;
    sizemode?: string;
  };
  hoverinfo?: string | boolean;
  text?: any[];
  textinfo?: string;
  textposition?: string;
  textangle?: number;
  textfont?: {
    size?: number;
    color?: string[];
  };
  direction?: string;
  domain?: {
    x: number[];
    y: number[];
  };
  sourceData?: Map<string, any>;
  hoverlabel?: {
    font?: {
      color?: string[];
    };
    bordercolor?: string;
    bgcolor?: string;
  };
  hover?: any[];
  x?: any;
  y?: any;
  z?: any;
  yaxis?: any;
  orientation?: number;
  insidetextfont?: {
    size?: number;
    color?: string;
  };
  error_y?: any;
  offsetgroup?: string;
  mode?: string;
  fill?: string;
  boxpoints?: string;
  jitter?: number;
  pointpos?: number;
  colorscale?: string | (string | number)[][];
  xgap?: number;
  ygap?: number;
}
\end{minted}
\begin{listing}[H]
      \caption{Definizione dell'interfaccia \texttt{Series}}
      \label{listing:series}
\end{listing}
\begin{itemize}
      \item \textbf{name}: stringa che definisce il nome della serie;
      \item \textbf{type}: stringa che definisce il tipo di serie;
      \item \textbf{data}: array che definisce i dati della serie;
      \item \textbf{visible}: booleano che definisce se la serie è visibile o meno;
      \item \textbf{values}: array di numeri che definisce i valori della serie;
      \item \textbf{labels}: array di stringhe che definisce le label della serie;
      \item \textbf{hole}: numero che definisce, nel caso di grafico a torta, il raggio del buco centrale;
      \item \textbf{marker}: oggetto che definisce le opzioni del \textit{marker};
      \item \textbf{hoverinfo}: stringa o booleano che definisce le informazioni del \textit{tooltip};
      \item \textbf{text}: array che definisce il testo della serie;
      \item \textbf{textinfo}: stringa che definisce le informazioni del testo;
      \item \textbf{textposition}: stringa che definisce la posizione del testo;
      \item \textbf{textangle}: numero che definisce l'angolo del testo;
      \item \textbf{textfont}: oggetto che definisce le opzioni del font del testo;
      \item \textbf{direction}: stringa che definisce la direzione della serie;
      \item \textbf{domain}: oggetto che definisce il dominio della serie;
      \item \textbf{sourceData}: mappa che definisce i dati sorgente;
      \item \textbf{hoverlabel}: oggetto che definisce le opzioni del \textit{tooltip};
      \item \textbf{hover}: array che definisce le opzioni del \textit{tooltip};
      \item \textbf{x}: array che definisce i valori sull'asse delle ascisse;
      \item \textbf{y}: array che definisce i valori sull'asse delle ordinate;
      \item \textbf{z}: array che definisce i valori sull'asse z;
      \item \textbf{yaxis}: array che definisce i valori sull'asse delle ordinate;
      \item \textbf{orientation}: numero che definisce l'orientamento;
      \item \textbf{insidetextfont}: oggetto che definisce le opzioni del font interno;
      \item \textbf{error\_y}: oggetto che definisce le opzioni degli errori sull'asse delle ordinate;
      \item \textbf{offsetgroup}: stringa che definisce il gruppo di \textit{offset};
      \item \textbf{mode}: stringa che definisce la modalità;
      \item \textbf{fill}: stringa che definisce il riempimento;
      \item \textbf{boxpoints}: stringa che definisce i punti del \textit{box};
      \item \textbf{jitter}: numero che definisce il \textit{jitter} (variazione);
      \item \textbf{pointpos}: numero che definisce la posizione del punto;
      \item \textbf{colorscale}: stringa o array che definisce lo schema dei colori;
      \item \textbf{xgap}: numero che definisce il gap sull'asse delle ascisse;
      \item \textbf{ygap}: numero che definisce il gap sull'asse delle ordinate;
      \item \textbf{colors}: array di stringhe che definisce i colori del \textit{marker};
      \item \textbf{color}: stringa che definisce il colore del \textit{marker};
      \item \textbf{line}: oggetto che definisce le opzioni della linea del \textit{marker};
      \item \textbf{size}: numero che definisce la grandezza del \textit{marker};
      \item \textbf{sizemode}: stringa che definisce la modalità di dimensionamento del \textit{marker}.
      \item \textbf{font}: oggetto che definisce le opzioni del font del \textit{marker};
      \item \textbf{bordercolor}: stringa che definisce il colore del bordo del \textit{marker};
      \item \textbf{bgcolor}: stringa che definisce il colore di sfondo del \textit{marker};
      \item \textbf{colors}: array di stringhe che definisce i colori delle rappresentazioni (multiple);
      \item \textbf{size}: numero che definisce la grandezza del font;
      \item \textbf{color}: stringa che definisce il colore della rappresentazione (singola).
\end{itemize}

\subsection{TableBaseVisualizationOptions}
Interfaccia che definisce i tipi dei dati relativi alle opzioni di visualizzazione del componente \texttt{Table}.
\begin{listing}[H]
      \begin{minted}{typescript}
      export interface TableBaseVisualizationOptions {
            columns?: Column[];
            itemsPerPage?: number;
            paginationSize?: string;
      }
      \end{minted}
      \caption{Definizione dell'interfaccia \texttt{TableBaseVisualizationOptions}}
      \label{listing:tableBaseVisualizationOptions}
\end{listing}
\begin{itemize}
      \item \textbf{columns}: array di oggetti che definiscono le colonne della tabella;
      \item \textbf{itemsPerPage}: numero che definisce il numero di elementi per pagina;
      \item \textbf{paginationSize}: stringa che definisce la grandezza della paginazione.
\end{itemize}

\subsection{VisualizationOptions}
Tipo che definisce le opzioni di visualizzazione delle componenti.
\begin{listing}[H]
      \begin{minted}{typescript}
      export type VisualizationOptions = 
            | CounterBaseVisualizationOptions 
            | ChartBaseVisualizationOptions 
            | TableBaseVisualizationOptions;
      \end{minted}
      \caption{Definizione del tipo \texttt{VisualizationOptions}}
      \label{listing:visualizationOptions}
\end{listing}
Questa definizione permette di definire le opzioni di visualizzazione specifiche per ciascun tipo di componente, garantendo una maggiore flessibilità
nella definizione di \texttt{RendererProps}.

\subsection{VisualizationSettings}
Interfaccia che definisce i tipi dei dati relativi alle impostazioni di visualizzazione delle componenti.
\begin{listing}[H]
      \begin{minted}{typescript}
      interface VisualizationSettings {
            dateFormat: string;
            dateTimeFormat: string;
            integerFormat: string;
            floatFormat: string;
            booleanValues: string[];
            tableCellMaxJSONSize: number;
            allowCustomJSVisualizations: boolean;
            hidePlotlyModeBar: boolean;
            choroplethAvailableMaps: any;
      }
      \end{minted}
      \caption{Definizione dell'interfaccia \texttt{VisualizationSettings}}
      \label{listing:visualizationSettings}
\end{listing}
\begin{itemize}
      \item \textbf{dateFormat}: stringa che definisce il formato della data;
      \item \textbf{dateTimeFormat}: stringa che definisce il formato della data e dell'ora;
      \item \textbf{integerFormat}: stringa che definisce il formato degli interi;
      \item \textbf{floatFormat}: stringa che definisce il formato dei numeri decimali;
      \item \textbf{booleanValues}: array di stringhe che definisce i valori booleani;
      \item \textbf{tableCellMaxJSONSize}: numero che definisce la grandezza massima delle celle della tabella;
      \item \textbf{allowCustomJSVisualizations}: booleano che definisce se permettere o meno le visualizzazioni personalizzate;
      \item \textbf{hidePlotlyModeBar}: booleano che definisce se nascondere o meno la barra di Plotly;
      \item \textbf{choroplethAvailableMaps}: oggetto che definisce le mappe disponibili per il choropleth.
\end{itemize}

\subsection{VisualizationType}
Enumerazione che definisce i possibili valori dei tipi di visualizzazione supportati dalla libreria grafica.
\begin{listing}[H]
      \begin{minted}{typescript}
      export enum VisualizationType {
            Counter = 'COUNTER',
            Chart = 'CHART',
            Table = 'TABLE',
            None = 'NONE',
      }
      \end{minted}
      \caption{Definizione dell'enumerazione \texttt{VisualizationType}}
      \label{listing:visualizationType}
\end{listing}
I possibili valori dell'enumerazione sono i seguenti:
\begin{itemize}
      \item \textbf{Counter}: componente che visualizza un contatore;
      \item \textbf{Chart}: componente che visualizza un grafico;
      \item \textbf{Table}: componente che visualizza una tabella;
      \item \textbf{None}: valore di default, utilizzato per la gestione di errori o situazioni non previste.
\end{itemize}

\section{Progettazione delle componenti}
Nella seguente sezione verranno presentate le componenti individuate durante l'attività di progettazione, con una descrizione dettagliata delle
funzionalità offerte e delle possibili ottimizzazioni individuate. \newline
L'attività di progettazione è avvenuta seguendo il \textit{single responsability principle}, invididuando componenti operabili all'interno
del design pattern \textit{Component Composition} proprio di \textit{React}, il quale prevede la combinazione di composizioni di componenti più piccole
per ottenere componenti più complesse, favorendo il riutilizzo, la modularità e la manutenibilità del codice. \newline
Le componenti verranno presentate in ordine alfabetico, secondo la seguente struttura:
\begin{itemize}
      \item \textbf{Nome della componente}: breve descrizione della componente;
      \item \textbf{Descrizione}: descrizione dettagliata delle funzionalità offerte dalla componente;
      \item \textbf{Props}: definizione delle props utilizzate dalla componente;
      \item \textbf{Ottimizzazioni}: possibili ottimizzazioni individuate per la componente.
\end{itemize}

\subsection{Counter}
\begin{itemize}
      \item \textbf{Nome della componente}: \texttt{Counter};
      \item \textbf{Descrizione}: La componente \texttt{Counter} prevede la visualizzazione di un contatore, il cui valore viene definito a partire dai dati passati
            come \textit{props} tra le varie componenti e forniti dalla risposta in formato \textit{JSON} ricevuta dai \textit{server Datasoil}. \newline
            La componente permette inoltre di confrontare il valore attuale del contatore con un valore target (opzionale) da raggiungere, anch'esso ricavato
            dalla risposta in formato \textit{JSON} ricevuta dal \textit{server}. \newline
            Il valore del contatore può essere formattato in base alle preferenze definite lato backend, presentando le seguenti opzioni:
            \begin{itemize}
                  \label{counter-format}
                  \item Prefisso: stringa da aggiungere prima del valore del contatore;
                  \item Suffisso: stringa da aggiungere dopo il valore del contatore;
                  \item Numero di cifre decimali: numero di cifre decimali da visualizzare nel valore del contatore;
                  \item Separatore delle migliaia: carattere da utilizzare come separatore delle migliaia;
                  \item Separatore decimale: carattere da utilizzare come separatore decimale.
            \end{itemize}
            (Queste formattazioni sono applicabili anche al valore target e ai \textit{tooltip} visualizzati al passaggio del mouse sul contatore).\\
            L'implementazione delle formattazioni numeriche prevede l'utilizzo della libreria \textit{numbro.js}.
      \item \textbf{Props}: I \textit{props} della componente \texttt{Counter} sono definiti dall'interfaccia \\
            \texttt{RendererProps}, su cui il componente ne effettua il \textit{picking}
            (ovvero la selezione) delle informazioni \textit{data, options} e \textit{visualizationName}.
            \begin{listing}[H]
                  \begin{minted}{typescript}
                  export function Counter({
                        data,
                        options,
                        visualizationName,
                        }: Pick<RendererProps, 
                              'data' 
                              | 'options' 
                              | 'visualizationName'>);
                  \end{minted}
                  \caption{Definizione delle \textit{props} della componente \texttt{Counter}}
                  \label{listing:counterProps}
            \end{listing}
      \item \textbf{Ottimizzazioni}:
            \begin{itemize}
                  \item Utilizzo della libreria \textit{numbro.js} a favore della libreria \textit{numeral.js} precedentemente utilizzata, più pesante e non più mantenuta,
                        per la formattazione delle label numeriche all'interno del \texttt{Counter};
                  \item La progettazione della presente componente prevede l'utilizzo di \textit{hooks} per la gestione del suo stato interno e per il calcolo delle dimensioni
                        del contatore, necessarie per permettere la visualizzazione responsiva della componente: l'utilizzo di tali \textit{hooks} permette di garantire
                        un'implementazione efficiente e performante, riducendo il numero di renderizzazioni e computazioni non necessarie.
            \end{itemize}
\end{itemize}

\subsection{Chart}
\begin{itemize}
      \item \textbf{Nome della componente}: \texttt{Chart};
      \item \textbf{Descrizione}: La componente \texttt{Chart} prevede la visualizzazione di un grafico, il cui tipo e i dati da visualizzare vengono definiti a partire dai dati passati
            come \textit{props} tra le varie componenti e forniti dalla risposta in formato \textit{JSON} ricevuta dai \textit{server Datasoil}. \newline
            I tipi di grafici che tale componente permette di renderizzare sono i seguenti:
            \begin{itemize}
                  \item \textit{Area Chart}: grafico ad area;
                  \item \textit{Bar Chart}: grafico a barre;
                  \item \textit{Box Plot Chart}: grafico a scatola;
                  \item \textit{Bubble Chart}: grafico a bolle;
                  \item \textit{HeatMap Chart}: grafico a matrice;
                  \item \textit{Line Chart}: grafico a linee;
                  \item \textit{Pie Chart}: grafico a torta;
                  \item \textit{Scatter Chart}: grafico a dispersione.
            \end{itemize}
            La componente permette inoltre, a seconda del tipo di grafico renderizzato, di:
            \begin{itemize}
                  \item Effettuare il download dell'immagine del grafico in formato \textit{.png};
                  \item Effettuare lo zoom sul grafico;
                  \item Effettuare lo zoom-in sul grafico;
                  \item Effettuare lo zoom-out sul grafico;
                  \item Effettuare il pan sul grafico;
                  \item Effettuare l'autoscale sul grafico;
                  \item Resettare la dimensione degli assi del grafico.
            \end{itemize}
            A fronte di tali funzionalità, l'implementazione della componente prevede l'utilizzo della libreria \textit{Plotly.js}, la quale permette
            di renderizzare grafici in modo reattivo. \newline
            Le evenutali label applicate ai grafici possono essere formattate in base alle preferenze definite lato backend, presentando le seguenti opzioni. \newline
            \begin{itemize}
                  \item Label numeriche:
                        \begin{itemize}
                              \item Numero di cifre decimali: numero di cifre decimali da visualizzare nel valore della label;
                              \item Separatore delle migliaia: carattere da utilizzare come separatore delle migliaia;
                              \item Separatore decimale: carattere da utilizzare come separatore decimale;
                              \item suffissi: stringa da aggiungere dopo il valore della label, ad esempio '\textit{\%}'.
                        \end{itemize}
                        L'implementazione delle formattazioni numeriche prevede l'utilizzo della libreria \textit{numbro.js}. \newline
                  \item Label temporali:
                        \begin{itemize}
                              \item Formato della data: formato da utilizzare per la visualizzazione della data;
                              \item Formato dell'ora: formato da utilizzare per la visualizzazione dell'ora.
                        \end{itemize}
                        L'implementazione delle formattazioni temporali prevede l'utilizzo della liberia \textit{day.js}.
            \end{itemize}
      \item \textbf{Props}: I \textit{props} della componente \texttt{Chart} sono definiti dall'interfaccia \\
            \texttt{RendererProps}, su cui il componente ne effettua il \textit{picking}
            delle informazioni \textit{data} e \textit{options}.
            \begin{listing}[H]
                  \begin{minted}{typescript}
                  export function Chart({
                        data,
                        options,
                        }: Pick<RendererProps, 
                              'data' 
                              | 'options'>);
                  \end{minted}
                  \caption{Definizione delle \textit{props} della componente \texttt{Chart}}
                  \label{listing:chartProps}
            \end{listing}
      \item \textbf{Ottimizzazioni}:
            \begin{itemize}
                  \item Utilizzo di un \textit{custom bundle} di \textit{Plotly.js} in modo da ridurre le dimensioni del \textit{bundle} finale, evitando la registrazione
                        di grafici non utilizzati all'interno delle dashboard \textit{Datasoil S.r.l.} Tale ottimizzazione permette inoltre di evitare la registrazione
                        manuale delle \textit{traces}, necessaria nelle versioni rese disponibili da \textit{Plotly.js} a causa di un bug noto non risolto della libreria.
                  \item Utilizzo della libreria \textit{numbro.js} a favore della libreria \textit{numeral.js} precedentemente utilizzata, più pesante e non più mantenuta,
                        per la formattazione delle label numeriche all'interno dei grafici;
                  \item Utilizzo della libreria \textit{day.js} a favore della libreria \textit{moment.js} precedentemente utilizzata, in quanto costituisce una
                        \textit{peerdependency} all'interno dei prodotti \textit{Datasoil S.r.l.}, riducendo così le dimensioni del \textit{bundle} finale;
                  \item Utilizzo di un \textit{custom hooks} per la gestione dello stato interno della componente e per il calcolo delle dimensioni del grafico, necessarie per permettere
                        la visualizzazione responsiva della componente. L'utilizzo dell'\textit{hooks} è volto a garantire un'implementazione efficiente e performante della componente.
            \end{itemize}
\end{itemize}

\subsection{Table}
\begin{itemize}
      \item \textbf{Nome della componente}: \texttt{Table};
      \item \textbf{Descrizione}: La componente \texttt{Table} prevede la visualizzazione di una tabella, i cui dati e la cui struttura (intesa come colonne e tipo di dato visualizzato
            all'interno di esse) vengono definiti a partire dai dati passati come \textit{props} tra le varie componenti e forniti dalla risposta in formato \textit{JSON} ricevuta dai \textit{server Datasoil}. \newline
            La componente permette inoltre di:
            \begin{itemize}
                  \item Ordinare le colonne della tabella;
                  \item Filtrare le \textit{entry} della tabella grazie ad un filtro globale;
                  \item Effettuare il download della tabella in formato \textit{.csv};
            \end{itemize}
            A fronte di tali funzionalità, l'implementazione della componente prevede l'utilizzo della componente fornita dalla libreria \textit{PrimeReact}.
      \item \textbf{Props}: I \textit{props} della componente \texttt{Table} sono definiti dall'interfaccia \\
            \texttt{RendererProps}, su cui il componente ne effettua il \textit{picking}
            delle informazioni \textit{data, options} e \textit{visualizationName}.
            \begin{listing}[H]
                  \begin{minted}{typescript}
                  export function Table({
                        data,
                        options,
                        visualizationName,
                        }: Pick<RendererProps, 
                              'data' 
                              | 'options' 
                              | 'visualizationName'>);
                  \end{minted}
                  \caption{Definizione delle \textit{props} della componente \texttt{Table}}
                  \label{listing:tableProps}
            \end{listing}
      \item \textbf{Ottimizzazioni}:
            \begin{itemize}
                  \item Utilizzo della componente \texttt{DataTable} fornita dalla libreria \textit{PrimeReact} a favore della componente \texttt{Table} di \textit{AntD} precedentemente utilizzata,
                        in quanto la libreria \textit{PrimeReact} offre una maggiore flessibilità e personalizzazione delle tabelle, oltre che a costituire una \textit{peerdependency} all'interno dei
                        prodotti \textit{Datasoil S.r.l.}, riducendo così le dimensioni del \textit{bundle} finale;
                  \item Utilizzo di \textit{hook} per la gestione del filtro globale, in modo da garantire un'implementazione efficiente e performante della componente, a differenza dell'implementazione
                        fornita da \textit{PrimeReact} che comporta un bug noto non risolto. \label{item:hookTable}
            \end{itemize}
\end{itemize}

\subsection{Renderer}
\begin{itemize}
      \item \textbf{Nome della componente}: \texttt{Renderer};
      \item \textbf{Descrizione}: La componente \texttt{Renderer} seleziona dinamicamente un sottocomponente tra quelli resi disponibili dalla libreria \textit{viz-lib}
            in base al tipo di visualizzazione definito nei dati passati come \textit{props}, costituendo l'\textit{entry point} della libreria. \newline
      \item \textbf{Props}: I \textit{props} della componente \texttt{Renderer} sono definiti dall'interfaccia \\
            \texttt{RendererProps}, su cui il componente ne effettua il \textit{picking}
            delle informazioni \textit{data, options} e \textit{visualizationName}.
            \begin{listing}[H]
                  \begin{minted}{typescript}
                  export function Renderer({
                        data,
                        options,
                        visualizationName,
                        }: Pick<RendererProps, 
                              'data' 
                              | 'options' 
                              | 'visualizationName'>);
                  \end{minted}
                  \caption{Definizione delle \textit{props} della componente \texttt{Renderer}}
                  \label{listing:rendererPropsInComponent}
            \end{listing}
      \item \textbf{Ottimizzazioni}:
            \begin{itemize}
                  \item La componente prevede l'utilizzo di \textit{hooks} per ottimizzare le prestazioni, rendendo il componente \gls{memoizzato}\glox, prevenendo i rendering
                        non necessari, garantendo così un'implementazione efficiente e performante.
            \end{itemize}
\end{itemize}

\subsection{VisualizationWidgetHeader}
\begin{itemize}
      \item \textbf{Nome della componente}: \texttt{VisualizationWidgetHeader};
      \item \textbf{Descrizione}: La componente \texttt{VisualizationWidgetHeader} appartiene alla libreria \textit{dashboard}, appartenente sempre al SDK prodotto. Questa componente costituisce l'\textit{header}
            del \textit{container} per la visualizzazione delle componenti da renderizzare all'interno della dashboard. \newline
      \item \textbf{Props}: I \textit{props} della componente \texttt{VisualizationWidgetHeader} sono definiti delle informazioni \textit{name, visualization} e \textit{dataDialog}.
            \begin{listing}[H]
                  \begin{minted}{typescript}
                  function VisualizationWidgetHeader({
                        name,
                        visualization,
                        dataDialog,
                        }: {
                        name: string;
                        visualization: Visualization;
                        dataDialog: WidgetData | undefined;
                        });
                  \end{minted}
                  \caption{Definizione delle \textit{props} della componente \texttt{VisualizationWidgetHeader}}
                  \label{listing:visualizationWidgetHeaderProps}
            \end{listing}
      \item \textbf{Ottimizzazioni}:
            \begin{itemize}
                  \item Ridefinizione dei \textit{props} della componente in modo da permettere la visualizzazione di un \textit{dialog} contenente la componente visualizzata all'interno del \textit{container};
                  \item Ridefinizione della componente in modo da permettere la visualizzazione di un \textit{dialog} contenente la componente visualizzata all'interno del \textit{container}, attraverso
                        l'utilizzo di componenti fornite dalla libreria \textit{PrimeReact}.
            \end{itemize}
\end{itemize}
\chapter{Realizzazione e testing}
\label{chap:realizzazione-testing}

\section{Realizzazione delle componenti}

\subsection{Ambiente di sviluppo}
Durante lo sviluppo della libreria, è stata utilizzata la versione 18.12.1 di \textit{Node.js}, configurata tramite l'utilizzo di \textit{NVM}
(Node Version Manager), un gestore di versioni di \textit{Node.js} che permette di installare e gestire più versioni in modo semplice
e veloce. \newline
Per quanto riguarda la gestione delle dipendenze, è stato utilizzato \textit{Yarn}, un package manager per JavaScript che permette di gestire
le dipendenze del progetto in modo efficiente e veloce. \newline
Il versionamento del codice è stato gestito tramite \textit{Git}, un sistema di controllo di versione distribuito, configurando e
mantenendo un repository remoto su \textit{Github} interno all'organizzazione dell'azienda. \newline
Per il processo di building del codice è stato utilizzato \textit{Rollup}, un bundler di moduli JavaScript che permette:
\begin{itemize}
    \item di creare bundle di moduli in formato \textit{ESM} (ECMAScript Module) e \textit{CJS} (CommonJS);
    \item di utilizzare TypeScript all'interno del progetto, integrando la configurazione dichiarata nel file \textit{tsconfig.json};
    \item la risoluzione delle dipendenze tra i moduli, escludendo le \textit{peerdependency} dal bundle;
    \item la produzione di bundle di dimensioni ridotte grazie alla sua capacità di effettuare il tree-shaking;
    \item il supporto al plugin \textit{terser}, utilizzato per la minimizzazione del codice prodotto attraverso la rimozione dei commenti e degli spazi vuoti,
          effettuando il \textit{munging} dei nomi delle variabili e introducendo ottimizzazioni per ridurre la dimensione finale;
    \item il supportare a \textit{sourcemaps} per facilitare il debug del codice, permettendo di mappare il codice minificato con il codice sorgente originale;
    \item la generazione di file di dichiarazione TypeScript;
    \item il supporto a plugin per il calcolo della dimensione del bundle, specificando la dimensione di ogni singola dipendenze
          all'interno del progetto, generando un file di report \textit{html}.
\end{itemize}
All'interno del file \textit{package.json} è stata configurata la sezione \textit{scripts} per definire i comandi necessari per l'esecuzione:
\begin{itemize}
    \item \textit{start-watcher}: avvia il processo di building del codice di \textit{rollup} in modalità \textit{watch}, in modo da monitorare le modifiche effettuate
          ai file sorgente e aggiornare automaticamente il bundle prodotto e utilizzato dall'example all'interno del progetto;
    \item \textit{start-example}: avvia l'esecuzione dell'esempio all'interno del progetto nel server locale, permettendo di visualizzare il funzionamento della libreria
          all'interno di un'applicazione di test;
    \item \textit{build-package}: avvia il processo di building del codice di \textit{rollup}, generando i bundle finali configurati all'interno del file \textit{rollup.config.js};
    \item \textit{publish-package}: avvia il processo di pubblicazione del pacchetto all'interno del registry privato configurato nel \textit{package.json} dei packages, permettendo
          di distribuire la libreria all'interno dell'organizzazione;
    \item \textit{test}: avvia il processo di testing del codice tramite \textit{Jest}, effettuando i test definiti all'interno della repository.
\end{itemize}
\begin{listing}[H]
    \begin{minted}{json}
    "scripts": {
        "start-example": "cd packages/example && yarn start",
        "start-watcher": "cd packages/dsdashboard2 && yarn rollup-watch",
        "build-package": "cd packages/dsdashboard2 && yarn build",
        "publish-package": "cd packages/dsdashboard2 && yarn publish"
    }
    \end{minted}
    \caption{Scripts del file package.json di dsdashboard2}
    \label{listing:scripts_package_json_dsdashboard2}
\end{listing}

\begin{listing}[H]
    \begin{minted}{json}
    "scripts": {
        "rollup": "rollup -c --bundleConfigAsCjs",
        "rollup-watch": "rollup -c --bundleConfigAsCjs --watch",
        "clean": "rimraf dist",
        "build": "yarn clean && yarn rollup",
        "test": "jest"
    }
    \end{minted}
    \caption{Scripts del file package.json dei packages}
    \label{listing:scripts_package_json_packages}
\end{listing}
Tutti i precedenti comandi vengono eseguiti tramite il comando \textit{yarn} seguito dal nome dello script definito all'interno del file \textit{package.json}.

\begin{figure}[H]
    \centering
    \includegraphics[alt={Esempio di bundle visualizer report}, width=1 \columnwidth]{img/bundle-visualizer.png}
    \caption{Esempio di bundle visualizer report}
    \label{fig:bundle-visualizer}
\end{figure}

\subsection{Componenti}
Nella presente sezione viene descritta l'implementazione delle componenti e delle loro ottimizzazioni all'interno della libreria,
con l'obiettivo di fornire una panoramica sulle funzionalità offerte, sulle tecnologie utilizzate e sul flusso di dati all'interno del sistema.

\subsubsection{Counter}

\subsubsection{Chart}

\subsubsection{Table}
La componente \textit{Table} costituisce una tabella che permette di visualizzare i dati in forma tabellare. \newline
Tale componente utilizza \textit{DataTable}, una componente resa disponibile dalla libreria \textit{PrimeReact} che offre, attraverso
la definizione di determinate props, di:
\begin{itemize}
    \item alternare i colori delle righe della tabella, al fine di aumentare la leggibilità dei dati visualizzati;
    \item selezionare più righe della tabella, permettendo di effettuare operazioni di \textit{multi-selection} (attraverso
          l'utilizzo di uno \textit{useState});
    \item impostare un filtro globale per la tabella, permettendo di definire su quali colonne effettuare la ricerca dei dati;
    \item impostare un \textit{empty message} personalizzato, visualizzato nel caso in cui la tabella non contenga dati;
    \item ordinare i dati all'interno delle singole colonne, permettendo di visualizzare i dati in ordine crescente o decrescente.
\end{itemize}
Come introdotto nella sezione \ref{item:hookTable}{ \textit{proprietà della componenti: Table}}, la gestione del filtro globale tramite
la funzione \textit{onChangeGlobalFilterValue} presenta l'utilizzo di due valori tramite \textit{useState}:
\begin{itemize}
    \item \textit{globalFilterValueTmp}: stringa utilizzata per memorizzare il valore del filtro impostato all'interno di una
          componente \textit{InputText} di \textit{PrimeReact}, aggiornata in tempo reale all'input fornito dell'utente;
    \item \textit{globalFilterValue}: stringa utilizzata per impostare il valore del filtro globale della tabella, aggiornata
          tramite una chiamata \textit{debounced} di 200 ms alla funzione per settate il valore sullo stato. \newline
          (La funzione \textit{debounced} fa uso dell'hook \textit{useDebounceCallback} fornito dalla libreria \textit{usehook-ts}).
\end{itemize}
Tramite questa implementazione, il recupero dei dati visualizzati all'interno della tabella viene eseguito correttamente, permettendo di risolvere
il bug presente all'interno della componente \textit{DataTable}, il quale comportava la perdita di entry nel caso in cui il filtro globale subisse
più modifiche in rapida successione. \newline
La tabella è resa \textit{scrollable}, impostando una altezza che viene calcolata in modo responsivo a seconda dello spazio disponibile
all'interno del widget in cui è contenuta. \newline
Attraverso la definizione di un \textit{ref} alla componente \textit{DataTable}, è possibile effettuare il download della tabella in formato
\textit{CSV}, tramite l'utilizzo della funzione \textit{exportCSV} resa disponibile dalla libreria \textit{PrimeReact}: tale funzione viene
invocata tramite un \textit{Button} posizionato a fianco della componente \textit{InputText} utilizzata per l'input del filtro globale. \newline
\textbf{Elaborazione dati}\newline
I dati renderizzati all'interno della tabella vengono estratti dal props \textit{data}, passando a sua volta alla componente \textit{DataTable}
tramite la prop \textit{value} l'array \textit{rows} contenente i dati da visualizzare. \newline
Le colonne della tabella sono definite a partire dalle \textit{options} e dai \textit{data} passati come props alla componente \textit{Table}.
Inizialmente sono ridefinite mediante la funzione \textit{getOptions}, la quale imposta delle opzioni di default, ordina la visualizzazione delle colonne
in base all'ordine definito nelle \textit{options}, impostando le proprietà di filtro delle colonne in base alle opzioni indicate e verificando la presenza
di dati all'interno dei \textit{data} (nel caso di assenza di riscontri vengono definite le colonne a partire dai \textit{data}).
Successivamente dalle \textit{options} elaborate vengono definite le vere e proprie colonne della tabella (\textit{tableColumns}), attraverso la funzione \textit{prepareColumns},
la quale attua il filtraggio delle colonne e costruisce l'header delle colonne in base alle opzioni elaborate. \newline
Le \textit{tableColumns} vengono in seguito mappate per creare le componenti \textit{Column} presentate all'interno della tabelle, impostando:
\begin{itemize}
    \item l'header a partire dalle opzioni elaborate;
    \item il \textit{field} a cui fanno riferimento a partire dalle opzioni elaborate;
    \item il props \textit{sortable} per permettere l'ordinamento sulla colonna;
    \item il props \textit{body}, utilizzato per permettere il corretto formattamento e rendering dei dati visualizzati all'interno della singola colonna,
          invocando la funzione \textit{formatRowValue}.
\end{itemize}
Le formattazioni dei dati avvengono mediante i formatter resi disponibili:
\begin{itemize}
    \item \textit{string}: formatta il testo in base indicazioni presenti nell'item, eventualmente restituendo del contenuto HTML;
    \item \textit{datetime}: formatta le date in base alle opzioni elaborate mediante l'uso di \textit{day.js};
    \item \textit{number}: formatta i numeri in base alle opzioni elaborate mediante l'uso di \textit{numbro};
    \item \textit{boolean}: formatta i booleani in base alle opzioni fornite, elaborando anche array;
    \item \textit{json}: formatta i dati JSON in base alle opzioni fornite, permettendo di visualizzare i dati in modo strutturato;
    \item \textit{image}: formatta le immagini in base alle opzioni fornite, permettendo di visualizzare le immagini all'interno della tabella.
\end{itemize}

\begin{figure}[H]
    \centering
    \includegraphics[alt={Esempio di Table viz-lib}, width=1 \columnwidth, height=\maxdimen, keepaspectratio]{img/ex_table.png}
    \caption{Esempio di Table \textit{viz-lib}}
    \label{fig:table-example}
\end{figure}

\subsubsection{Renderer}
La componente \textit{Renderer} costituisce l'entry point della libreria, permettendo di renderizzare tutte le componenti presenti all'interno di
\textit{viz-lib}: per questo motivo tale componente costituisce l'unico \textit{export} della libreria accessibile all'\textit{index.ts} del SDK. \newline
Tale componente costituisce il \textit{wrap} della componente \textit{RendererR}, ottenuto tramite l'utilizzo della funzione \textit{React.memo}.
\begin{adjustwidth}{4em}{0pt}
    \textit{React.memo} è una funzione offerta da \textit{React} per effettuare la \textit{memoization} di una componente funzionale, permettendo di evitare
    nuove renderizzazioni nel caso in cui il componente padre lo richiedesse, a meno che le sue props non abbiano subito modifiche. \newline
    Tale funzione accetta due parametri:
    \begin{itemize}
        \item la componente da memoizzare;
        \item una funzione che accetta due argomenti: le props correnti e le props precedenti della componente, restituendo un valore booleano che indica
              se la componente debba essere renderizzata o meno.
    \end{itemize}
    La presenza del secondo parametro permette di effettuare un controllo specifico, andando a definire i criteri su cui basare i cambiamenti delle props
    rilevanti ai fini della renderizzazione della componente. \newline
\end{adjustwidth}
Nel nostro caso, la funzione \textit{React.memo} utilizza come criterio di controllo tra le props la funzione \textit{isEqual} resa disponibile da \textit{lodash},
la quale permette di effettuare un confronto profondo tra due valori per determinare se sono equivalenti. \newline
\begin{listing}[H]
    \begin{minted}{typescript}
        const Renderer = React.memo(RendererR, (prev, next) =>
            isEqual(prev.data, next.data)
        );
  \end{minted}
    \caption{React.memo della componente Renderer}
    \label{listing:react_memo}
\end{listing}
Per quanto riguarda la componente \textit{RendereR} in questione, essa ricava dai suoi props, tramite l'utilizzo della funzione \textit{getVisualizationType}, il valore
dell'enumerazione \textit{VisualizationType}, assegnandolo alla costante \textit{visualizationType}, in modo da renderizzare la componente corretta in base al tipo di
visualizzazione passato come parametro.

\subsection{Documentazione}
L'implementazione della libreria prodotta è stata documentata tramite l'utilizzo di \textit{Confluence}, la piattaforma
di gestione della conoscenza e di collaborazione sviluppata da \textit{Atlassian}. \newline
La documentazione, su richiesta dell'azienda, è stata redatta in lingua inglese, attraverso una descrizione dettagliata
delle componenti e delle funzionalità offerte dalla libreria. \newline

\begin{figure}[H]
    \centering
    \includegraphics[alt={Esempio documentazione Confluence}, width=1 \textwidth]{img/ex_confluence.png}
    \caption{Esempio documentazione Confluence}
    \label{fig:ex_confluence}
\end{figure}

La documentazione è stata strutturata in modo da essere facilmente consultabile e comprensibile, con l'obiettivo di
fornire un supporto efficace agli sviluppatori futuri che potrebbero dover utilizzare la libreria. \newline
La produzione di una buona documentazione ricopre infatti un ruolo fondamentale al fine di garantire la manutenibilità
del codice e la facilità di comprensione delle funzionalità offerte dalla libreria, in modo da ridurre i tempi di
apprendimento e di sviluppo necessari per gli utilizzi futuri del prodotto implementato.

\section{Testing}
Nella presente sezione verranno descritte le attività di testing effettuate durante lo sviluppo della libreria, con l'obiettivo
di garantire la qualità del prodotto implementato e la corretta esecuzione delle funzionalità offerte.

\subsection{Jest}
Lo strumento utilizzato e configurato all'interno del progetto per l'esecuzione dei test è \textit{Jest}, un framework di testing
per JavaScript sviluppato da \textit{Facebook}. \newline
Jest permette di effettuare test su funzioni, classi e moduli, fornendo un'ampia gamma di funzionalità per la scrittura e l'esecuzione
dei test. \newline
In particolare, Jest offre le seguenti funzionalità:
\begin{itemize}
    \item \textbf{Mocking components}: permette di creare mock di componenti, in modo da simularne il suo comportamento:
          \begin{listing}[H]
              \begin{minted}{typescript}
            jest.mock('percorso.componente', () => ({
                Componente: () => mock_value,
            }));
            \end{minted}
              \caption{Esempio di mock di una componente}
              \label{listing:mock_component}
          \end{listing}
    \item \textbf{Mocking function}: permette di creare mock di funzioni, in modo da simularne il suo comportamento:
          \begin{listing}[H]
              \begin{minted}{typescript}
                jest.spyOn(file_funzione, 'nomeFunzione')
                    .mockReturnValue(mock_value);
            \end{minted}
              \caption{Esempio di mock di una funzione}
              \label{listing:mock_function}
          \end{listing}
    \item \textbf{Suite di test}: permette di creare suite di test, organizzando i test in modo gerarchico:
          \begin{listing}[H]
              \begin{minted}{typescript}
                describe('Nome suite di test', () => {
                    it('Nome test', () => {
                        // Codice del test
                    });
                });
            \end{minted}
              \caption{Esempio di suite di test}
              \label{listing:test_suite}
          \end{listing}
    \item \textbf{Expect}: permette di effettuare asserzioni sui valori restituiti dalle funzioni, verificando la correttezza
          del risultato ottenuto:
          \begin{listing}[H]
              \begin{minted}{typescript}
                    expect(valore).toBe(valore_aspettato);
                \end{minted}
              \caption{Esempio di expect su valore}
              \label{listing:expect}
          \end{listing}
          oppure di verificare la presenza di un elemento all'interno del DOM:
          \begin{listing}[H]
              \begin{minted}{typescript}
                    expect(screen.getByText('Testo')).toBeInTheDocument();
                \end{minted}
              \caption{Esempio di expect su elemento del DOM}
              \label{listing:expect_dom}
          \end{listing}
\end{itemize}
La sua configurazione è stata effettuata all'interno del file \textit{packages.json}, in cui sono state definite le impostazioni di
esecuzione dei test e le dipendenze necessarie per il loro corretto funzionamento, come il preset che consente di utilizzare \textit{Typescript},
l'ambiente di test \textit{jsdom} che simula un ambiente browser e i vari formati di file che deve considerare o ignorare. \newline
Di seguito viene riportata la configurazione utilizzata.
\begin{listing}[H]
    \begin{minted}{json}
    "jest": {
        "preset": "ts-jest",
        "testEnvironment": "jsdom",
        "moduleFileExtensions": [
            "ts",
            "tsx",
            "js",
            "jsx",
            "json",
            "node"
        ],
        "transform": {
            "^.+\\.(ts|tsx)$": "ts-jest",
            "^.+\\.(js|jsx)$": "babel-jest"
        },
        "transformIgnorePatterns": [
            "node_modules/(?!(d3-color)/)"
        ]
    }
    \end{minted}
    \caption{Configurazione Jest all'interno del file packages.json}
    \label{listing:jest_config}
\end{listing}

\subsection{Unit testing}
Per garantire la correttezza delle funzionalità offerte dalla libreria, è stato effettuato un processo di testing a livello di unità. \newline
Nel presente progetto sono stati implementati test per le singole componenti, mockando le dipendenze esterne ed eventuali altre componenti
della libreria utilizzate, verificando il corretto funzionamento all'interno della singola unità.

\begin{listing}[H]
    \begin{minted}[escapeinside=||]{typescript}
    jest.mock('./charts/Counter', () => ({
        Counter: () => <div>Counter Component|</|div>,
    }));

    jest.mock('./charts/Chart', () => ({
        Chart: () => <div>Chart Component|</|div>,
    }));

    jest.mock('./charts/Table', () => ({
        Table: () => <div>Table Component|</|div>,
    }));

    describe('Renderer test right components', () => {
        it('renders Counter component when visualizationType is Counter', 
            () => {
            jest.spyOn(helper, 'getVisualizationType')
                .mockReturnValue(VisualizationType.Counter);

            const rendererProps: RendererProps = {
                visualizationName: 'Counter',
                type: 'Counter',
                data: {},
                options: {},
            };

            render(<Renderer {...rendererProps} />);

            expect(screen.getByText('Counter Component')).toBeInTheDocument();
            });
    // ...
    });
    \end{minted}
    \caption{Esempio di unit test: Renderer component}
    \label{listing:test_Renderer}
\end{listing}

Come si può osservare nell'esempio di codice riportato in \ref{listing:test_Renderer}, è stato effettuato un test sulla componente \textit{Renderer},
mockando le componenti \textit{Counter}, \textit{Chart} e \textit{Table} utilizzate all'interno della componente stessa, verficando che venga renderizzata
la componente corretta in base al tipo di visualizzazione passato come parametro. \newline
Il tipo di visualizzazione a sua volta viene ottenuto dal mock della funzione \textit{getVisualizationType}, la quale restituisce il tipo di visualizzazione
corretto in base al parametro passato come props al Renderer: in questo modo viene garantita la corretta logica della singola componente, senza far affidamento
su funzioni o componenti esterne. \newline
La correttezza delle funzioni è stata verificata tramite appositi test di unità sulle singole funzioni, eventualmente mockando le dipendenze esterne utilizzate
all'interno della funzione stessa, in modo da garantire la correttezza logica implementata. \newline
Di seguito viene riportato un esempio di test di unità effettuato sulla funzione \textit{getVisualizationType}.

\begin{listing}[H]
    \begin{minted}{typescript}
    it('should return VisualizationType.Table when type is TABLE', () => {
        expect(
            getVisualizationType('TABLE')
        ).toBe(VisualizationType.Table);
    });
    \end{minted}
    \caption{Esempio di unit test: getVisualizationType}
    \label{listing:test_getVisualizationType}
\end{listing}

\subsection{Integration testing}
Per garantire la corretta integrazione delle componenti all'interno della libreria, è stato effettuato un processo di testing a livello di integrazione,
tramite l'utilizzo di \textit{Jest}. \newline
I test di integrazione permettono di verificare il corretto funzionamento delle componenti all'interno del sistema, testando il comportamento
dei singoli moduli all'interno del contesto in cui sono utilizzati. \newline
Nel presente progetto sono stati implementati test di integrazione per verificare il corretto funzionamento delle componenti in relazione con dipendenze
esterne.

\begin{listing}[H]
    \begin{minted}[escapeinside=||]{typescript}
    jest.mock('./charts/Counter', () => ({
        Counter: () => <div>Counter Component|</|div>,
    }));

    jest.mock('./charts/Chart', () => ({
        Chart: () => <div>Chart Component|</|div>,
    }));

    jest.mock('./charts/Table', () => ({
        Table: () => <div>Table Component|</|div>,
    }));
    
    describe('Renderer test right components', () => {
        it('renders Counter component when visualizationType is Counter', () => {

            const rendererProps: RendererProps = {
                visualizationName: 'Counter',
                type: 'Counter',
                data: {},
                options: {},
            };

            render(<Renderer {...rendererProps} />);

            expect(screen.getByText('Counter Component')).toBeInTheDocument();
        });
    // ...
    });
    \end{minted}
    \caption{Esempio di integration test: Renderer component}
    \label{listing:test_Renderer_integration}
\end{listing}

Come si può osservare nell'esempio di codice riportato in \ref{listing:test_Renderer_integration}, è stato effettuato un test di integrazione sulla componente \textit{Renderer},
mockando le componenti \textit{Counter}, \textit{Chart} e \textit{Table} utilizzate all'interno della componente stessa, verficando che venga renderizzata
la componente corretta in base al tipo di visualizzazione passato come parametro. \newline
Il tipo di visualizzazione a sua volta viene ottenuto internamente alla componente grazie alla funzione \textit{getVisualizationType}, la quale restituisce il tipo di visualizzazione
corretto in base al parametro passato come props al Renderer: in questo modo viene garantita la corretta integrazione delle componenti all'interno del sistema, testando il comportamento
delle unità in relazione tra loro all'interno del contesto in cui sono utilizzate.
\chapter{Rilascio}
\label{chap:rilascio}   
\chapter{Conclusioni}
\label{chap:conclusioni}
Nella seguente ed ultima sezione vengono presentate le conclusioni del progetto di stage, valutando i risultati conseguiti, esponendo
le riflessioni finali e proponendo possibili sviluppi futuri.

\section{Consuntivo finale}
Nella seguente tabella vengono riportate le ore effettive svolte durante il progetto di stage, suddivise per attività e per periodo di svolgimento.

\begin{center}
    \rowcolors{1}{white}{tableGray}
    \begin{longtable}{|p{2.5cm}|p{2.5cm}|p{7.5cm}|}
        \hline
        \rowcolor{gray!30}
        \textbf{Ore}                       & \multicolumn{1}{|c|}{\textbf{Settimane}}         & \multicolumn{1}{|c|}{\textbf{Descrizione}}                                      \\
        \hline
        \endfirsthead
        \hline
        \multicolumn{3}{|c|}{{\tablename\ \thetable{} -- continua dalla pagina precedente}}                                                                                     \\
        \hline
        \rowcolor{gray!30}
        \textbf{Ore}                       & \multicolumn{1}{|c|}{\textbf{Settimane}}         & \multicolumn{1}{|c|}{\textbf{Descrizione}}                                      \\
        \endhead
        \hline
        \multicolumn{3}{|r|}{{Continua nella prossima pagina...}}                                                                                                               \\
        \hline
        \endfoot
        \hline
        \multicolumn{1}{|c|}{\textbf{304}} & \multicolumn{2}{|c|}{\textbf{Totale ore svolte}}                                                                                   \\
        \hline
        \endlastfoot
        \hline
        54                                 & 1, 2                                             & Formazione su tecnologie utilizzate, studio del progetto e del SDK preesistente \\
        \hline
        68                                 & 2, 3, 4                                          & Progettazione SDK: layer di rappresentazione - grafici                          \\
        \hline
        90                                 & 4, 5, 6                                          & Sviluppo SDK: layer di  rappresentazione - grafici                              \\
        \hline
        30                                 & 6                                                & Progettazione SDK: layer user interaction                                       \\
        \hline
        52                                 & 7, 8                                             & Refactor SDK: layer user interaction                                            \\
        \hline
        10                                 & 8                                                & Deployment dell'SDK prodotto                                                    \\
    \end{longtable}
    \captionof{table}{Tabella consuntivo finale}
    \label{tab:consuntivo_finale}
\end{center}


\section{Valutazione del progetto}

\section{Possibili sviluppi futuri}


\section{Riflessioni finali}
Durante lo svolgimento dello stage ho avuto modo di apprendere nuove competenze e conoscenze, applicandomi per la prima volta in un contesto lavorativo reale. \\
Ho avuto modo di studiare e lavorare con tecnologie che non avevo mai avuto l'opportunità di approfondire, come ad esempio la libreria di componenti \textit{PrimeReact} e la
libreria grafica \textit{Plotly.js}, oltre che approfondire le mie conoscenze in ambito di sviluppo web con \textit{React} e \textit{TypeScript}, utilizzando la mia
creatività e le mie competenze per sviluppare un prodotto finale che rispondesse alle esigenze dell'azienda. \\
Ho avuto l'opportunità di lavorare all'interno di un team di sviluppo, confrontandomi con colleghi più esperti e apprendendo costantentemente da loro, oltre che collaborare
con il mio tutor aziendale, il quale mi ha guidato e supportato durante tutto il percorso di stage. \\
Durante questa esperienza, è stato quindi fondamentale il saper comunicare con i miei colleghi, affinando le mie capacità di comunicazione e collaborazione, prestando la massima
attenzione ai dettagli e consigli ricevuti, cercando di migliorare costantemente il mio \textit{way of working} e le mie competenze. \\
Il progetto mi ha permesso di acquisire una maggior consapevolezza delle mie capacità, mettendomi alla prova in un ambiente professionale e affrontando problematiche concrete
che richiedessero soluzioni immediate ma allo stesso tempo funzinali e ben strutturate: ciò che più emerge in rilievo da questa esperienza è infatti l'importanza delle soluzioni individuate,
valutando accuratamente i pro e i contro di ciascuna opzione possibile, in quanto ogni scelta effettuata avrà delle ripercussioni sul progetto e sul lavoro futuro di manutenzione da svolgere
sul prodotto sviluppato. \\
E' risultato dunque fondamentale il saper sviluppare una valutazione critica che permettesse di individuare le soluzioni migliori e non quelle più immediate, cercando di prevedere
possibili problematiche future. \\
In conclusione, valuto quindi positivamente l'esperienza di stage svoltasi presso l'azienda Datasoil S.r.l., in quanto mi ha permesso di crescere professionalmente e personalmente,
acquisendo e affinando competenze e conoscenze funzionali al mio percorso di studi e alla mia futura carriera lavorativa.

% Considerazioni Finali e Conclusioni
%     Riflessioni sull'esperienza di stage.
%     Importanza delle competenze acquisite durante lo stage.
%     Impatto delle ottimizzazioni sulla libreria e sull'azienda.
%     Suggerimenti per futuri sviluppi e ottimizzazioni.

% Bibliografia
%     Elenco delle fonti utilizzate per la stesura della tesi (articoli, libri, documentazione tecnica, ecc.).

% Appendici (se necessario)
%     Codice sorgente rilevante.
%     Risultati dei test di benchmarking.
%     Documentazione aggiuntiva.
\backmatter
\pagenumbering{roman}
\input{references/bibliography}
\input{references/webliography}
\end{document}