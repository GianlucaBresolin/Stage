\documentclass[
    table,
    12pt,
    oneside,
    a4paper,
    italian
]{book}

\input{config/packages}
\input{config/thesis_config}

\date{}

\hypersetup{pdfstartview=}
\begin{document}
    \frontmatter
    \input{preface/1_title_page}
    \increasepagenumbering % increase the page numbrering by 1, in order to cout the frontispiece
    \input{preface/2_copyright}
    \cleardoublepage
\phantomsection
\pdfbookmark{Ringraziamenti}{Ringraziamenti}

\begin{flushright}{
        \slshape
        'L'importante non è vincere, è pensare in modo vincente. La vita è fatta per il 10\% da quello che succede e per il 90\% da come lo affrontiamo.'} \\
    \medskip
    Gianluca Vialli.
\end{flushright}

\begingroup
\let\clearpage\relax
\let\cleardoublepage\relax
\let\cleardoublepage\relax

\chapter*{Ringraziamenti}

\noindent Desidero esprimere la mia gratitudine al professor \myProf, mio relatore, per l'aiuto e il sostegno che mi ha dato durante la stesura dell'elaborato.

\vspace{0.35cm}

\noindent Vorrei esprimere il mio più affettuoso ringraziamento ai miei genitori, Marta e Riccardo, e a mia sorella Emma, per il costante sostegno e
per i valori che mi hanno trasmesso, i quali mi hanno permesso di diventare la persona che sono oggi e di raggiungere i miei traguardi.

\vspace{0.35cm}

\noindent Desidero esprimere un ringraziamento speciale alla mia amata, Rebecca, per il suo supporto, per il suo amore e per la sua costante presenza che mi ha permesso di
affrontare con serenità e tranquillità questo percorso di studi, non facendomi mai sentire solo.

\vspace{0.35cm}

\noindent Ringrazio infine i miei amici Nicola, Manuel, Leonardo, Matteo e Andrea per la loro compagnia, per i momenti di divertimento e per avermi sempre mantenuto in
contatto con la vita e il mondo esterno.

\vspace{0.75cm}

\noindent{\myLocation, \myTime}
\hfill \textit{\myName}

\endgroup

    \cleardoublepage
\phantomsection
\blankpage
\pdfbookmark{Compendio}{Compendio}
\begingroup
\let\clearpage\relax
\let\cleardoublepage\relax
\chapter*{Sommario}

Il presente documento descrive il lavoro svolto durante il periodo di stage, della durata di 304 ore, dal laureando Gianluca Bresolin
presso l'azienda \textit{Datasoil S.r.l.} Lo stage ha previsto la realizzazione di una libreria ReactJS attraverso il refactor
e l'ottimizzazione di una libreria utilizzata per la visualizzazione di dashboard parametriche, volta ad ampliare soluzioni e servizi
nel contesto di applicazioni SPA e nella realizzazione di interfacce utente, ponendo attenzione ai requisiti di scalabilità,
usabilità e performance. \newline
La prima attività condotta è stata l'analisi del codice sorgente della libreria esistente, in modo da poter comprendere il funzionamento e le opportunità
di ottimizzazione. Successivamente, sono state individuate le funzionalità da mantenere, quelle da migliorare e quelle da introdurre,
per poi procedere con la progettazione e l'implementazione della nuova libreria. \newline
Infine, la libreria realizzata è stata integrata nei prodotti \textit{Datasoil} attraverso un micro servizio dedicato, rendendo disponibile un SDK
compatibile con le risposte delle API Datasoil, senza richiedere modifiche lato server ai prodotti esistenti. \newline
Il risultato ottenuto al termine del periodo di stage consiste un SDK per la composizione dinamica di dashboard, dalle dimensioni ridotte rispetto
al boundle precedentemente utilizzato dall'azienda e dotato di una maggior robustezza e consistenza grazie ad un uso più opportuno di \textit{TypeScript},
permettendo inoltre di raggiungere una migliore aderenza a quelle che sono le esigenze dei clienti, grazie alla rimozione di grafici inutilizzati e
l'introduzione di nuove funzionalità.


\endgroup
\vfill

    \input{preface/5_table_of_contents}
    \printglossary[type=\acronymtype, title=Acronimi e abbreviazioni, toctitle=Acronimi e abbreviazioni]
    \printglossary[type=main, title=Glossario, toctitle=Glossario]
    \blankpage % example of a blank page that also increases the page couter by 1

    \mainmatter
    \chapter{Introduzione}
\label{chap:introduzione}

\section{L'azienda}
Lo stage è stato svolto presso l'azienda Datasoil S.r.l. situata nei dintorni della stazione ferroviaria di Padova.
Fondata nel 2016, Datasoil S.r.l. è un'azienda di prodotto che si dedica allo sviluppo di piattaforme per l'industria 4.0,
smart building e smart city.
L'obiettivo dell'azienda è integrare informazioni ed eventi provenienti dai vari livelli aziendali per creare insight proattivi
in tempo reale, garantendo che le informazioni corrette raggiungano le persone giuste al momento più opportuno.
Il punto di partenza è la spazialità aziendale e tutti gli asset che risiedono all'interno di essa, da cui
provengono tutti i dati che vengono analizzati trasversalmente grazie a sempre più persone e dispositivi connessi.
Da qui il nome dell'azienda, Datasoil: fare le informazioni da questo suolo fertile di dati da cui siamo circondati.

\begin{figure}[H]
      \centering
      \includegraphics[alt={Logo Datasoil S.r.l.}, width=0.25\columnwidth]{img/datasoil_logo.jpg}
      \caption{Logo Datasoil S.r.l.}
      \label{fig:datasoil}
\end{figure}

\section{L'idea}
\subsection{Il contesto applicativo}
L'azienda Datasoil S.r.l., essendo un'azienda di prodotto, nasce con l'idea di proporre servizi di monitoraggio e controllo
di impianti industriali, smart building e smart city, offrendo come principale prodotto la piattaforma SYN.

\begin{figure}[H]
      \centering
      \includegraphics[alt={Logo SYN}, width=0.25\columnwidth]{img/syn_logo.jpg}
      \caption{Logo SYN}
      \label{fig:syn}
\end{figure}

SYN è una piattaforma di monitoraggio e controllo di impianti industriali che raccoglie dati da sensori e dispositivi
distribuiti all'interno di un impianto o su più stabilimenti, permettendo di visualizzare in tempo reale lo stato di
funzionamento dei vari asset, di analizzare i dati raccolti e di attuare azioni di controllo o segnalazione tramite
ticketing. All'interno della piattaforma, a seconda dei piani attivi del cliente, è possibile visualizzare dashboard dinamiche
in merito ad informazioni filtrate e aggiornate live, permettendo di raggiungere una panoramica completa dello stato dei vari
asset monitorati in modo rapido, intuitivo ed efficace, grazie all'utilizzo di molteplici grafici e widget.

\subsection{Il progetto}
Il progetto svolto durante lo stage consiste nello sviluppo di una libreria TypeScript di componenti per la creazione di dashboard dinamiche.
Questo è stato realizzato tramite il refactoring e l'ottimizzazione di un tool grafico preesistente, integrato nei vari prodotti di Datasoil S.r.l.,
tra cui \textit{SYN}. La libreria è stata implementata a partire da un modulo open-source utilizzato in \textit{Redash},
una piattaforma per la creazione di dashboard dinamiche tramite interrogazioni sulle fonti di dati configurate all'interno del servizio. \newline
L'esigenza di tale progetto nasce dalla necessità da parte di Datasoil S.r.l. di avere una libreria grafica aggiornata alle versioni
correnti delle dipendenze utilizzate, raggiungendo una maggior performance e una maggiore manutenibilità del codice rispetto alla versione
preesistente, riducendo dove possibile le dipendenze esterne e introducendo migliorie grafiche e funzionali. \newline
La libreria sviluppata costituisce, insieme alla già esistente libreria \textit{dashboard} (la quale ha subito anch'essa in parte un processo di refactoring
per permettere l'integrazione con la nuova libreria e l'introduzione di nuove funzionalità), il nuovo SDK utilizzato all'interno dei prodotti Datasoil S.r.l.
per la realizzazione di dashboard dinamiche.

\section{Principali problematiche}
Durante l'analisi iniziale della libreria preesistente in uso all'interno dei prodotti di Datasoil S.r.l., sono emerse alcune problematiche
legate alla manutenibilità e alla performance del tool grafico a seguito di una attenta revisione del codice sorgente. \newline
Di seguito vengono presentate le principali problematiche riscontrate.

\subsection{Dipendenze obsolete}
La libreria preesistente utilizzava versioni obsolete delle dipendenze esterne, con conseguente degradazione delle prestazioni in termini
di utilizzo di spazio e di tempo di caricamente delle risorse, data l'assenza di ottimizzazioni e di aggiornamenti del codice sorgente. \newline
Essendo inoltre la libreria preesistente basata su una versione del modulo di visualizzazioni utilizzato nella piattaforma \textit{Redash},
le tecnologie utilizzate il più delle volte rappresentavano alternative a quelle già utilizzate nei prodotti Datasoil S.r.l., creando dipendenze
non necessarie, con conseguente aumento delle dimensioni del SDK utilizzato per la generazione delle dashboard.

\subsection{Manutenibilità del codice}
Il codice sorgente della libreria preesistente, sviluppato in tempistiche rapide a fronte di una esigenza specifica dell'azienda Datasoil S.r.l.,
risultava essere poco manutenibile e assente di documentazione. \newline
Inoltre, la libreria di visualizzazioni utilizzata da \textit{Redash} su cui si basa l'SDK utilizzato dall'azienda, inizialmente implementata in JavaScript,
ha subito solo successivamente un refactoring in TypeScript: questo refactoring non è però avvenuto con l'introduzione di interfacce e tipi, bensì con la semplice aggiunta
di tipi \textit{any} per le variabili e i parametri delle funzioni, introducendo numerosi \textit{@ts-ignore} per ignorare gli errori di compilazione, rendendo il codice
poco leggibile e difficile da mantenere.

\subsection{Componenti inutilizzate}
In quanto la libreria preesistente fosse realizzata a partire dal modulo utilizzato dalla piattaforma \textit{Redash}, non tutti i componenti presenti
in essa trovavano utilizzo all'interno dei prodotti realizzati da Datasoil S.r.l.: questo in gran parte era dovuto da un'offerta di grafici inopportuni
per quello che è il contesto applicativo dell'azienda.

\subsection{Assenza di funzionalità}
La libreria preesistente non soddisfaceva pienamente le esigenze dei clienti di Datasoil S.r.l., mancando di alcune funzionalità fondamentali.
Questa limitazione ha reso necessario un intervento di refactoring per colmare le lacune e migliorare le prestazioni complessive.

\section{Soluzione scelte}
Per risolvere le problematiche emerse durante l'analisi iniziale della libreria preesistente, si è optato per il refactoring e l'ottimizzazione
del codice sorgente, individuando le soluzioni presentate di seguito.

\subsection{Aggiornamento delle dipendenze}
Per risolvere il problema delle dipendenze obsolete, è stata effettuata un'analisi delle versioni correnti delle dipendenze utilizzate all'interno della libreria,
verificando che le funzionalità utilizzate non fossero deprecate o rimosse, procedendo con l'eventuale refactoring ad un codice compatibile che tenesse inoltre conto
delle nuove funzionalità introdotte nelle versioni più recenti. \newline
Grazie ad uno studio accurato delle librerie utilizzate nei prodotti Datasoil S.r.l., è stato possibile sostituire con librerie equivalenti dipendenze preesistenti,
riducendo così l'onere di utilizzo dell'SDK all'interno dei prodotti aziendali.

\subsection{Introduzione di interfacce e tipi}
Al fine di migliorare la manutenibilità e la comprensione del codice sorgente, è stato svolto un lavoro di introduzione di interfacce e tipi per le variabili e i parametri
delle funzioni, in modo da rendere il codice più leggibile e permettere controlli statici sul codice sorgente. \newline
Questo lavoro ha permesso di ridurre la presenza di tipi \textit{any} all'interno del codice sorgente, garantendo una maggiore sicurezza e affidabilità del prodotto finale.

\subsection{Refactoring del codice}
Per migliorare la comprensione e la manutenibilità del codice sorgente, è stato intrapreso un lavoro di refactoring su alcune porzioni della libreria preesistente.
Questo intervento ha comportato la riscrittura di diverse funzioni e la rimozione di componenti minori utilizzate per eseguire funzionalità semplici, attraverso un codice
più leggibile e immediato, eliminando complessità superflue e mantenendo l'efficacia nell'implementazione di operazioni elementari.

\subsection{Rimozione di componenti inutilizzate}
In merito alla presenza di componenti inutilizzate all'interno della libreria preesistente, è stata effettuata un'accurata analisi delle componenti presenti, selezionando
quelle che non trovavano un diretto utilizzo all'interno dei prodotti Datasoil S.r.l. e procedendo con la loro rimozione. \newline
Questo lavoro ha permesso di ridurre le dimensioni del codice sorgente e di migliorare le prestazioni complessive della libreria, diminuendo inoltre il numero di dipendenze
esterne richiesto per il corretto funzionamento dell'SDK prodotto.

\subsection{Implementazione di nuove funzionalità}
Per colmare le lacune riscontrate nella libreria preesistente in merito all'assenza di alcune funzionalità richieste dai clienti di Datasoil S.r.l. all'interno dei prodotti forniti,
è stato condotto un lavoro di implementazione di nuove funzionalità. \newline
In questo processo, si è cercato di utilizzare dipendenze esterne preferibilmente già esistenti o che fossero compatibili con il contesto applicativo dell'azienda. Qualora ciò non
fosse stato possibile, sono state introdotte nuove dipendenze esterne selezionate a seguito di un'analisi dettagliata che tenesse conto delle implicazioni in termini di dimensioni
finali dell'SDK prodotto.

\section{Strumenti Utilizzati}
La seguente sezione fornisce gli strumenti utilizzati durante lo svolgimento dello stage per la realizzazione della libreria grafica.
Gli strumenti verranno presentati in ordine alfabetico secondo la seguente struttura:
\begin{itemize}
      \item \textbf{Nome strumento}: nome dello strumento utilizzato;
      \item \textbf{Versione}: versione utilizzata nel progetto durante lo stage;
      \item \textbf{Link}: link di riferimento per ulteriori informazioni sullo strumento.
      \item \textbf{Descrizione}: breve descrizione dello strumento e delle sue funzionalità.
      \item \textbf{Vantaggi}: principali vantaggi derivanti dall'utilizzo dello strumento. \textit{(Opzionale)}
      \item \textbf{Svantaggi}: principali svantaggi derivanti dall'utilizzo dello strumento. \textit{(Opzionale)}
      \item \textbf{Alternative Esaminate}: alternative studiate e prese in considerazione durante la scelta dello strumento, con una breve considerazione
            su di esse e la motivazione per la quale non sono state selezionate. \textit{(Opzionale)}
\end{itemize}

\subsection{Tecnologie Frontend}

\subsubsection{Day.js}
\begin{itemize}
      \item \textbf{Nome strumento}: Dayjs
      \item \textbf{Versione}: 1.11.7
      \item \textbf{Link}: \href{https://day.js.org/}{day.js.org}
      \item \textbf{Descrizione}: Day.js è una libreria JavaScript open-source utilizzata per la manipolazione delle date e degli orari.
      \item \textbf{Vantaggi}:
            \begin{itemize}
                  \item Day.js è una libreria molto leggera, con un'ampia serie di funzionalità per la manipolazione delle date e degli orari;
                  \item Day.js viene fornito con dichiarazioni ufficiali di tipo per TypeScript
                  \item Day.js è tutt'ora supportato da una vasta e attiva community.
            \end{itemize}
      \item \textbf{Alternative Esaminate}:
            \begin{itemize}
                  \item \textit{Moment.js}: Moment.js è una libreria JavaScript open-source utilizzata per la manipolazione delle date e degli orari, ma risulta essere più pesante
                        di Day.js. Moment.js è inoltre considerato \textit{deprecated} a favore di Day.js, che offre una maggiore leggerezza e una maggiore efficienza.
            \end{itemize}
\end{itemize}

\subsubsection{Lodash}
\begin{itemize}
      \item \textbf{Nome strumento}: Lodash
      \item \textbf{Versione}: 4.14.0
      \item \textbf{Link}: \href{https://lodash.com/}{lodash.com}
      \item \textbf{Descrizione}: Lodash è una libreria JavaScript open-source utilizzata per la manipolazione di oggetti e array.
      \item \textbf{Vantaggi}:
            \begin{itemize}
                  \item Lodash offre un'ampia serie di funzionalità implementate con efficienza per la manipolazione di oggetti e array,
                        quali la ricerca, la modifica e la rimozione di elementi;
                  \item Lodash è supportato da una vasta e attiva community.
            \end{itemize}
      \item \textbf{Svantaggi}
            \begin{itemize}
                  \item Lodash è una libreria molto pesante, con un'ampia serie di funzionalità che possono non essere utilizzate all'interno del progetto:
                        per questo motivo in questo progetto sono eseguiti degli import sui moduli specifici, riducendo così le dimensioni del package finale.
            \end{itemize}
\end{itemize}

\subsubsection{React}
\begin{itemize}
      \item \textbf{Nome strumento}: React
      \item \textbf{Versione}: 18.0.0
      \item \textbf{Link}: \href{https://reactjs.org/}{reactjs.org}
      \item \textbf{Descrizione}: React è una libreria JavaScript open-source per la creazione di interfacce utente, sviluppata da Facebook.
      \item \textbf{Vantaggi}:
            \begin{itemize}
                  \item React offre un'ampia serie di funzionalità per la creazione di interfacce utente, quali il \textit{Virtual DOM}, il \textit{JSX} e \textit{Hooks};
                  \item React è supportato da una vasta e attiva community;
                  \item React è una libreria molto leggera, permettendo di creare interfacce utente performanti e veloci.
            \end{itemize}
\end{itemize}

\subsubsection{Plotly.js}
\begin{itemize}
      \item \textbf{Nome strumento}: Plotly.js
      \item \textbf{Versione}: custom-bundle: 2.33.0
      \item \textbf{Link}: \href{https://plotly.com/javascript/}{plotly.com/javascript}
      \item \textbf{Descrizione}: Plotly.js è una libreria JavaScript open-source, con supporto per TypeScript, utilizzata per la visualizazione di dati mediante grafici interattivi.
            Costruito sopra \textit{D3.js}, Plotly.js offre un vasto panorama di grafici dinamici in formato SVG, altamente personalizzabili.
      \item \textbf{Vantaggi}:
            \begin{itemize}
                  \item Plotly.js offre funzionalità di interattività avanzate, quali zoom, pan, selezione e salvataggio dei grafici;
                  \item Le disponibilità di grafici offerte da Plotly.js sono molto ampie, permettendo di soddisfare la maggior parte delle esigenze
                        all'interno della libreria grafica;
                  \item Plotly.js è supportato da una vasta e attiva community, con ampie serie di esempi disponibili su \textit{CodePen} (una piattaforma di condivisione di codice
                        online che permette di visualizzare e modificare codice HTML, CSS e JavaScript);
                  \item Plotly.js offre la possibilità di generare custom-bundle personalizzati, permettendo di registrare le sole \textit{traces} che si vogliono utilizzare, riducendo
                        così le dimensioni del pacchetto finale;
                  \item Plotly.js era già precedentemente utilizzato all'interno della libreria preesistente, non necessitando così di modifiche lato backend negli editor
                        utilizzati per la generazione delle risposte JSON da parte dei server Datasoil S.r.l. per la generazione delle dashboard;
                  \item Plotly.js offre una funzionalità, \textit{Plotly.react}, che permette di aggiornare i grafici in modo efficiente, riducendo il tempo di rendering
                        e migliorando le prestazioni complessive della libreria.
            \end{itemize}
      \item \textbf{Svantaggi}:
            \begin{itemize}
                  \item Plotly.js è una libreria molto pesante, dovuto anche dal fatto che è implementata sopra un wrapper proprietario di \textit{D3.js},
                        impedendo l'ottimizzazione di alcune dipendenze non necessarie in quanto non utilizzate;
                  \item La documentazione ufficiale di Plotly.js è vaga e poco esaustiva;
                  \item La versione ufficiale di Plotly.js presenta degli errori durante la registrazine delle \textit{traces}, impedendo di importare correttamente
                        \tetxit{Plotly} all'interno dei moduli TypeScript, motivo per il quale è stata utilizzata una versione custom-bundle di Plotly.js.
            \end{itemize}
      \item \textbf{Alternative Esaminate}:
            \begin{itemize}
                  \item \textit{D3.js}: D3.js è una libreria JavaScript open-source utilizzata per la generazione di grafici dinamici e manipolazione dati, la quale dalla sua parte risulta però
                        essere più complessa rispetto a Plotly.js, richiedendo una maggiore curva di apprendimento e una maggiore quantità di codice per la generazione di grafici.
                  \item \textit{Chart.js}: Chart.js è una libreria JavaScript open-source utilizzata per la generazione di grafici dinamici, la quale risulta essere più leggera di Plotly.js,
                        ma offre una minore quantità di grafici disponibili; il suo utilizzo avrebbe inoltre comportato la richiesta di modifiche lato backend.
            \end{itemize}
\end{itemize}

\subsubsection{TypeScript}
\begin{itemize}
      \item \textbf{Nome strumento}: TypeScript
      \item \textbf{Versione}: 5.1.6
      \item \textbf{Link}: \href{https://www.typescriptlang.org/}{typescriptlang.org}
      \item \textbf{Descrizione}: TypeScript è un linguaggio di programmazione open-source sviluppato da Microsoft, che fa da super-set a JavaScript.
      \item \textbf{Vantaggi}:
            \begin{itemize}
                  \item TypeScript permette di definire tipi ed interfacce per le variabili e i parametri delle funzioni, garantendo una maggiore sicurezza e affidabilità del codice;
                  \item TypeScript permette di effettuare controlli statici sul codice sorgente, riducendo il numero di errori a runtime.
            \end{itemize}
\end{itemize}

\subsection{Strumenti per lo Sviluppo}

\subsubsection{NVM}
\begin{itemize}
      \item \textbf{Nome strumento}: NVM
      \item \textbf{Versione}: 1.1.12
      \item \textbf{Link}: \href{https://github.com/nvm-sh/nvm}{github.com/nvm-sh/nvm}
      \item \textbf{Descrizione}: NVM (Node Version Manager) è uno strumento open-source utilizzato per la gestione delle versioni di Node.js.
      \item \textbf{Vantaggi}:
            \begin{itemize}
                  \item NVM permette di installare e gestire più versioni di Node.js all'interno del sistema, permettendo di selezionare la versione corretta
                        per il progetto in corso;
                  \item NVM permette di gestire le versioni di Node.js in modo semplice e veloce, permettendo di passare da una versione all'altra con un solo comando.
            \end{itemize}
\end{itemize}

\subsubsection{Rollup}
\begin{itemize}
      \item \textbf{Nome strumento}: Rollup
      \item \textbf{Versione}: 4.18.0
      \item \textbf{Link}: \href{https://rollupjs.org/}{rollupjs.org}
      \item \textbf{Descrizione}: Rollup è un bundler di moduli JavaScript che permette di risolvere le dipendenze tra i moduli, generando un unico file di output.
      \item \textbf{Vantaggi}:
            \begin{itemize}
                  \item Rollup effettua il tree-shaking, rimuovendo le dipendenze non utilizzate all'interno del codice sorgente;
                  \item Permette di generare bundle in diversi formati, quali \textit{CommonJS} e \textit{ESM};
                  \item Rollup è noto per la sua velocità ed efficienza nelle build, specialmente per progetti più piccoli o librerie.
                        Questo può risultare in tempi di build più rapidi e in una migliore esperienza di sviluppo;
                  \item Rollup ha un sistema di plug-in molto potente e flessibile che permette di estendere le sue funzionalità;
                  \item Rollup permette di utilizzare TypeScript all'interno del progetto, integrando la configurazione dichiarata nel file \textit{tsconfig.json};
                  \item Rollup permette di generare file di dichiarazione TypeScript.
            \end{itemize}
\end{itemize}

\subsubsection{Yarn}
\begin{itemize}
      \item \textbf{Nome strumento}: Yarn
      \item \textbf{Versione}: 1.22.22
      \item \textbf{Link}: \href{https://yarnpkg.com/}{yarnpkg.com}
      \item \textbf{Descrizione}: Yarn è un package manager per JavaScript, sviluppato da Facebook, Google e Tilde.
      \item \textbf{Vantaggi}:
            \begin{itemize}
                  \item Yarn è più veloce di npm, ha un sistema di cache più efficiente e permette di installare pacchetti in parallelo.
            \end{itemize}
      \item \textbf{Svantaggi}:
            \begin{itemize}
                  \item Yarn presenta un registry di dimensione minore rispetto a quello di \textit{npm}.
            \end{itemize}
      \item \textbf{Alternative Esaminate}:
            \begin{itemize}
                  \item \textit{npm}: npm è il package manager di default per Node.js, ma è più lento di Yarn e ha un sistema di cache meno efficiente,
                        offrendo inoltre un output meno comprensibile.
            \end{itemize}
\end{itemize}

\subsection{Strumenti per la Collaborazione e la Gestione del Progetto}
Nella seguente sezione vengono presentati gli strumenti utilizzati per la collaborazione e la gestione del progetto durante lo svolgimento dello stage.

\subsubsection{Confluence}
\begin{itemize}
      \item \textbf{Nome strumento}: Confluence
      \item \textbf{Versione}: Cloud
      \item \textbf{Link}: \href{https://www.atlassian.com/software/confluence}{atlassian.com/software/confluence}
      \item \textbf{Descrizione}: Confluence è un software di collaborazione sviluppato da Atlassian, utilizzato per la creazione e la gestione della documentazione
            all'interno di un'azienda.
\end{itemize}

\subsubsection{Slack}
\begin{itemize}
      \item \textbf{Nome strumento}: Slack
      \item \textbf{Versione}: Cloud
      \item \textbf{Link}: \href{https://slack.com/}{slack.com}
      \item \textbf{Descrizione}: Slack è un software di collaborazione sviluppato da Slack Technologies, utilizzato per la comunicazione quotidiana e la collaborazione
            con il team.
\end{itemize}

\section{Descrizione del prodotto ottenuto}
Il prodotto conseguito al termine dello stage consiste in una libreria TypeScript di componenti utilizzati nel processo di creazione di dashboard dinamiche,
ottenuto a partire dal refactor e l'ottimizzazione della libreria open-source utilizzata all'interno della piattaforma \textit{Redash}.
La libreria implementata, sotto il nome di \textit{viz-lib}, è stata infatti integrata assieme alla già esistente libreria \textit{dashboard} all'interno del SDK
\textit{dsdashboard2}, utilizzato all'interno dei prodotti Datasoil S.r.l. per la generazione di dashboard dinamiche.
La libreria \textit{viz-lib} offre una vasta gamma di grafici, quali:
\begin{itemize}
      \item Contatori;
      \item Grafici ad area;
      \item Grafici a barre;
      \item Grafici a bolle;
      \item Grafici a dispersione;
      \item Grafici a istogrammi;
      \item Grafici a linee;
      \item Grafici a torta;
      \item Tabelle.
\end{itemize}
Grazie al refactor e all'ottimizzazione della libreria preesistente, l'SDK prodotto risulta essere più performante, più manutenibile e soprattutto più leggero rispetto
alla versione preesistente, con un bundle di dimensioni 1.26 MB → 418.03 kB (gzip) rispetto ai 4.11 MB → 1.203 MB (gzip) iniziali.

\section{Organizzazione del testo}
Nel presente capitolo viene presentata l'introduzione della tesi, fornendo una panoramica sull'azienda, sul contesto applicativo,
sul progetto, sugli strumenti utilizzati e sul prodotto portato a termine durante lo svolgimento dello stage. \newline
In seguito il documento presenterà la seguente organizzazione:

\begin{description}
      \item[{\hyperref[chap:analisi-requisiti]{Analisi dei requisiti}}] descrive la fase di analisi dei requisiti che è stata
            svolta dall'azienda in fase antecedente all'inizio dell'attività di stage, in modo da permettere una comprensione più profonda
            di quelle che sono le necessità da soddisfare e gli obiettivi da raggiungere all'interno di questo progetto;

      \item[{\hyperref[chap:progettazione]{Progettazione}}] illustra l'attività di progettazione che è stata svolta in vista dell'implementazione
            dei grafici prodotti, definendo e individuando le soluzioni implementative che sono state attuate durante la successiva attività di codifica;

      \item[{\hyperref[chap:realizzazione-testing]{Realizzazione e Testing}}] approfondisce l'attività di codifica delle componenti grafiche presentate
            all'interno della libreria implementata e dei relativi test effettuati per garantire la qualità e la funzionalità del prodotto finale;

      \item[{\hyperref[chap:rilascio]{Rilascio}}] descrive l'attività di rilascio dell'SDK realizzato, presentando le modalità di integrazione
            all'interno di un prodotto Datasoil S.r.l.;

      \item[{\hyperref[chap:conclusioni]{Conclusioni}}] presentano un epilogo del progetto svolto, includendo una valutazione oggettiva
            degli strumenti utilizzati e una riflessione sui possibili punti di insoddisfazione e dei relativi miglioramenti applicabili al prodotto realizzato.
            La sezione infine si conclude con proposte per future estensioni e sviluppi del progetto.
\end{description}
In merito alla stesura del testo, all'interno del presente documento sono state adottate le seguenti convenzioni tipografiche:
\begin{itemize}
      \item gli acronimi, le abbreviazioni e i termini ambigui o di uso non comune menzionati vengono definiti nel glossario, situato alla fine del presente documento;
      \item per la prima occorrenza dei termini riportati nel glossario viene utilizzata la seguente nomenclatura: \textit{parola}\glox\gloxspacing;
      \item i termini in lingua straniera o facenti parti del gergo tecnico sono evidenziati con il carattere \textit{corsivo}.
\end{itemize}


\newpage
    \chapter{Lo stage}
\label{chap:descrizione-stage}
Nel seguente capito si descriverà in dettaglio la proposta di stage accettata, indicandonene gli obiettivi, la pianificazione
delle attività, i prodotti attesi, gli strumenti e le tecnologie utilizzate, terminando con le motivazioni che hanno portato 
alla scelta di tale proposta.


\section{Introduzione al progetto}
Lo 

\section{Analisi preventiva dei rischi}

Durante la fase di analisi iniziale sono stati individuati alcuni possibili rischi a cui si potrà andare incontro.
Si è quindi proceduto a elaborare delle possibili soluzioni per far fronte a tali rischi.

\begin{risk}{Performance del simulatore hardware}
    \riskdescription{le performance del simulatore hardware e la comunicazione con questo potrebbero risultare lenti o non abbastanza buoni da causare il fallimento dei test}
    \risksolution{coinvolgimento del responsabile a capo del progetto relativo il simulatore hardware}
    \label{risk:hardware-simulator} 
\end{risk}

\section{Requisiti e obiettivi}

\begin{center}
    \rowcolors{1}{}{tableGray}
    \begin{longtable}{|p{2.25cm}|p{7.75cm}|p{2.25cm}|}
    \hline
    \multicolumn{1}{|c|}{\textbf{A}} & \multicolumn{1}{c|}{\textbf{B}}\\ 
    \hline 
    \endfirsthead
    \rowcolor{white}
    \multicolumn{3}{c}{{\bfseries \tablename\ \thetable{} -- Continuo della tabella}}\\
    \hline
    \multicolumn{1}{|c|}{\textbf{A}} & \multicolumn{1}{c|}{B}\\ \hline 
    \endhead
    \hline
    \rowcolor{white}
    \multicolumn{3}{|r|}{{Continua nella prossima pagina...}}\\
    \hline
    \endfoot
    \endlastfoot 
    
    AA & BB \\
    \hline
    AA & BB \\
    \hline
    AA & BB \\
    \hline
    AA & BB \\
    \hline
    \hiderowcolors
    \caption{Lorem.}
    \label{tab:requisiti_obbiettivi}
    \end{longtable}
\end{center}

\section{Pianificazione}
\begin{figure}[H]
    \centering
    \includegraphics[alt={Testo alternativo dell'immagine}, width=0.5\columnwidth]{img/pk_estate.jpeg}
    \caption{Caption}
    \label{fig:pk_estate_2}
\end{figure}
\lipsum[1]

\begin{enumerate}
    \item[\textbf{F}:] Funzionale.
    \item[\textbf{Q}:] Qualitativo.
    \item[\textbf{V}:] Di vincolo.
    \item[\textbf{N}:] Obbligatorio (necessario).
    \item[\textbf{D}:] Desiderabile.
    \item[\textbf{Z}:] Opzionale.
\end{enumerate}

Nelle tabelle \ref{tab:requisiti_funzionali}, \ref{tab:requisiti_qualitativi} e \ref{tab:requisiti_vincolo} sono riassunti i requisiti e il loro tracciamento con gli use case delineati in fase di analisi.

\section{Tabelle dei requisiti}
\begin{center}
    \rowcolors{1}{}{tableGray}
    \begin{longtable}{|p{2.25cm}|p{7.75cm}|p{2.25cm}|}
    \hline
    %\rowcolor{hyperColor!5}
    \multicolumn{1}{|c|}{\textbf{Requisito}} & \multicolumn{1}{c|}{\textbf{Descrizione}} & \multicolumn{1}{c|}{\textbf{Use Case}}\\
    \hline 
    \endfirsthead
    \rowcolor{white}
    \multicolumn{3}{c}{{\bfseries \tablename\ \thetable{} -- Continuo della tabella}}\\
    \hline
    %\rowcolor{hyperColor!5}
    \multicolumn{1}{|c|}{\textbf{Requisito}} & \multicolumn{1}{c|}{\textbf{Descrizione}} & \multicolumn{1}{c|}{\textbf{Use Case}}\\
    \hline 
    \endhead
    \hline
    \rowcolor{white}
    \multicolumn{3}{|r|}{{Continua nella prossima pagina...}}\\
    \hline
    \endfoot
    \endlastfoot
    
    RFN-1 & L’interfaccia permette di configurare il tipo di sonde del test & UC1 \\
    \hline
    \hiderowcolors
    \caption{Tabella del tracciamento dei requisiti funzionali.}
    \label{tab:requisiti_funzionali}
    \end{longtable}
\end{center}

\begin{center}
    \rowcolors{1}{}{tableGray}
    \begin{longtable}{|p{2.25cm}|p{7.75cm}|p{2.25cm}|}
    \hline
    %\rowcolor{hyperColor!5}
    \multicolumn{1}{|c|}{\textbf{Requisito}} & \multicolumn{1}{c|}{\textbf{Descrizione}} & \multicolumn{1}{c|}{\textbf{Use Case}}\\
    \hline 
    \endfirsthead
    \rowcolor{white}
    \multicolumn{3}{c}{{\bfseries \tablename\ \thetable{} -- Continuo della tabella}}\\
    \hline
    %\rowcolor{hyperColor!5}
    \multicolumn{1}{|c|}{\textbf{Requisito}} & \multicolumn{1}{c|}{\textbf{Descrizione}} & \multicolumn{1}{c|}{\textbf{Use Case}}\\
    \hline 
    \endhead
    \hline
    \rowcolor{white}
    \multicolumn{3}{|r|}{{Continua nella prossima pagina...}}\\
    \hline
    %\caption{Tabella del tracciamento dei requisiti qualitativi.}
    \endfoot
    \endlastfoot
    
    RQD-1n & Le prestazioni del simulatore hardware deve garantire la giusta esecuzione dei test e non la generazione di falsi negativi & - \\
    \hline
    RQD-1n & Le prestazioni del simulatore hardware deve garantire la giusta esecuzione dei test e non la generazione di falsi negativi & - \\
    \hline
    RQD-1n & Le prestazioni del simulatore hardware deve garantire la giusta esecuzione dei test e non la generazione di falsi negativi & - \\
    \hline
    RQD-1n & Le prestazioni del simulatore hardware deve garantire la giusta esecuzione dei test e non la generazione di falsi negativi & - \\
    \hline
    RQD-1n & Le prestazioni del simulatore hardware deve garantire la giusta esecuzione dei test e non la generazione di falsi negativi & - \\
    \hline
    RQD-1n & Le prestazioni del simulatore hardware deve garantire la giusta esecuzione dei test e non la generazione di falsi negativi & - \\
    \hline
    \hiderowcolors
    \caption{Tabella del tracciamento dei requisiti qualitativi.}
    \label{tab:requisiti_qualitativi}
    \end{longtable}
\end{center}

\begin{center}
    \rowcolors{1}{}{tableGray}
    \begin{longtable}{|p{2.25cm}|p{7.75cm}|p{2.25cm}|}
    \hline
    %\rowcolor{hyperColor!5}
    \multicolumn{1}{|c|}{\textbf{Requisito}} & \multicolumn{1}{c|}{\textbf{Descrizione}} & \multicolumn{1}{c|}{\textbf{Use Case}}\\
    \hline 
    \endfirsthead
    \rowcolor{white}
    \multicolumn{3}{c}{{\bfseries \tablename\ \thetable{} -- Continuo della tabella}}\\
    \hline
    %\rowcolor{hyperColor!5}
    \multicolumn{1}{|c|}{\textbf{Requisito}} & \multicolumn{1}{c|}{\textbf{Descrizione}} & \multicolumn{1}{c|}{\textbf{Use Case}}\\
    \hline 
    \endhead
    \hline
    \rowcolor{white}
    \multicolumn{3}{|r|}{{Continua nella prossima pagina...}}\\
    \hline
    \endfoot
    \endlastfoot
    
    RVO-1 & La libreria per l'esecuzione dei test automatici deve essere riutilizzabile & - \\
    \hline
    \hiderowcolors
    \caption{Tabella del tracciamento dei requisiti di vincolo.}
    \label{tab:requisiti_vincolo}
    \end{longtable}
\end{center}

\subsection{subsection}
\lipsum[1]

\subsubsection{subsubsection}
\lipsum[1]

\paragraph{paragraph}
\lipsum[1]

\newpage
    \chapter{Fondamenti di Refactoring e Ottimizzazione}
\label{chap:fondamenti-refactoring-ottimizzazione}

\section{Definizione di Refactoring e Ottimizzazione}

\subsection{Refactoring}
Il refactoring del software è una attività di manutenzione che prevede la ristrutturazione del codice sorgente senza modificarne 
il comportamento esterno. L'obiettivo che si pone tale attività consiste nel migliorare la leggibilità e la manutenibilità del codice,
riducendo la complessità e rendendo più semplice l'aggiunta di nuove funzionalità. \newline
I vantaggi che si possono ottenere dal refactoring sono molteplici e impliciti nella qualità del codice prodotto, permettendo di ridurre il 
tempo necessario per la correzione di bug e l'introduzione di nuove funzionalità, grazie all'agevolazione della comprensione del codice.
% TODO: aggiungere riferimento ISO/IEC . Esempio JS -> TS

\subsection{Ottimizzazione}
L'ottimizzazione del software è un'attività di manutenzione che prevede la modifica del codice sorgente: in quanto l'ambito delle ottimizzazioni 
risulta essere vasto e con aspetti fortemente legati al contesto di attuazione, in questa tesi verranno presentate le sole ottimizzazioni operate
durante lo svolgimento dell'attività di stage. \newline
% sorgente per migliorarne le prestazioni, 
% l'utilizzo delle risorse, la scalabilità e l'efficienza del codice stesso.


\newpage
    \input{chapters/4_viz_lib}
    \input{chapters/5_verify}
    \input{chapters/6_processes}
    \input{chapters/7_conclusions}


    % Introduzione
    %     Presentazione dell'argomento della tesi.
    %     Obiettivi della tesi.
    %     Struttura della tesi.

    % Lo Stage
    %     Descrizione dell'azienda ospitante.
    %     Contesto del progetto e obiettivi aziendali.
    %     Ruolo e responsabilità durante lo stage.

    % Fondamenti di Refactoring e Ottimizzazione
    %     Definizione di refactoring e ottimizzazione del software.
    %     Importanza del refactoring per la manutenzione del codice.
    %     Principi e best practices del refactoring.
    %     Tipi di ottimizzazioni (prestazionali, di memoria, di leggibilità, ecc.).

    % La Libreria Grafica Viz Lib
    %     Descrizione della libreria grafica esistente.
    %     Analisi dello stato iniziale della libreria.
    %     Obiettivi del refactoring e dell'ottimizzazione.

    % Processo di Refactoring della Libreria Viz Lib
    %     Metodologia adottata per il refactoring.
    %     Strumenti utilizzati (IDE, plugin, software di analisi statica, ecc.).
    %     Principali modifiche e miglioramenti apportati.
    %     Sfide incontrate e soluzioni adottate.

    % Ottimizzazioni Implementate
    %     Ottimizzazioni delle prestazioni: tecniche utilizzate (caching, lazy loading, riduzione della complessità computazionale, ecc.).
    %     Ottimizzazioni della memoria: gestione della memoria, riduzione del footprint, ecc.
    %     Miglioramento della leggibilità del codice: riorganizzazione del codice, naming convention, commenti, ecc.
    %     Test e validazione delle ottimizzazioni: benchmarking, profiling, testing automatico.

    % Risultati e Benefici
    %     Risultati quantitativi delle ottimizzazioni (tempi di esecuzione, utilizzo della memoria, ecc.).
    %     Benefici qualitativi (manutenibilità, leggibilità, facilità di estensione, ecc.).
    %     Feedback ricevuto dall'azienda e dagli utenti della libreria.

    % Strumenti e Metodi di Lavoro
    %     Comunicazione e Collaborazione
    %         Uso di Slack per la comunicazione quotidiana e la collaborazione con il team.
    %         Gestione delle riunioni e degli aggiornamenti di stato.
    %     Gestione della Documentazione
    %         Utilizzo di Confluence per la creazione e la gestione della documentazione del progetto.
    %         Descrizione dei tipi di documentazione prodotti (manuali utente, documentazione tecnica, report di avanzamento, ecc.).
    %     Gestione del Codice e del Versionamento
    %         Utilizzo di sistemi di controllo versione (come Git).
    %         Workflow di gestione del codice (branching, pull requests, code reviews).
    %     Strumenti di Sviluppo e Analisi
    %         IDE e plugin utilizzati.
    %         Strumenti per il refactoring e l'analisi del codice (es. SonarQube, linters, profiler).

    % Considerazioni Finali e Conclusioni
    %     Riflessioni sull'esperienza di stage.
    %     Importanza delle competenze acquisite durante lo stage.
    %     Impatto delle ottimizzazioni sulla libreria e sull'azienda.
    %     Suggerimenti per futuri sviluppi e ottimizzazioni.

    % Bibliografia
    %     Elenco delle fonti utilizzate per la stesura della tesi (articoli, libri, documentazione tecnica, ecc.).

    % Appendici (se necessario)
    %     Codice sorgente rilevante.
    %     Risultati dei test di benchmarking.
    %     Documentazione aggiuntiva.

    \pagenumbering{roman}
    \backmatter
    \input{references/bibliography}
    \input{references/webliography}
\end{document}