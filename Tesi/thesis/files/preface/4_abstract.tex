\cleardoublepage
\phantomsection
\pdfbookmark{Compendio}{Compendio}
\begingroup
\let\clearpage\relax
\let\cleardoublepage\relax
\chapter*{Sommario}

Il presente documento descrive il lavoro svolto durante il periodo di stage, della durata di X ore, dal laureando Gianluca Bresolin
presso l'azienda Datasoil S.r.l. Lo stage ha previsto la realizzazione di una libreria ReactJS attraverso il refactor
e l'ottimizzazione di una libreria per la visualizzazione di dashboard parametriche, volta ad ampliare soluzioni e servizi 
nel contesto di applicazioni SPA e nella realizzazione di interfacce utente, ponendo attenzione ai requisiti di scalabilità,
usabilità e performance. \n
La prima attività svolta è stata l'analisi del codice sorgente della libreria esistente, per comprendere il funzionamento e le possibilità
di ottimizzazione. Successivamente, sono state individuate le funzionalità da mantenere, quelle da migliorare e quelle da aggiungere,
per poi procedere con la progettazione e l'implementazione della nuova libreria. \n
Infine, la libreria realizzata è stata integrata nei prodotti Datasoil attraverso un micro servizio dedicato, rendendo disponibile un SDK
compatibile le risposte delle API Datasoil, senza richiedere modifiche ai prodotti esistenti lato server. \n
Il risultato ottenuto è stato una libreria ReactJS che permette agli sviluppatori di comporre in maniera rapida e intuitiva dashboard
dinamiche e personalizzate per la visualizzazione dei dati, riducendo la dimensione della codebase e aggiungendo nuovi grafici e funzionalità
rispetto all'SDK precedentemente utilizzato dall'azienda.


\endgroup
\vfill
