\cleardoublepage
\phantomsection
\pdfbookmark{Compendio}{Compendio}
\begingroup
\let\clearpage\relax
\let\cleardoublepage\relax
\chapter*{Sommario}

Il presente documento descrive il lavoro svolto durante il periodo di stage, della durata di 304 ore, dal laureando Gianluca Bresolin
presso l'azienda Datasoil S.r.l. Lo stage ha previsto la realizzazione di una libreria ReactJS attraverso il refactor
e l'ottimizzazione di una libreria utilizzata per la visualizzazione di dashboard parametriche, volta ad ampliare soluzioni e servizi
nel contesto di applicazioni SPA e nella realizzazione di interfacce utente, ponendo attenzione ai requisiti di scalabilità,
usabilità e performance. \newline
La prima attività condotta è stata l'analisi del codice sorgente della libreria esistente, in modo da poter comprendere il funzionamento e le opportunità
di ottimizzazione. Successivamente, sono state individuate le funzionalità da mantenere, quelle da migliorare e quelle da introdurre,
per poi procedere con la progettazione e l'implementazione della nuova libreria. \newline
Infine, la libreria realizzata è stata integrata nei prodotti Datasoil attraverso un micro servizio dedicato, rendendo disponibile un SDK
compatibile con le risposte delle API Datasoil, senza richiedere modifiche lato server ai prodotti esistenti. \newline
Il risultato ottenuto al termine del periodo di stage consiste un SDK per la composizione dinamica di dashboard, dalle dimensioni ridotte rispetto
al boundle precedentemente utilizzato dall'azienda e dotato di una maggior robustezza e consistenza grazie ad un uso più opportuno di \textit{TypeScript},
permettendo inoltre di raggiungere una migliore aderenza a quelle che sono le esigenze dei clienti, grazie alla rimozione di grafici inutilizzati e
l'introduzione di nuove funzionalità.


\endgroup
\vfill
