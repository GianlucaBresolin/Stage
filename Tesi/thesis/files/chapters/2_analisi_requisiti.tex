\chapter{Analisi dei requisiti}
\label{chap:analisi-requisiti}
Nel seguente capitolo verrà descritta l'attività di analisi dei requisiti inerente al progetto di stage. \newline
In quanto l'SDK in questione è uno strumento live sui diversi prodotti Datasoil S.r.l., l'analisi dei requisiti è stata svolta
in fase antecedente all'inizio dello stage da parte del team di sviluppo frontend dell'azienda. \newline
A fronte di tale contesto, nella seguente sezione verranno riportati i requisiti individuati in modo da poter fornire il contesto
necessario alla comprensione del progetto, senza approfondire l'attività di analisi svolta.

\section{Classificazione dei requisiti}
I requisiti presentati all'interno del documento vengono classificati in tre categorie, in base alle esigenze che definiscono, venendo inoltre
associati ad un livello di priorità che ne determina l'obbligatorietà o l'opzionalità. \newline

\newline
Le categorie di classificazione definite all'interno del progetto di stage sono:

\begin{itemize}
    \item[\textbf{F}:] \textit{Funzionale}: il requisito definisce le funzionalità che il sistema deve offrire;
    \item[\textbf{A}:] \textit{Di attributo}: il requisito definisce le caratteristiche che il sistema deve avere;
    \item[\textbf{V}:] \textit{Di vincolo}: il requisito definisce i vincoli che il sistema deve rispettare.
\end{itemize}

\newline
I livelli di priorità definiti all'interno del progetto di stage sono:

\begin{itemize}
    \item[\textbf{O}:] \textit{Obbligatorio}: requisito essenziale al fine di concludere il progetto di stage con successo;
    \item[\textbf{OP}:] \textit{Opzionale}: requisito non essenziale al fine di concludere il progetto di stage con successo, ma che aggiunge valore al prodotto finale.
\end{itemize}

\newline
La nomeclatura utilizzata per la classificazione dei requisiti all'interno del progetto di stage è la seguente:
\begin{itemize}
    \item \textbf{R}: dove R sta per requisito;
    \item \textbf{X.}: dove X indica la categoria di classificazione del requisito;
    \item \textbf{Y.}: dove Y indica il livello di priorità del requisito;
    \item \textbf{Z}: dove Z indica il numero progressivo del requisito. Nel caso di sotto-requisiti, il numero progressivo è composto da più cifre, separate da un punto,
          dove la cifra antecedente rappresenta il requisito padre e la cifra conseguente rappresenta il sotto-requisito.
\end{itemize}

\newpage
\section{Requisiti}

\subsection{Requisiti funzionali}
Nella seguente sezione viene presentata la lista dei requisiti funzionali.

\begin{center}
    \rowcolors{1}{white}{tableGray}
    \begin{longtable}{|p{2.5cm}|p{10cm}|}
        \hline
        \rowcolor{gray!30}
        \textbf{Requisito} & \multicolumn{1}{c|}{\textbf{Descrizione}}                                                                    \\
        \hline
        \endfirsthead
        \hline
        \multicolumn{2}{|c|}{{\tablename\ \thetable{} -- continua dalla pagina precedente}}                                               \\
        \hline
        \rowcolor{gray!30}
        \textbf{Requisito} & \multicolumn{1}{c|}{\textbf{Descrizione}}                                                                    \\
        \endhead
        \hline
        \multicolumn{2}{|r|}{{Continua nella prossima pagina...}}                                                                         \\
        \hline
        \endfoot
        \hline
        \endlastfoot
        R.F.O.1            & La libreria deve permettere la renderizzazione di grafici.                                                   \\
        \hline
        R.F.O.1.1          & La libreria deve permettere la renderizzazione di grafici a barre (\textit{bar or columns}).                 \\
        \hline
        R.F.O.1.2          & La libreria deve permettere la renderizzazione di grafici a torta (\textit{pie}).                            \\
        \hline
        R.F.O.1.3          & La libreria deve permettere la renderizzazione di grafici a scatola (\textit{box-plot}).                     \\
        \hline
        R.F.O.1.4          & La libreria deve permettere la renderizzazione di grafici a bolle (\textit{bubbles}).                        \\
        \hline
        R.F.O.1.5          & La libreria deve permettere la renderizzazione di grafici a dispersione (\textit{scatter}).                  \\
        \hline
        R.F.O.1.6          & La libreria deve permettere la renderizzazione di grafici a mappe di calore (\textit{heatmap}).              \\
        \hline
        R.F.O.2            & La libreria deve permettere la renderizzazione di contatori.                                                 \\
        \hline
        R.F.O.2.1          & La libreria deve permettere la renderizzazione di contatori con target.                                      \\
        \hline
        R.F.OP.2.1.1       & La libreria deve permettere la renderizzazione di contatori con target, definendo una grafica
        differente a seconda del raggiungimento o meno del target.                                                                        \\
        \hline
        R.F.O.2.2          & La libreria deve permettere la renderizzazione di contatori con formattazione personalizzata.                \\
        \hline
        R.F.O.2.3          & La libreria deve permettere la renderizzazione di contatori con tooltip con formattazione personalizzata.    \\
        \hline
        R.F.O.3            & La libreria deve permettere la renderizzazione di tabelle.                                                   \\
        \hline
        R.F.O.3.1          & La libreria deve permettere la renderizzazione di tabelle con ordinamento lungo le varie colonne.            \\
        \hline
        R.F.O.3.2          & La libreria deve permettere la renderizzazione di tabelle con filtro globale che agisce lungo tutti i dati.  \\
        \hline
        R.F.O.3.3          & La libreria deve permettere la renderizzazione di tabelle con le colonne ordinate secondo l'evenutale ordine
        fornito dal backend.                                                                                                              \\
        \hline
        R.F.O.4            & La libreria deve permettere lo zoom nei i vari grafici.                                                      \\
        \hline
        R.F.O.5            & La libreria deve permettere di effettuare il download in \textit{.png} dei grafici renderizzati.             \\
        \hline
        R.F.O.6            & La libreria deve permettere di effettuare il download in \textit{.csv} delle tabelle visualizzate.           \\
        \hline
        R.F.O.7            & I widget renderizzati devono adattarsi alla dimensione del contenitore in modo responsivo.                   \\
        \hline
        R.F.O.8            & La libreria deve permettere di visualizzare i widget in modalità \textit{fullscreen} attraverso
        l'utilizzo di \textit{Dialog}.                                                                                                    \\
    \end{longtable}
    \captionof{table}{Tabella del tracciamento dei requisiti funzionali.}
    \label{tab:requisiti_funzionali}
\end{center}

\subsection{Requisiti qualitativi}
Nella seguente sezione viene presentata la lista dei requisiti qualitativi.

\begin{center}
    \rowcolors{1}{white}{tableGray}
    \begin{longtable}{|p{2.5cm}|p{10cm}|}
        \hline
        \rowcolor{gray!30}
        \textbf{Requisito} & \multicolumn{1}{c|}{\textbf{Descrizione}}                                                                                    \\
        \hline
        \endfirsthead
        \hline
        \multicolumn{2}{|c|}{{\tablename\ \thetable{} -- continua dalla pagina precedente}}                                                               \\
        \hline
        \rowcolor{gray!30}
        \textbf{Requisito} & \multicolumn{1}{c|}{\textbf{Descrizione}}                                                                                    \\
        \endhead
        \hline
        \multicolumn{2}{|r|}{{Continua nella prossima pagina...}}                                                                                         \\
        \hline
        \endfoot
        \hline
        \endlastfoot
        \hline
        R.A.O.1            & La libreria deve essere facilmente manutenibile, rispettando le convenzioni di codifica
        attualmente in atto all'interno dell'azienda Datasoil S.r.l.                                                                                      \\
        \hline
        R.A.O.2            & La libreria deve essere performante in termini di velocità di rendering dei widget forniti.                                  \\
        \hline
        R.A.O.3            & La libreria deve essere ottimizzata per richiedere il minor numero di rendering possibili durante l'interazione dell'utente,
        eseguendoli in modo efficiente per minimizzare l'uso delle risorse.                                                                               \\
        \hline
        R.A.O.3            & La libreria deve essere leggera in termini di dimensione di memoria occupata.                                                \\
        \hline
        R.A.O.3.1          & Le dipendenze esterne devono essere ridotte al minimo indispensabile.                                                        \\
        \hline
        R.A.0.3.2          & La libreria deve essere composta esclusivamente da codice effettivamente utilizzato per le funzionalità e i widget resi
        disponibili. Ciò comporta che qualsiasi componente di codice inutilizzata, superflua o non necessaria deve essere identificata e rimossa.         \\
    \end{longtable}
    \captionof{table}{Tabella del tracciamento dei requisiti qualitativi.}
    \label{tab:requisiti_qualitativi}
\end{center}

\subsection{Requisiti di vincolo}
Nella seguente sezione viene presentata la lista dei requisiti di vincolo.

\begin{center}
    \rowcolors{1}{white}{tableGray}
    \begin{longtable}{|p{2.5cm}|p{10cm}|}
        \hline
        \rowcolor{gray!30}
        \textbf{Requisito} & \multicolumn{1}{c|}{\textbf{Descrizione}}                                                 \\
        \hline
        \endfirsthead
        \hline
        \multicolumn{2}{|c|}{{\tablename\ \thetable{} -- continua dalla pagina precedente}}                            \\
        \hline
        \rowcolor{gray!30}
        \textbf{Requisito} & \multicolumn{1}{c|}{\textbf{Descrizione}}                                                 \\
        \endhead
        \hline
        \multicolumn{2}{|r|}{{Continua nella prossima pagina...}}                                                      \\
        \hline
        \endfoot
        \hline
        \endlastfoot
        \hline
        R.V.O.1            & La libreria deve essere sviluppata in \textit{TypeScript}.                                \\
        \hline
        R.V.O.2            & La libreria deve essere sviluppata con \textit{React}.                                    \\
        \hline
        R.V.O.3            & La libreria deve essere sviluppata con l'utilizzo di componenti di \textit{PrimeReact}.   \\
        \hline
        R.V.O.4            & La libreria deve essere sviluppata con l'utilizzo di \textit{PrimeFlex}.                  \\
        \hline
        R.V.O.5            & La libreria deve essere sviluppata con l'utilizzo di \textit{PrimeIcons}.                 \\
        \hline
        R.V.O.6            & Il codice della libreria deve essere formattato mediante l'utilizzo di \textit{Prettier}. \\
        \hline
        R.V.OP.7           & Il processo di rialscio deve avvenire all'interno del registro \textit{Github Packages}
        di Datasoil S.r.l.                                                                                             \\
    \end{longtable}
    \captionof{table}{Tabella del tracciamento dei requisiti di vincolo.}
    \label{tab:requisiti_vincolo}
\end{center}
