\chapter{Introduzione}
\label{chap:introduzione}

\section{L'azienda}
Lo stage è stato svolto presso l'azienda \textit{Datasoil S.r.l.} situata nei dintorni della stazione ferroviaria di Padova.
Fondata nel 2016, \textit{Datasoil S.r.l.} è un'azienda di prodotto che si dedica allo sviluppo di piattaforme per l'industria 4.0,
\textit{smart building} e \textit{smart city}.
L'obiettivo dell'azienda è integrare informazioni ed eventi provenienti dai vari livelli aziendali per creare \textit{insight} proattivi
in tempo reale, garantendo che le informazioni corrette raggiungano le persone giuste al momento più opportuno.
Il punto di partenza è la spazialità aziendale e tutti gli \textit{asset} che risiedono all'interno di essa, da cui
provengono tutti i dati raccolti, i quali vengono analizzati trasversalmente grazie a sempre più persone e dispositivi connessi.
Da qui il nome dell'azienda, \textit{Datasoil}: fare emergere le informazioni da questo suolo fertile di dati da cui siamo circondati.

\begin{figure}[H]
      \centering
      \includegraphics[alt={Logo \textit{Datasoil S.r.l.}}, width=0.25\columnwidth]{img/datasoil_logo.jpg}
      \caption{Logo \textit{Datasoil S.r.l.}}
      \label{fig:datasoil}
\end{figure}

\section{L'idea}
\subsection{Il contesto applicativo}
L'azienda \textit{Datasoil S.r.l.}, essendo un'azienda di prodotto, nasce con l'idea di proporre servizi di monitoraggio e controllo
di impianti industriali, \textit{smart building} e \textit{smart city}, offrendo come principale prodotto la piattaforma SYN.

\begin{figure}[H]
      \centering
      \includegraphics[alt={Logo SYN}, width=0.25\columnwidth]{img/syn_logo.jpg}
      \caption{Logo SYN}
      \label{fig:syn}
\end{figure}

SYN è una piattaforma di monitoraggio e controllo di impianti industriali che raccoglie dati da sensori e dispositivi
distribuiti all'interno di un impianto o su più stabilimenti, permettendo di visualizzare in tempo reale lo stato di
funzionamento dei vari \textit{asset}, di analizzare i dati raccolti e di attuare azioni di controllo o segnalazione tramite
\textit{ticketing}. All'interno della piattaforma, a seconda dei piani attivi del cliente, è possibile visualizzare dashboard dinamiche
in merito ad informazioni filtrate e aggiornate \textit{live}, permettendo di raggiungere una panoramica completa dello stato dei vari
\textit{asset} monitorati in modo rapido, intuitivo ed efficace, grazie all'utilizzo di molteplici grafici e \textit{widget}.

\subsection{Il progetto}
Il progetto svolto durante lo stage consiste nello sviluppo di una libreria \textit{TypeScript} di componenti per la creazione di dashboard dinamiche.
Questo è stato realizzato tramite il \gls{refactoring}\glox e l'ottimizzazione di un \textit{tool} grafico preesistente, integrato nei vari prodotti di \textit{Datasoil S.r.l.},
tra cui \textit{SYN}. La libreria è stata implementata a partire da un modulo \textit{\gls{open-source}}\glox utilizzato in \textit{Redash} (\href{https://redash.io}{redash.io}),
una piattaforma per la creazione di dashboard dinamiche tramite interrogazioni sulle fonti di dati configurate all'interno del servizio. \newline
L'esigenza di tale progetto nasce dalla necessità da parte di \textit{Datasoil S.r.l.} di avere una libreria grafica aggiornata alle versioni
correnti delle dipendenze utilizzate, migliorando l'\gls{FCP}\glox e il \gls{TTI}\glox, raggiungendo una maggiore manutenibilità del codice rispetto alla versione
preesistente, riducendo dove possibile le dipendenze esterne e introducendo migliorie grafiche e funzionali. \newline
La libreria sviluppata sotto il nome '\textit{viz-lib}', costituisce, insieme alla già esistente libreria '\textit{dashboard}' (la quale ha subito anch'essa in parte un processo di refactoring
per permettere l'integrazione con la nuova libreria e l'introduzione di nuove funzionalità), il nuovo \gls{SDK}\glox '\textit{dsdashbord2}'
utilizzato all'interno dei prodotti \textit{Datasoil S.r.l.} per la realizzazione di dashboard dinamiche.

\section{Principali problematiche}
Durante l'analisi iniziale della libreria preesistente in uso all'interno dei prodotti di \textit{Datasoil S.r.l.}, sono emerse alcune problematiche
legate alla manutenibilità e alla performance del \textit{tool} grafico a seguito di una attenta revisione del codice sorgente. \newline
Di seguito vengono presentate le principali problematiche riscontrate.

\subsection{Dipendenze obsolete}
La libreria preesistente utilizzava versioni obsolete delle dipendenze esterne, con conseguente degradazione delle prestazioni in termini
di utilizzo di spazio e di tempo di caricamente delle risorse, data l'assenza di ottimizzazioni e di aggiornamenti del codice sorgente. \newline
Essendo inoltre la libreria preesistente basata su una versione del modulo di visualizzazioni utilizzato nella piattaforma \textit{Redash},
le tecnologie utilizzate il più delle volte rappresentavano alternative a quelle già utilizzate nei prodotti \textit{Datasoil S.r.l.}, creando dipendenze
non necessarie, con conseguente aumento delle dimensioni del SDK utilizzato per la generazione delle dashboard.

\subsection{Manutenibilità del codice}
Il codice sorgente della libreria preesistente, sviluppato in tempistiche rapide a fronte di una esigenza specifica dell'azienda \textit{Datasoil S.r.l.},
risultava essere poco manutenibile e assente di documentazione. \newline
Inoltre, la libreria di visualizzazioni utilizzata da \textit{Redash} su cui si basa l'SDK utilizzato dall'azienda, inizialmente implementata in JavaScript,
ha subito solo successivamente un refactoring in TypeScript: questo refactoring non è però avvenuto con l'introduzione di interfacce e tipi, bensì con la semplice aggiunta
di tipi \textit{any} per le variabili e per i parametri delle funzioni, introducendo numerosi \textit{@ts-ignore} per ignorare gli errori di compilazione, rendendo il codice
poco leggibile e difficile da mantenere.

\subsection{Componenti inutilizzate}
In quanto la libreria preesistente fosse realizzata a partire dal modulo utilizzato dalla piattaforma \textit{Redash}, non tutti i componenti presenti
in essa trovavano utilizzo all'interno dei prodotti realizzati da \textit{Datasoil S.r.l.}: questo in gran parte era dovuto ad un'offerta di grafici non adeguati
a quello che è il contesto applicativo dell'azienda.

\subsection{Assenza di funzionalità}
La libreria preesistente non soddisfaceva pienamente le esigenze dei clienti di \textit{Datasoil S.r.l.}, mancando di alcune funzionalità fondamentali,
quali la possibilità di effettuare il \textit{download} dei dati tabellari o la visualizzazione a pagina piena dei vari \textit{widget}.
Queste limitazioni hanno reso necessario un intervento di refactoring per colmare le lacune e migliorare le prestazioni complessive.

\section{Soluzione scelte}
Per risolvere le problematiche emerse durante l'analisi iniziale della libreria preesistente, si è optato per il refactoring e l'ottimizzazione
del codice sorgente, individuando le soluzioni presentate di seguito.

\subsection{Aggiornamento delle dipendenze}
Per risolvere il problema delle dipendenze obsolete, è stata effettuata un'analisi delle versioni correnti delle dipendenze utilizzate all'interno della libreria,
verificando che le funzionalità utilizzate non fossero deprecate o rimosse, procedendo con l'eventuale refactoring ad un codice compatibile che tenesse inoltre conto
delle nuove funzionalità introdotte nelle versioni più recenti. \newline
Grazie ad uno studio accurato delle librerie utilizzate nei prodotti \textit{Datasoil S.r.l.}, è stato possibile sostituire con librerie equivalenti dipendenze preesistenti,
riducendo così l'onere di utilizzo dell'SDK all'interno dei prodotti aziendali.

\subsection{Introduzione di interfacce e tipi}
Al fine di migliorare la manutenibilità e la comprensione del codice sorgente, è stato svolto un lavoro di introduzione di interfacce e tipi per le variabili e per i parametri
delle funzioni, in modo da rendere il codice più leggibile e permettere controlli statici sul codice sorgente. \newline
Questo lavoro ha permesso di ridurre la presenza di tipi \textit{any} all'interno del codice sorgente, garantendo una maggiore sicurezza e affidabilità del prodotto finale.

\subsection{Refactoring del codice}
Per migliorare la comprensione e la manutenibilità del codice sorgente, è stato intrapreso un lavoro di refactoring su alcune porzioni della libreria preesistente.
Questo intervento ha comportato la riscrittura di diverse funzioni e la rimozione di componenti minori utilizzate per eseguire funzionalità semplici, attraverso un codice
più leggibile e immediato, eliminando complessità superflue e mantenendo l'efficacia nell'implementazione di operazioni elementari.

\subsection{Rimozione di componenti inutilizzate}
In merito alla presenza di componenti inutilizzate all'interno della libreria preesistente, è stata effettuata un'accurata analisi delle componenti presenti, selezionando
quelle che non trovavano un diretto utilizzo all'interno dei prodotti \textit{Datasoil S.r.l.} e procedendo con la loro rimozione. \newline
Questo lavoro ha permesso di ridurre le dimensioni del codice sorgente e di migliorare le prestazioni complessive della libreria, diminuendo inoltre il numero di dipendenze
esterne richiesto per il corretto funzionamento dell'SDK prodotto.

\subsection{Implementazione di nuove funzionalità}
Per colmare le lacune riscontrate nella libreria preesistente in merito all'assenza di alcune funzionalità richieste dai clienti di \textit{Datasoil S.r.l.} all'interno dei prodotti forniti,
è stato condotto un lavoro di implementazione di nuove funzionalità. \newline
In questo processo, si è cercato di utilizzare dipendenze esterne preferibilmente già in dotazione o che fossero compatibili con il contesto applicativo dell'azienda. Qualora ciò non
fosse stato possibile, sono state introdotte nuove dipendenze esterne selezionate a seguito di un'analisi dettagliata che tenesse conto delle implicazioni in termini di dimensioni
finali dell'SDK prodotto.

\section{Strumenti utilizzati}
La seguente sezione fornisce gli strumenti utilizzati durante lo svolgimento dello stage per la realizzazione della libreria grafica.
Gli strumenti verranno presentati in ordine alfabetico secondo la seguente struttura:
\begin{itemize}
      \item \textbf{Nome strumento}: nome dello strumento utilizzato;
      \item \textbf{Versione}: versione utilizzata nel progetto durante lo stage;
      \item \textbf{Link}: link di riferimento per ulteriori informazioni sullo strumento;
      \item \textbf{Descrizione}: breve descrizione dello strumento e delle sue funzionalità;
      \item \textbf{Vantaggi}: principali vantaggi derivanti dall'utilizzo dello strumento; \textit{(Opzionale)}
      \item \textbf{Svantaggi}: principali svantaggi derivanti dall'utilizzo dello strumento; \textit{(Opzionale)}
      \item \textbf{Alternative Esaminate}: alternative studiate e prese in considerazione durante la scelta dello strumento, con una breve considerazione
            su di esse e la motivazione per la quale non sono state selezionate. \textit{(Opzionale)}
\end{itemize}

\subsection{Tecnologie frontend}

\subsubsection{D3-color}
\begin{itemize}
      \item \textbf{Nome strumento}: D3-color
      \item \textbf{Versione}: 3.1.0
      \item \textbf{Link}: \href{https://d3js.org/d3-color}{d3js.org/d3-color}
      \item \textbf{Descrizione}: D3-color è una libreria JavaScript open-source utilizzata per la manipolazione dei colori.
      \item \textbf{Vantaggi}:
            \begin{itemize}
                  \item \textit{D3-color} offre un'ampia serie di funzionalità per la manipolazione dei colori, quali la conversione tra spazi di colori, la manipolazione dei colori
                        e la generazione di scale di colori;
                  \item \textit{D3-color} è supportato da una vasta e attiva \textit{community}.
            \end{itemize}
      \item \textit{Alternative Esaminate}:
            \begin{itemize}
                  \item \textit{Chroma-js}: \textit{Chroma-js} è una libreria \textit{JavaScript} \textit{open-source} utilizzata per la manipolazione dei colori, ma risulta essere più pesante di \textit{D3-color}.
            \end{itemize}
\end{itemize}

\subsubsection{D3-scale}
\begin{itemize}
      \item \textbf{Nome strumento}: D3-scale
      \item \textbf{Versione}: 4.0.2
      \item \textbf{Link}: \href{https://d3js.org/d3-scale}{d3js.org/d3-scale}
      \item \textbf{Descrizione}: \textit{D3-scale} è una libreria \textit{JavaScript} open-source utilizzata per la generazione di scale di colori, di posizioni e di dimensioni.
      \item \textbf{Vantaggi}:
            \begin{itemize}
                  \item \textit{D3-scale} offre un'ampia serie di funzionalità per la generazione di scale di colori, di posizioni e di dimensioni, quali la generazione di scale lineari,
                        logaritmiche e ordinali;
                  \item \textit{D3-scale} è supportato da una vasta e attiva \textit{community}.
            \end{itemize}
\end{itemize}

\subsubsection{Day.js}
\begin{itemize}
      \item \textbf{Nome strumento}: Dayjs
      \item \textbf{Versione}: 1.11.7
      \item \textbf{Link}: \href{https://day.js.org/}{day.js.org}
      \item \textbf{Descrizione}: \textit{Day.js} è una libreria \textit{JavaScript} open-source utilizzata per la manipolazione delle date e degli orari.
      \item \textbf{Vantaggi}:
            \begin{itemize}
                  \item \textit{Day.js} è una libreria molto leggera, con un'ampia serie di funzionalità per la manipolazione delle date e degli orari;
                  \item \textit{Day.js} viene fornito con dichiarazioni ufficiali di tipo per \textit{TypeScript};
                  \item \textit{Day.js} è tutt'ora supportato da una vasta e attiva \textit{community}.
            \end{itemize}
      \item \textbf{Alternative Esaminate}:
            \begin{itemize}
                  \item \textit{Moment.js}: \textit{Moment.js} è una libreria \textit{JavaScript} open-source utilizzata per la manipolazione delle date e degli orari, ma risulta essere più pesante
                        di \textit{Day.js}. \textit{Moment.js} è inoltre considerato \textit{deprecated} a favore di Day.js, che offre una maggiore leggerezza e una maggiore efficienza.
            \end{itemize}
\end{itemize}

\subsubsection{Lodash}
\begin{itemize}
      \item \textbf{Nome strumento}: Lodash
      \item \textbf{Versione}: 4.14.0
      \item \textbf{Link}: \href{https://lodash.com/}{lodash.com}
      \item \textbf{Descrizione}: \textit{Lodash} è una libreria \textit{JavaScript} \textit{open-source} utilizzata per la manipolazione di oggetti e array.
      \item \textbf{Vantaggi}:
            \begin{itemize}
                  \item \textit{Lodash} offre un'ampia serie di funzionalità implementate con efficienza per la manipolazione di oggetti e array,
                        quali la ricerca, la modifica e la rimozione di elementi;
                  \item \textit{Lodash} è supportato da una vasta e attiva \textit{community}.
            \end{itemize}
      \item \textbf{Svantaggi}
            \begin{itemize}
                  \item \textit{Lodash} è una libreria molto pesante, con un'ampia serie di funzionalità che possono non essere utilizzate all'interno del progetto:
                        per questo motivo in questo progetto sono eseguiti degli \texttt{import} sui moduli specifici, riducendo così le dimensioni del \textit{\gls{package}\glox} finale.
            \end{itemize}
\end{itemize}

\subsubsection{Numbro}
\begin{itemize}
      \item \textbf{Nome strumento}: Numbro
      \item \textbf{Versione}: 2.5.0
      \item \textbf{Link}: \href{http://numbrojs.com/}{numbrojs.com}
      \item \textbf{Descrizione}: \textit{Numbro} è una libreria \textit{JavaScript} open-source utilizzata per la formattazione dei numeri.
      \item \textbf{Vantaggi}:
            \begin{itemize}
                  \item \textit{Numbro} offre un'ampia serie di funzionalità per la formattazione dei numeri, quali la formattazione delle cifre decimali, la formattazione delle valute
                        e la formattazione dei numeri in notazione scientifica;
                  \item \textit{Numbro} è supportato da una vasta e attiva \textit{community}.
            \end{itemize}
      \item \textbf{Alternative Esaminate}:
            \begin{itemize}
                  \item \textit{Numeral.js}: libreria \textit{JavaScript} open-source utilizzata per la formattazione dei numeri, ma risulta non essere più mantenuta attivamente dalla \textit{community}, oltre
                        che ad essere più pesante di \textit{Numbro}.
            \end{itemize}
\end{itemize}

\subsubsection{Plotly.js}
\begin{itemize}
      \item \textbf{Nome strumento}: Plotly.js
      \item \textbf{Versione}: custom-bundle: 2.33.0
      \item \textbf{Link}: \href{https://plotly.com/javascript/}{plotly.com/javascript}
      \item \textbf{Descrizione}: \textit{Plotly.js} è una libreria \textit{JavaScript open-source}, con supporto per \textit{TypeScript}, utilizzata per la visualizzazione di dati mediante grafici interattivi.
            Costruito sopra \textit{D3.js}, \textit{Plotly.js} offre un vasto panorama di grafici dinamici in formato \textit{\gls{SVG}\glox}, altamente personalizzabili.
      \item \textbf{Vantaggi}:
            \begin{itemize}
                  \item \textit{Plotly.js} offre funzionalità di interattività avanzate, quali zoom, pan, selezione e salvataggio dei grafici;
                  \item Le disponibilità di grafici offerte da \textit{Plotly.js} sono molto ampie, permettendo di soddisfare la maggior parte delle esigenze
                        all'interno della libreria grafica;
                  \item \textit{Plotly.js} è supportato da una vasta e attiva \textit{community}, con ampie serie di esempi disponibili su \textit{CodePen} (\href{https://codepen.io/}{codepen.io}, una piattaforma di condivisione di codice
                        \textit{online} che permette di visualizzare e modificare codice \textit{HTML, CSS e JavaScript});
                  \item \textit{Plotly.js} offre la possibilità di generare \textit{custom \gls{bundle}\glox} personalizzati, permettendo di registrare le sole \textit{traces} che si vogliono utilizzare, riducendo
                        così le dimensioni del pacchetto finale;
                  \item \textit{Plotly.js} era già precedentemente utilizzato all'interno della libreria preesistente, non necessitando così di modifiche lato \textit{\gls{backend}\glox} negli editor
                        utilizzati per la generazione delle risposte \textit{JSON} da parte dei server \textit{Datasoil S.r.l.} per la generazione delle dashboard;
                  \item \textit{Plotly.js} offre una funzionalità, \textit{Plotly.react}, che permette di aggiornare i grafici in modo efficiente, riducendo il tempo di rendering
                        e migliorando le prestazioni complessive della libreria.
            \end{itemize}
      \item \textbf{Svantaggi}:
            \begin{itemize}
                  \item \textit{Plotly.js} è una libreria molto pesante in termini sia di spazio che di rallentamenti \textit{\gls{runtime}\glox}, dovuto anche dal fatto che è implementata sopra un \textit{wrapper} proprietario di \textit{D3.js},
                        impedendo l'ottimizzazione di alcune dipendenze non necessarie in quanto non utilizzate;
                  \item La documentazione ufficiale di \textit{Plotly.js} è vaga e poco esaustiva;
                  \item La versione ufficiale di \textit{Plotly.js} presenta degli errori durante la registrazione delle \textit{traces}, impedendo di importare correttamente
                        \textit{Plotly} all'interno dei moduli \textit{TypeScript}, motivo per il quale è stata utilizzata una versione \textit{custom bundle} di \textit{Plotly.js}.
            \end{itemize}
      \item \textbf{Alternative Esaminate}:
            \begin{itemize}
                  \item \textit{D3.js}: \textit{D3.js} è una libreria \textit{JavaScript open-source} utilizzata per la generazione di grafici dinamici e manipolazione dati, la quale dalla sua parte risulta però
                        essere più complessa rispetto a \textit{Plotly.js}, richiedendo una maggiore curva di apprendimento e una maggiore quantità di codice per la generazione di grafici.
                  \item \textit{Chart.js}: \textit{Chart.js} è una libreria \textit{JavaScript open-source} utilizzata per la generazione di grafici dinamici, la quale risulta essere più leggera di \textit{Plotly.js},
                        ma offre una minore quantità di grafici disponibili; il suo utilizzo avrebbe inoltre comportato la richiesta di modifiche lato \textit{backend}.
            \end{itemize}
\end{itemize}

\subsubsection{PrimeFlex}
\begin{itemize}
      \item \textbf{Nome strumento}: PrimeFlex
      \item \textbf{Versione}: 3.3.0
      \item \textbf{Link}: \href{https://primereact.org/}{primefaces.org/primeflex}
      \item \textbf{Descrizione}: \textit{PrimeFlex} è una libreria \textit{CSS open-source} utilizzata per la creazione di \textit{layout} flessibili e \textit{\gls{responsive}\glox}.
      \item \textbf{Vantaggi}:
            \begin{itemize}
                  \item \textit{PrimeFlex} offre un'ampia serie di classi \textit{CSS} per la creazione di \textit{layout} flessibili e responsivi, permettendo di adattare il \textit{layout}
                        in base alla grandezza dello schermo;
                  \item \textit{PrimeFlex} è una libreria molto leggera;
                  \item \textit{PrimeFlex} è fortemente integrata con le componenti di \textit{PrimeReact};
                  \item \textit{PrimeFlex} può essere integrato a \textit{tailwind};
                  \item \textit{PrimeFlex} è supportato da una vasta e attiva \textit{community}.
            \end{itemize}
\end{itemize}

\subsubsection{PrimeIcons}
\begin{itemize}
      \item \textbf{Nome strumento}: PrimeIcons
      \item \textbf{Versione}: 6.0.1
      \item \textbf{Link}: \href{https://primefaces.org/primeicons}{primefaces.org/primeicons}
      \item \textbf{Descrizione}: \textit{PrimeIcons} è una libreria di icone \textit{open-source} utilizzata per la creazione di interfacce utente.
      \item \textbf{Vantaggi}:
            \begin{itemize}
                  \item \textit{PrimeIcons} offre un'ampia serie di icone per la creazione di interfacce utente, le quali sono altamente personalizzabili e facilmente integrabili;
                  \item \textit{PrimeIcons} è una libreria molto leggera, permettendo di creare interfacce utente performanti e veloci;
                  \item \textit{PrimeIcons} è fortemente integrata con le componenti di \textit{PrimeReact};
                  \item \textit{PrimeIcons} è supportato da una vasta e attiva community.
            \end{itemize}
\end{itemize}

\subsubsection{PrimeReact}
\begin{itemize}
      \item \textbf{Nome strumento}: PrimeReact
      \item \textbf{Versione}: 10.0.0
      \item \textbf{Link}: \href{https://primefaces.org/primereact}{primefaces.org/primereact}
      \item \textbf{Descrizione}: \textit{PrimeReact} è una libreria di componenti \textit{React open-source} utilizzata per la creazione di interfacce utente.
      \item \textbf{Vantaggi}:
            \begin{itemize}
                  \item \textit{PrimeReact} offre un'ampia serie di componenti \textit{React} per la creazione di interfacce utente, le quali a loro volta sono altamente personalizzabili, con numerose \textit{feature} integrate;
                  \item \textit{PrimeReact} è una libreria molto leggera, permettendo di creare interfacce utente performanti e veloci;
                  \item \textit{PrimeReact} costituisce una \textit{\gls{peerdependency}\glox} all'interno dei prodotti \textit{Datasoil S.r.l.}, permettendo di utilizzare le componenti \textit{PrimeReact} all'interno
                        della libreria grafica senza comportare l'aggiunta di dipendenze esterne, riducendo così le dimensioni del \textit{bundle} finale;
                  \item \textit{PrimeReact} è supportato da una vasta e attiva \textit{community}.
            \end{itemize}
      \item \textbf{Alternative Esaminate}:
            \begin{itemize}
                  \item \textit{Ant Design}: \textit{Ant Design} è una libreria di componenti \textit{React open-source} utilizzata per la creazione di interfacce utente, la quale non risulta però
                        essere integrata all'interno dei prodotti \textit{Datasoil S.r.l.}
            \end{itemize}
\end{itemize}


\subsubsection{React}
\begin{itemize}
      \item \textbf{Nome strumento}: React
      \item \textbf{Versione}: 18.0.0
      \item \textbf{Link}: \href{https://reactjs.org/}{reactjs.org}
      \item \textbf{Descrizione}: \textit{React} è una libreria \textit{React open-source} per la creazione di interfacce utente, sviluppata da Facebook.
      \item \textbf{Vantaggi}:
            \begin{itemize}
                  \item \textit{React} offre un'ampia serie di funzionalità per la creazione di interfacce utente, quali il \textit{Virtual DOM}, il \textit{JSX} e \textit{Hooks};
                  \item \textit{React} è supportato da una vasta e attiva \textit{community};
                  \item \textit{React} è una libreria molto leggera, permettendo di creare interfacce utente performanti e veloci.
            \end{itemize}
\end{itemize}

\subsubsection{TypeScript}
\begin{itemize}
      \item \textbf{Nome strumento}: TypeScript
      \item \textbf{Versione}: 5.1.6
      \item \textbf{Link}: \href{https://www.typescriptlang.org/}{typescriptlang.org}
      \item \textbf{Descrizione}: \textit{TypeScript} è un linguaggio di programmazione open-source sviluppato da \textit{Microsoft}, che fa da \textit{\gls{super-set}\glox} a \textit{JavaScript}.
      \item \textbf{Vantaggi}:
            \begin{itemize}
                  \item \textit{TypeScript} permette di definire tipi ed interfacce per le variabili e i parametri delle funzioni, garantendo una maggiore sicurezza e affidabilità del codice;
                  \item \textit{TypeScript} permette di effettuare controlli statici sul codice sorgente, riducendo il numero di errori a \textit{runtime}.
            \end{itemize}
\end{itemize}

\subsubsection{usehook-ts}
\begin{itemize}
      \item \textbf{Nome strumento}: usehook-ts
      \item \textbf{Versione}: 3.1.0
      \item \textbf{Link}: \href{https://usehooks-ts.com/}{usehooks-ts.com}
      \item \textbf{Descrizione}: \textit{usehook-ts} è una libreria \textit{open-source} di \textit{custom hook React}.
      \item \textbf{Vantaggi}:
            \begin{itemize}
                  \item \textit{usehook-ts} offre un'ampia serie di \textit{hook} personalizzati per la creazione di interfacce utente;
                  \item \textit{usehook-ts} è supportato da una vasta e attiva \textit{community}.
            \end{itemize}
\end{itemize}

\subsection{Strumenti per lo sviluppo}

\subsubsection{ESLint}
\begin{itemize}
      \item \textbf{Nome strumento}: ESLint
      \item \textbf{Versione}: 8.3.0
      \item \textbf{Link}: \href{https://eslint.org/}{eslint.org}
      \item \textbf{Descrizione}: \textit{ESLint} è uno strumento \textit{open-source} utilizzato per l'analisi statica del codice sorgente, per identificare pattern problematici o codice che non rispetta
            le linee guida definite all'interno del progetto.
      \item \textbf{Vantaggi}:
            \begin{itemize}
                  \item \textit{ESLint} permette di definire regole personalizzate per l'analisi del codice sorgente, garantendo la coerenza e la qualità del codice;
                  \item \textit{ESLint} permette di integrarsi con gli strumenti di \textit{build}, permettendo di eseguire l'analisi statica del codice sorgente durante il processo di \textit{build}.
            \end{itemize}
\end{itemize}

\subsubsection{Jest}
\begin{itemize}
      \item \textbf{Nome strumento}: Jest
      \item \textbf{Versione}: 29.0.7
      \item \textbf{Link}: \href{https://jestjs.io/}{jestjs.io}
      \item \textbf{Descrizione}: \textit{Jest} è un \textit{\gls{framework}\glox} di testing \textit{open-source} utilizzato per il testing di codice \textit{JavaScript} e \textit{TypeScript}.
      \item \textbf{Vantaggi}:
            \begin{itemize}
                  \item \textit{Jest} permette di effettuare test unitari, test di integrazione e test \textit{end-to-end}, garantendo la qualità e la stabilità del codice;
                  \item \textit{Jest} permette di effettuare test in parallelo, riducendo i tempi di esecuzione dei test;
                  \item \textit{Jest} permette di generare report dettagliati sui test effettuati, permettendo di identificare e correggere eventuali errori.
            \end{itemize}
\end{itemize}


\subsubsection{NVM}
\begin{itemize}
      \item \textbf{Nome strumento}: NVM
      \item \textbf{Versione}: 1.1.12
      \item \textbf{Link}: \href{https://github.com/nvm-sh/nvm}{github.com/nvm-sh/nvm}
      \item \textbf{Descrizione}: \textit{NVM (Node Version Manager)} è uno strumento \textit{open-source} utilizzato per la gestione delle versioni di \textit{Node.js}.
      \item \textbf{Vantaggi}:
            \begin{itemize}
                  \item \textit{NVM} permette di installare e gestire più versioni di \textit{Node.js} all'interno del sistema, permettendo di selezionare la versione corretta
                        per il progetto in corso;
                  \item \textit{NVM} permette di gestire le versioni di \textit{Node.js} in modo semplice e veloce, permettendo di passare da una versione all'altra con un solo comando.
            \end{itemize}
\end{itemize}

\subsubsection{Prettier}
\begin{itemize}
      \item \textbf{Nome strumento}: Prettier
      \item \textbf{Versione}: 3.2.5
      \item \textbf{Link}: \href{https://prettier.io/}{prettier.io}
      \item \textbf{Descrizione}: \textit{Prettier} è uno strumento \textit{open-source} utilizzato per la formattazione del codice sorgente.
      \item \textbf{Vantaggi}:
            \begin{itemize}
                  \item \textit{Prettier} permette di formattare il codice sorgente in modo automatico, garantendo uno stile uniforme all'interno del progetto.
            \end{itemize}
\end{itemize}

\subsubsection{Rollup}
\begin{itemize}
      \item \textbf{Nome strumento}: Rollup
      \item \textbf{Versione}: 4.18.0
      \item \textbf{Link}: \href{https://rollupjs.org/}{rollupjs.org}
      \item \textbf{Descrizione}: \textit{Rollup} è un \textit{\gls{bundler}\glox} di moduli \textit{JavaScript} che permette di risolvere le dipendenze tra i moduli, generando un unico file di output.
      \item \textbf{Vantaggi}:
            \begin{itemize}
                  \item \textit{Rollup} effettua il \textit{\gls{tree-shaking}\glox}, rimuovendo le dipendenze non utilizzate all'interno del codice sorgente;
                  \item Permette di generare bundle in diversi formati, quali \textit{CommonJS} e \textit{ESM};
                  \item \textit{Rollup} è noto per la sua velocità ed efficienza nelle \textit{build}, specialmente per progetti più piccoli o librerie.
                        Questo può risultare in tempi di \textit{build} più rapidi e in una migliore esperienza di sviluppo;
                  \item \textit{Rollup} ha un sistema di \textit{plug-in} molto potente e flessibile che permette di estendere le sue funzionalità;
                  \item \textit{Rollup} permette di utilizzare \textit{TypeScript} all'interno del progetto, integrando la configurazione dichiarata nel file \textit{tsconfig.json};
                  \item \textit{Rollup} permette di generare file di dichiarazione \textit{TypeScript}.
            \end{itemize}
\end{itemize}

\subsubsection{Yarn}
\begin{itemize}
      \item \textbf{Nome strumento}: Yarn
      \item \textbf{Versione}: 1.22.22
      \item \textbf{Link}: \href{https://yarnpkg.com/}{yarnpkg.com}
      \item \textbf{Descrizione}: \textit{Yarn} è un \textit{package manager} per \textit{JavaScript}, sviluppato da \textit{Facebook, Google} e \textit{Tilde}.
      \item \textbf{Vantaggi}:
            \begin{itemize}
                  \item \textit{Yarn} è più veloce di \textit{npm}, ha un sistema di cache più efficiente e permette di installare pacchetti in parallelo.
            \end{itemize}
      \item \textbf{Svantaggi}:
            \begin{itemize}
                  \item \textit{Yarn} presenta un \textit{registry} di dimensione minore rispetto a quello di \textit{npm}.
            \end{itemize}
      \item \textbf{Alternative Esaminate}:
            \begin{itemize}
                  \item \textit{npm}: npm è il \textit{package manager} di default per \textit{Node.js}, ma è più lento di \textit{Yarn} e ha un sistema di cache meno efficiente,
                        offrendo inoltre un output meno comprensibile.
            \end{itemize}
\end{itemize}

\subsection{Strumenti per la Collaborazione e la Gestione del Progetto}
Nella seguente sezione vengono presentati gli strumenti utilizzati per la collaborazione e la gestione del progetto durante lo svolgimento dello stage.

\subsubsection{Confluence}
\begin{itemize}
      \item \textbf{Nome strumento}: Confluence
      \item \textbf{Versione}: Cloud
      \item \textbf{Link}: \href{https://www.atlassian.com/software/confluence}{atlassian.com/software/confluence}
      \item \textbf{Descrizione}: \textit{Confluence} è un software di collaborazione sviluppato da \textit{Atlassian}, utilizzato per la creazione e la gestione della documentazione
            all'interno di un'azienda.
\end{itemize}

\subsubsection{Slack}
\begin{itemize}
      \item \textbf{Nome strumento}: Slack
      \item \textbf{Versione}: Cloud
      \item \textbf{Link}: \href{https://slack.com/}{slack.com}
      \item \textbf{Descrizione}: \textit{Slack} è un software di collaborazione sviluppato da \textit{Slack Technologies}, utilizzato per la comunicazione quotidiana e la collaborazione
            con il team.
\end{itemize}

\section{Descrizione del prodotto ottenuto}
Il prodotto conseguito al termine dello stage consiste in una libreria \textit{TypeScript} di componenti utilizzati nel processo di creazione di dashboard dinamiche,
ottenuto a partire dal refactor e l'ottimizzazione della libreria \textit{open-source} utilizzata all'interno della piattaforma \textit{Redash}.
La libreria implementata, sotto il nome di \textit{viz-lib}, è stata infatti integrata assieme alla già esistente libreria \textit{dashboard} all'interno del SDK
\textit{dsdashboard2}, utilizzato all'interno dei prodotti \textit{Datasoil S.r.l.} per la generazione di dashboard dinamiche.
La libreria \textit{viz-lib} offre una vasta gamma di grafici, quali:
\begin{itemize}
      \item Contatori;
      \item Grafici ad area;
      \item Grafici a barre;
      \item Grafici a bolle;
      \item Grafici a dispersione;
      \item Grafici a istogrammi;
      \item Grafici a linee;
      \item Grafici a torta;
      \item Tabelle.
\end{itemize}
Grazie al refactor e all'ottimizzazione della libreria preesistente, l'SDK prodotto risulta essere più performante, più manutenibile e soprattutto più leggero rispetto
alla versione preesistente, con un \textit{bundle} di dimensioni 1.26 MB → 418.03 kB (gzip) rispetto ai 4.11 MB → 1.203 MB (gzip) iniziali.

\section{Organizzazione del testo}
Nel presente capitolo viene presentata l'introduzione della tesi, fornendo una panoramica sull'azienda, sul contesto applicativo,
sul progetto, sugli strumenti utilizzati e sul prodotto portato a termine durante lo svolgimento dello stage. \newline
In seguito il documento presenterà la seguente organizzazione:

\begin{description}
      \item[{\hyperref[chap:analisi-requisiti]{Analisi dei requisiti}}] descrive la fase di analisi dei requisiti che è stata
            svolta dall'azienda in fase antecedente all'inizio dell'attività di stage, in modo da permettere una comprensione più profonda
            di quelle che sono le necessità da soddisfare e gli obiettivi da raggiungere all'interno di questo progetto;

      \item[{\hyperref[chap:progettazione]{Progettazione}}] illustra l'attività di progettazione che è stata svolta in vista dell'implementazione
            dei grafici prodotti, definendo e individuando le soluzioni implementative che sono state attuate durante la successiva attività di codifica;

      \item[{\hyperref[chap:realizzazione-testing]{Realizzazione e Testing}}] approfondisce l'attività di codifica delle componenti grafiche presentate
            all'interno della libreria implementata e dei relativi test effettuati per garantire la qualità e la funzionalità del prodotto finale;

      \item[{\hyperref[chap:rilascio]{Rilascio}}] descrive l'attività di rilascio dell'SDK realizzato, presentando le modalità di integrazione
            all'interno di un prodotto \textit{Datasoil S.r.l.};

      \item[{\hyperref[chap:conclusioni]{Conclusioni}}] presentano un epilogo del progetto svolto, includendo un consuntivo finale delle attività svolte,
            una valutazione del progetto e una sezione dedicata ai possibili sviluppi futuri del prodotto.
            La sezione infine si conclude con una riflessione finale sull'esperienza di stage svolta.
\end{description}
In merito alla stesura del testo, all'interno del presente documento sono state adottate le seguenti convenzioni tipografiche:
\begin{itemize}
      \item gli acronimi, le abbreviazioni e i termini ambigui o di uso non comune menzionati vengono definiti nel glossario;
      \item per la prima occorrenza dei termini riportati nel glossario viene utilizzata la seguente nomenclatura: \textit{parola}\glox\gloxspacing;
      \item i termini in lingua straniera o facenti parti del gergo tecnico sono evidenziati con il carattere \textit{corsivo}.
\end{itemize}
