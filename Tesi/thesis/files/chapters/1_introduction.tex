\chapter{Introduzione}
\label{chap:introduzione}

% * INTRODUZIONE

% - Breve descrizione del progetto
% - Principali problematiche
% - Soluzione scelta
% - Strumenti utilizzati
% - Descrizione del prodotto ottenuto
% - Struttura del resto della relazione

\section{L'azienda}
Lo stage è stato svolto presso l'azienda Datasoil S.r.l. situata nei dintorni della stazione ferroviaria di Padova.
Fondata nel XXXX, Datasoil S.r.l. è un'azienda di prodotto che si dedica allo sviluppo di piattaforme per l'industria 4.0, 
smart building e smart city. 
L'obiettivo dell'azienda è integrare informazioni ed eventi provenienti dai vari livelli aziendali per creare insight proattivi 
in tempo reale, garantendo che le informazioni corrette raggiungano le persone giuste al momento più opportuno.
Il punto di partenza è la spazialità aziendale e tutti gli asset che risiedono all'interno di essa, da cui 
provengono tutti i dati che vengono analizzati trasversalmente grazie a sempre più persone e dispositivi connessi.
Da qui il nome dell'azienda, Datasoil: fare le informazioni da questo suolo fertile di dati da cui siamo circondati.

\begin{figure}[H]
    \centering
    \includegraphics[alt={Logo Datasoil S.r.l.}, width=0.25\columnwidth]{img/datasoil_logo.jpg}
    \caption{Logo Datasoil S.r.l.}
    \label{fig:datasoil}
\end{figure}

\section{L'idea}
\subsection{Il contesto applicativo}
L'azienda Datasoil S.r.l., essendo un'azienda di prodotto, nasce con l'idea di proporre servizi di monitoraggio e controllo
di impianti industriali, smart building e smart city, offrendo come principale prodotto la piattaforma SYN.

\begin{figure}[H]
    \centering
    \includegraphics[alt={Logo SYN}, width=0.25\columnwidth]{img/syn_logo.jpg}
    \caption{Logo SYN}
    \label{fig:syn}
\end{figure}

SYN è una piattaforma di monitoraggio e controllo di impianti industriali che raccoglie dati da sensori e dispositivi
distribuiti all'interno di un impianto o su più stabilimenti, permettendo di visualizzare in tempo reale lo stato di
funzionamento dei vari asset, di analizzare i dati raccolti e di attuare azioni di controllo o segnalazione tramite 
ticketing. All'interno della piattaforma, a seconda dei piani attivati dal cliente, è possibile visualizzare dashboard dinamiche
in merito ad informazioni filtrate e aggiornate live, permettendo di raggiungere una panoramica completa dello stato dei vari 
asset monitorati in modo rapido, intuitivo ed efficace, grazie all'utilizzo di molteplici grafici e widget.

\subsection{Il progetto}
Il progetto svolto durante lo stage consiste nello sviluppo di una libreria TypeScript di componenti per la creazione di dashboard dinamiche. 
Questo è stato realizzato tramite il refactoring e l'ottimizzazione di un tool grafico preesistente, integrato nei vari prodotti di Datasoil S.r.l., 
tra cui \textit{SYN}. La libreria è stata implementata a partire da una libreria open-source utilizzata in \textit{Redash}, 
una piattaforma per la creazione di dashboard dinamiche tramite interrogazioni sulle fonti di dati configurate all'interno del servizio. \newline
L'esigenza di tale progetto nasce dalla necessità da parte di Datasoil S.r.l. di avere una libreria grafica aggiornata alle versioni 
correnti delle dipendenze utilizzate, raggiungendo una maggior performance e una maggiore manutenibilità del codice rispetto alla versione
preesistente, riducendo dove possibile le dipendenze esterne e introducendo migliorie grafiche e funzionali.

\section{Principali problematiche}
Durante l'analisi iniziale della libreria preesistente in uso all'interno dei prodotti di Datasoil S.r.l., sono emerse a fronte di una 
revisione del codice sorgente alcune problematiche legate alla manutenibilità e alla performance del tool grafico.

\subsection{Dipendenze obsolete}
La libreria preesistente utilizzava versioni obsolete delle dipendenze esterne, con conseguente rallentamento delle prestazioni in termini
di utilizzo di spazio e di tempo di caricamente delle risorse, data l'assenza di ottimizzazioni e di aggiornamenti del codice sorgente. \newline
Essendo inoltre la libreria preesistente basata su una versione della piattaforma \textit{Redash}, i moduli utilizzati il più delle volte
rappresentavono alternative a quelli già utilizzati all'interno dei prodotti Datasoil S.r.l., creando una dipendenze non necessarie che 
causavano un aumento delle dimensioni del SDK utilizzato per la creazione delle dashboard.

\subsection{Manutenibilità del codice}
Il codice sorgente della libreria preesistente, essendo stato sviluppato a fronte di una esigenza specifica dell'azienda Datasoil S.r.l.,
risultava essere poco manutenibile e assente di documentazione. \newline
Inoltre, la libreria utilizzata da \textit{Redash}, inizialmente era stata sviluppata in JavaScript, subendo solo successivamente un 
refactoring in TypeScript: questo refactoring non è però avvenuto con l'introduzione di interfacce e tipi, bensì con la semplice aggiunta 
di tipi \textit{any} per le variabili e i parametri delle funzioni, rendendo il codice poco leggibile e difficile da mantenere.

\subsection{Componenti inutilizzate}
In quanto la libreria preesistente fosse realizzata a partire dal modulo utilizzato dalla piattaforma \textit{Redash}, non tutti i componenti presenti 
in essa trovavano utilizzo all'interno dei prodotti realizzati da Datasoil S.r.l.: questo in gran parte era dovuto da una offerta di grafici inopportuni
per quello che è il contesto applicativo dell'azienda.

\subsection{Assenza di funzionalità}
La libreria preesistente non soddisfaceva pienamente le esigenze dei clienti di Datasoil S.r.l., mancando di alcune funzionalità fondamentali. 
Questa limitazione ha reso necessario un intervento di refactoring per colmare le lacune e migliorare le prestazioni complessive.

\section{Soluzione scelta}
Per risolvere le problematiche emerse durante l'analisi iniziale della libreria preesistente, si è optato per il refactoring e l'ottimizzazione
del codice sorgente, individuando le soluzioni presentate di seguito.

\subsection{Aggiornamento delle dipendenze}
Per risolvere il problema delle dipendenze obsolete, è stata effettuata un'analisi delle versioni correnti delle dipendenze utilizzate all'interno della libreria,
verificando che le funzionalità utilizzate non fossero state deprecate o rimosse, procedendo con l'eventuale refactoring ad un codice compatibile, che tenesse conto 
inoltre delle nuove funzionalità introdotte nelle versioni più recenti. \newline
Grazie ad uno studio accurato delle librerie utilizzate nei prodotti Datasoil S.r.l., è stato possibile sostituire con librerie equivalenti dipendenze preesistenti, 
riducendo così l'onere di utilizzo dell'SDK all'interno dei prodotti aziendali. 

\subsection{Introduzione di interfacce e tipi}
Al fine di migliorare la manutenibilità e la comprensione del codice sorgente, è stato svolto un lavoro di introduzione di interfacce e tipi per le variabili e i parametri
delle funzioni, in modo da rendere il codice più leggibile e permettere controlli statici sul codice sorgente. \newline
Questo lavoro ha permesso di ridurre la presenza di tipi \textit{any} all'interno del codice sorgente, garantendo una maggiore sicurezza e affidabilità del prodotto finale.

\subsection{Refactoring del codice}
Per migliorare la comprensione e la manutenibilità del codice sorgente, è stato intrapreso un lavoro di refactoring su alcune porzioni della libreria preesistente. 
Questo intervento ha comportato la riscrittura di diverse funzioni e la rimozione di componenti minori utilizzate per eseguire funzionalità semplici, attraverso un codice
più leggibile e immediato, eliminando complessità superflue e mantenendo l'efficacia nell'implementazione di operazioni elementari.

\subsection{Rimozione di componenti inutilizzate}
In merito alla presenza di componenti inutilizzate all'interno della libreria preesistente, è stata effettuata un'accurata analisi delle componenti presenti, selezionando
quelle che non trovavano un diretto utilizzo all'interno dei prodotti Datasoil S.r.l. e procedendo con la loro rimozione. \newline
Questo lavoro ha permesso di ridurre le dimensioni del codice sorgente e di migliorare le prestazioni complessive della libreria, diminuendo inoltre il numero di dipendenze
esterne richiesto per il corretto funzionamento dell'SDK prodotto. 

\subsection{Implementazione di nuove funzionalità}
Per colmare le lacune riscontrate nella libreria preesistente in merito all'assenza di alcune funzionalità richieste dai clienti di Datasoil S.r.l. all'interno dei prodotti forniti,
è stato condotto un lavoro di implementazione di nuove funzionalità. \newline
In questo processo, si è cercato preferibilmente di utilizzare dipendenze esterne già esistenti o che fossero compatibili con il contesto applicativo dell'azienda. Qualora ciò non 
fosse stato possibile, sono state introdotte nuove dipendenze esterne selezionate a seguito di analisi dettagliata che tenesse conto delle implicazioni in termini di dimensioni 
finali dell'SDK prodotto.


\section{Strumenti utilizzati}

\section{Descrizione del prodotto ottenuto}

\section{Organizzazione del testo}
Nel presente capitolo viene presentata l'introduzione della tesi, fornendo una panoramica sull'azienda, sul contesto applicativo, 
sul progetto e sul prodotto portato a termine durante lo svolgimento dello stage. \newline
In seguito il documento presenterà la seguente organizzazione:

\begin{description}
    \item[{\hyperref[chap:processi-metodologie]{Analisi dei requisiti}}] descrive la fase di analisi dei requisiti che è stata
    svolta dall'azienda antecedentemente all'inizio dell'attività di stage, in modo da permettere una comprensione più profonda
    di quelle che sono le necessità da soddisfare e gli obiettivi da raggiungere all'interno di questo progetto;
    
    \item[{\hyperref[chap:descrizione-stage]{Progettazione}}] illustra l'attività di progettazione che è stata svolta in vista dell'implementazione
    dei grafici prodotti, definendo le tecnologie e le soluzioni implementative che sono state attuate durante la successiva attività di codifica;
    
    \item[{\hyperref[chap:analisi-requisiti]{Realizzazione e Testing}}] approfondisce l'attività di codifica delle componenti grafiche presentate 
    all'interno della libreria implementata e dei relativi test effettuati per garantire la qualità e la funzionalità del prodotto finale;
    
    \item[{\hyperref[chap:conclusioni]{Conclusioni}}] presentano un epilogo del progetto svolto, includendo una valutazione oggettiva
    degli strumenti utilizzati e una riflessione sui possibili punti di insoddisfazione e dei relativi miglioramenti applicabili al prodotto realizzato.
    La sezione infine si conclude con proposte per future estensioni e sviluppi del progetto.
    
Riguardo la stesura del testo, relativamente al documento sono state adottate le seguenti convenzioni tipografiche:
\begin{itemize}
	\item gli acronimi, le abbreviazioni e i termini ambigui o di uso non comune menzionati vengono definiti nel glossario, situato alla fine del presente documento;
	\item per la prima occorrenza dei termini riportati nel glossario viene utilizzata la seguente nomenclatura: \textit{parola}\glox\gloxspacing;
	\item i termini in lingua straniera o facenti parti del gergo tecnico sono evidenziati con il carattere \textit{corsivo}.
\end{itemize}


\newpage