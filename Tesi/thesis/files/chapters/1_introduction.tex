\chapter{Introduzione}
\label{chap:introduzione}

\section{L'azienda}
Lo stage è stato svolto presso l'azienda Datasoil S.r.l. situata nei dintorni della stazione ferroviaria di Padova.
Fondata nel XXXX, Datasoil S.r.l. è un'azienda di prodotto che si dedica allo sviluppo di piattaforme per l'industria 4.0, 
smart building e smart city. 
L'obiettivo dell'azienda è integrare informazioni ed eventi provenienti dai vari livelli aziendali per creare insight proattivi 
in tempo reale, garantendo che le informazioni corrette raggiungano le persone giuste al momento più opportuno.
Il punto di partenza è la spazialità aziendale e tutti gli asset che risiedono all'interno di essa, da cui 
provengono tutti i dati che vengono analizzati trasversalmente grazie a sempre più persone e dispositivi connessi.
Da qui il nome dell'azienda, Datasoil: fare le informazioni da questo suolo fertile di dati da cui siamo circondati.

\begin{figure}[H]
    \centering
    \includegraphics[alt={Logo Datasoil S.r.l.}, width=0.25\columnwidth]{img/datasoil_logo.jpg}
    \caption{Logo Datasoil S.r.l.}
    \label{fig:data_soil}
\end{figure}

\section{L'idea}
\subsection{Il contesto applicativo}
L'azienda Datasoil S.r.l., essendo un'azienda di prodotto, nasce con l'idea di proporre servizi di monitoraggio e controllo
di impianti industriali, smart building e smart city, offrendo come principale prodotto la piattaforma SYN.

SYN è una piattaforma di monitoraggio e controllo di impianti industriali che raccoglie dati da sensori e dispositivi
distribuiti all'interno di un impianto o su più stabilimenti, permettendo di visualizzare in tempo reale lo stato di
funzionamento dei vari asset, di analizzare i dati raccolti e di attuare azioni di controllo o segnalazione tramite 
ticketing.

\subsection{Il progetto libreria dashboard dinamiche}



Lorem Figure \ref{fig:entanglement}

Esempio di utilizzo di un termine nel glossario \gls{api}.

Esempio di citazione in linea \cite{site:agile-manifesto}.

Esempio di citazione nel piè di pagina citazione\footcite{womak:lean-thinking}

\lipsum[1-2]

\section{Organizzazione del testo}
\begin{description}
    \item[{\hyperref[chap:processi-metodologie]{Il secondo capitolo}}] descrive ...
    
    \item[{\hyperref[chap:descrizione-stage]{Il terzo capitolo}}] approfondisce ...
    
    \item[{\hyperref[chap:analisi-requisiti]{Il quarto capitolo}}] approfondisce ...
    
    \item[{\hyperref[chap:progettazione-codifica]{Il quinto capitolo}}] approfondisce ...
    
    \item[{\hyperref[chap:verifica-validazione]{Il sesto capitolo}}] approfondisce ...
    
    \item[{\hyperref[chap:conclusioni]{Nel settimo capitolo}}] descrive ...
\end{description}

Riguardo la stesura del testo, relativamente al documento sono state adottate le seguenti convenzioni tipografiche:
\begin{itemize}
	\item gli acronimi, le abbreviazioni e i termini ambigui o di uso non comune menzionati vengono definiti nel glossario, situato alla fine del presente documento;
	\item per la prima occorrenza dei termini riportati nel glossario viene utilizzata la seguente nomenclatura: \textit{parola}\glox\gloxspacing;
	\item i termini in lingua straniera o facenti parti del gergo tecnico sono evidenziati con il carattere \textit{corsivo}.
\end{itemize}

\begin{listing}[H]
\begin{minted}{c}
#include <stdio.h>
int main() {
    print("Hello, world!");
    return 0;
}
\end{minted}
\caption{Example of code}
\label{listing:a}
\end{listing}

\newpage