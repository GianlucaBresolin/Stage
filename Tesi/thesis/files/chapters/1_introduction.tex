\chapter{Introduzione}
\label{chap:introduzione}

\section{L'azienda}
Lo stage è stato svolto presso l'azienda \textit{Datasoil S.r.l.} situata nei dintorni della stazione ferroviaria di Padova.
Fondata nel 2016, \textit{Datasoil S.r.l.} è un'azienda di prodotto che si dedica allo sviluppo di piattaforme per l'industria 4.0,
\textit{smart building} e \textit{smart city}.
L'obiettivo dell'azienda è integrare informazioni ed eventi provenienti dai vari livelli aziendali per creare \textit{insight} proattivi
in tempo reale, garantendo che le informazioni corrette raggiungano le persone giuste al momento più opportuno.
Il punto di partenza è la spazialità aziendale e tutti gli \textit{asset} che risiedono all'interno di essa, da cui
provengono tutti i dati raccolti, i quali vengono analizzati trasversalmente grazie a sempre più persone e dispositivi connessi.
Da qui il nome dell'azienda, \textit{Datasoil}: fare emergere le informazioni da questo suolo fertile di dati da cui siamo circondati.

\begin{figure}[H]
      \centering
      \includegraphics[alt={Logo \textit{Datasoil S.r.l.}}, width=0.25\columnwidth]{img/datasoil_logo.jpg}
      \caption{Logo \textit{Datasoil S.r.l.}}
      \label{fig:datasoil}
\end{figure}

\section{L'idea}
\subsection{Il contesto applicativo}
L'azienda \textit{Datasoil S.r.l.}, essendo un'azienda di prodotto, nasce con l'idea di proporre servizi di monitoraggio e controllo
di impianti industriali, \textit{smart building} e \textit{smart city}, offrendo come principale prodotto la piattaforma SYN.

\begin{figure}[H]
      \centering
      \includegraphics[alt={Logo SYN}, width=0.25\columnwidth]{img/syn_logo.jpg}
      \caption{Logo SYN}
      \label{fig:syn}
\end{figure}

SYN è una piattaforma di monitoraggio e controllo di impianti industriali che raccoglie dati da sensori e dispositivi
distribuiti all'interno di un impianto o su più stabilimenti, permettendo di visualizzare in tempo reale lo stato di
funzionamento dei vari \textit{asset}, di analizzare i dati raccolti e di attuare azioni di controllo o segnalazione tramite
\textit{ticketing}. All'interno della piattaforma, a seconda dei piani attivi del cliente, è possibile visualizzare dashboard dinamiche
in merito ad informazioni filtrate e aggiornate \textit{live}, permettendo di raggiungere una panoramica completa dello stato dei vari
\textit{asset} monitorati in modo rapido, intuitivo ed efficace, grazie all'utilizzo di molteplici grafici e \textit{widget}.

\subsection{Il progetto}
Il progetto svolto durante lo stage consiste nello sviluppo di una libreria \textit{TypeScript} di componenti per la creazione di dashboard dinamiche.
Questo è stato realizzato tramite il \gls{refactoring}\glox e l'ottimizzazione di un \textit{tool} grafico preesistente, integrato nei vari prodotti di \textit{Datasoil S.r.l.},
tra cui \textit{SYN}. La libreria è stata implementata a partire da un modulo \textit{\gls{open-source}}\glox utilizzato in \textit{Redash} (\href{https://redash.io}{redash.io}),
una piattaforma per la creazione di dashboard dinamiche tramite interrogazioni sulle fonti di dati configurate all'interno del servizio. \newline
L'esigenza di tale progetto nasce dalla necessità da parte di \textit{Datasoil S.r.l.} di avere una libreria grafica aggiornata alle versioni
correnti delle dipendenze utilizzate, migliorando il \gls{FCP}\glox e il \gls{TTI}\glox, raggiungendo una maggiore manutenibilità del codice rispetto alla versione
preesistente e riducendo dove possibile le dipendenze esterne e introducendo migliorie grafiche e funzionali. \newline
La libreria sviluppata sotto il nome '\textit{viz-lib}', costituisce, insieme alla già esistente libreria '\textit{dashboard}' (la quale ha subito anch'essa in parte un processo di refactoring
per permettere l'integrazione con la nuova libreria e l'introduzione di nuove funzionalità), il nuovo \gls{SDK}\glox '\textit{dsdashbord2}'
utilizzato all'interno dei prodotti \textit{Datasoil S.r.l.} per la realizzazione di dashboard dinamiche.

\section{Principali problematiche}
Durante l'analisi iniziale della libreria preesistente in uso all'interno dei prodotti di \textit{Datasoil S.r.l.}, sono emerse alcune problematiche
legate alla manutenibilità e alla performance del \textit{tool} grafico a seguito di una attenta revisione del codice sorgente. \newline
Di seguito vengono presentate le principali problematiche riscontrate.

\textbf{Dipendenze obsolete}\\
La libreria preesistente utilizzava versioni obsolete delle dipendenze esterne, con conseguente degradazione delle prestazioni in termini
di utilizzo di spazio e di tempo di caricamente delle risorse, data l'assenza di ottimizzazioni e di aggiornamenti del codice sorgente. \newline
Essendo inoltre la libreria preesistente basata su una versione del modulo di visualizzazioni utilizzato nella piattaforma \textit{Redash},
le tecnologie utilizzate il più delle volte rappresentavano alternative a quelle già utilizzate nei prodotti \textit{Datasoil S.r.l.}, creando dipendenze
non necessarie, con conseguente aumento delle dimensioni del SDK utilizzato per la generazione delle dashboard.\\
\\
\textbf{Manutenibilità del codice}\\
Il codice sorgente della libreria preesistente, sviluppato in tempistiche rapide a fronte di una esigenza specifica dell'azienda \textit{Datasoil S.r.l.},
risultava essere poco manutenibile e assente di documentazione. \newline
Inoltre, la libreria di visualizzazioni utilizzata da \textit{Redash} su cui si basa l'SDK utilizzato dall'azienda, inizialmente implementata in JavaScript,
ha subito solo successivamente un refactoring in TypeScript: questo refactoring non è però avvenuto con l'introduzione di interfacce e tipi, bensì con la semplice aggiunta
di tipi \textit{any} per le variabili e per i parametri delle funzioni, introducendo numerosi \textit{@ts-ignore} per ignorare gli errori di compilazione, rendendo il codice
poco leggibile e difficile da mantenere.\\
\\
\textbf{Componenti inutilizzate}\\
In quanto la libreria preesistente fosse realizzata a partire dal modulo utilizzato dalla piattaforma \textit{Redash}, non tutti i componenti presenti
in essa trovavano utilizzo all'interno dei prodotti realizzati da \textit{Datasoil S.r.l.}: questo in gran parte era dovuto ad un'offerta di grafici non adeguati
a quello che è il contesto applicativo dell'azienda.\\
\\
\textbf{Assenza di funzionalità}\\
La libreria preesistente non soddisfaceva pienamente le esigenze dei clienti di \textit{Datasoil S.r.l.}, mancando di alcune funzionalità fondamentali,
quali la possibilità di effettuare il \textit{download} dei dati tabellari o la visualizzazione a pagina piena dei vari \textit{widget}.
Queste limitazioni hanno reso necessario un intervento di refactoring per colmare le lacune e migliorare le prestazioni complessive.

\section{Soluzione scelte}
Per risolvere le problematiche emerse durante l'analisi iniziale della libreria preesistente, si è optato per il refactoring e l'ottimizzazione
del codice sorgente, individuando le soluzioni presentate di seguito.\\
\\
\textbf{Aggiornamento delle dipendenze}\\
Per risolvere il problema delle dipendenze obsolete, è stata effettuata un'analisi delle versioni correnti delle dipendenze utilizzate all'interno della libreria,
verificando che le funzionalità utilizzate non fossero deprecate o rimosse, procedendo con l'eventuale refactoring ad un codice compatibile che tenesse inoltre conto
delle nuove funzionalità introdotte nelle versioni più recenti. \newline
Grazie ad uno studio accurato delle librerie utilizzate nei prodotti \textit{Datasoil S.r.l.}, è stato possibile sostituire con librerie equivalenti dipendenze preesistenti,
riducendo così l'onere di utilizzo dell'SDK all'interno dei prodotti aziendali. \\
\\
\textbf{Introduzione di interfacce e tipi}\\
Al fine di migliorare la manutenibilità e la comprensione del codice sorgente, è stato svolto un lavoro di introduzione di interfacce e tipi per le variabili e per i parametri
delle funzioni, in modo da rendere il codice più leggibile e permettere controlli statici sul codice sorgente. \newline
Questo lavoro ha permesso di ridurre la presenza di tipi \textit{any} all'interno del codice sorgente, garantendo una maggiore sicurezza e affidabilità del prodotto finale. \\
\\
\textbf{Refactoring del codice}\\
Per migliorare la comprensione e la manutenibilità del codice sorgente, è stato intrapreso un lavoro di refactoring su alcune porzioni della libreria preesistente.
Questo intervento ha comportato la riscrittura di diverse funzioni e la rimozione di componenti minori utilizzate per eseguire funzionalità semplici, attraverso un codice
più leggibile e immediato, eliminando complessità superflue e mantenendo l'efficacia nell'implementazione di operazioni elementari.\\
\\
\textbf{Rimozione di componenti inutilizzate}\\
In merito alla presenza di componenti inutilizzate all'interno della libreria preesistente, è stata effettuata un'accurata analisi delle componenti presenti, selezionando
quelle che non trovavano un diretto utilizzo all'interno dei prodotti \textit{Datasoil S.r.l.} e procedendo con la loro rimozione. \newline
Questo lavoro ha permesso di ridurre le dimensioni del codice sorgente e di migliorare le prestazioni complessive della libreria, diminuendo inoltre il numero di dipendenze
esterne richiesto per il corretto funzionamento dell'SDK prodotto.\\
\\
\textbf{Implementazione di nuove funzionalità}\\
Per colmare le lacune riscontrate nella libreria preesistente in merito all'assenza di alcune funzionalità richieste dai clienti di \textit{Datasoil S.r.l.} all'interno dei prodotti forniti,
è stato condotto un lavoro di implementazione di nuove funzionalità. \newline
In questo processo, si è cercato di utilizzare dipendenze esterne preferibilmente già in dotazione o che fossero compatibili con il contesto applicativo dell'azienda. Qualora ciò non
fosse stato possibile, sono state introdotte nuove dipendenze esterne selezionate a seguito di un'analisi dettagliata che tenesse conto delle implicazioni in termini di dimensioni
finali dell'SDK prodotto.

\section{Risorse Tecnologiche}
La realizzazione della libreria grafica prodotta ha visto l'utilizzo di differenti tecnologie. \\
Per quanto riguarda l'implementazione dell'SDK in \textit{TypeScript}, è stato utilizzato il \textit{\gls{framework}\glox} \textit{React} per la creazione dei componenti grafici,
con il supporto di \textit{PrimeReact, PrimeFlex} e \textit{PrimeIcons}, utilizzando la libreria \textit{Plotly.js} per la generazione dei grafici. \\
Per la gestione delle dipendenze è stato utilizzato il \textit{package manager} \textit{npm}, mentre per la generazione e gestione del \textit{\gls{bundle}\glox} è stato impiegato \textit{Rollup}. \\
In merito al testing del prodotto, è stato integrato il \textit{framework} di testing \textit{Jest} all'interno del progetto per la scrittura ed esecuzione dei test. \\
Infine, per la documentazione del codice sorgente è stato utilizzato \textit{Confluence} di \textit{Atlassian} mentre, per quanto riguarda la comunicazione interna con il team di sviluppo,
è stato impiegato \textit{Slack}.

\section{Descrizione del prodotto ottenuto}
Il prodotto conseguito al termine dello stage consiste in una libreria \textit{TypeScript} di componenti utilizzati nel processo di creazione di dashboard dinamiche,
ottenuto a partire dal refactor e l'ottimizzazione della libreria \textit{open-source} utilizzata all'interno della piattaforma \textit{Redash}.
La libreria implementata, sotto il nome di \textit{viz-lib}, è stata infatti integrata assieme alla già esistente libreria \textit{dashboard} all'interno del SDK
\textit{dsdashboard2}, utilizzato all'interno dei prodotti \textit{Datasoil S.r.l.} per la generazione di dashboard dinamiche.
La libreria \textit{viz-lib} offre una vasta gamma di grafici, quali:
\begin{itemize}
      \item Contatori;
      \item Grafici ad area;
      \item Grafici a barre;
      \item Grafici a bolle;
      \item Grafici a dispersione;
      \item Grafici a istogrammi;
      \item Grafici a linee;
      \item Grafici a torta;
      \item Tabelle.
\end{itemize}
Grazie al refactor e all'ottimizzazione della libreria preesistente, l'SDK prodotto risulta essere più performante, più manutenibile e soprattutto più leggero rispetto
alla versione preesistente, con un \textit{bundle} di dimensioni 1.26 MB → 418.03 kB (gzip) rispetto ai 4.11 MB → 1.203 MB (gzip) iniziali.

\section{Organizzazione del testo}
Nel presente capitolo viene presentata l'introduzione della tesi, fornendo una panoramica sull'azienda, sul contesto applicativo,
sul progetto, sugli strumenti utilizzati e sul prodotto portato a termine durante lo svolgimento dello stage. \newline
In seguito il documento presenterà la seguente organizzazione:

\begin{description}
      \item Il {\hyperref[chap:analisi-requisiti]{Capitolo 2}} descrive la fase di analisi dei requisiti che è stata
            svolta dall'azienda in fase antecedente all'inizio dell'attività di stage, in modo da permettere una comprensione più profonda
            di quelle che sono le necessità da soddisfare e gli obiettivi da raggiungere all'interno di questo progetto;

      \item Il {\hyperref[chap:progettazione]{Capitolo 3}} illustra l'attività di progettazione che è stata svolta in vista dell'implementazione
            dei grafici prodotti, definendo e individuando le soluzioni implementative che sono state attuate durante la successiva attività di codifica;

      \item Il {\hyperref[chap:realizzazione-testing]{Capitolo 4}} approfondisce l'attività di codifica delle componenti grafiche presentate
            all'interno della libreria implementata e dei relativi test effettuati per garantire la qualità e la funzionalità del prodotto finale;

      \item Il {\hyperref[chap:rilascio]{Capitolo 5}} descrive l'attività di rilascio dell'SDK realizzato, presentando le modalità di integrazione
            all'interno di un prodotto \textit{Datasoil S.r.l.};

      \item Il {\hyperref[chap:conclusioni]{Capitolo 6}} presentano un epilogo del progetto svolto, includendo un consuntivo finale delle attività svolte,
            una valutazione del progetto e una sezione dedicata ai possibili sviluppi futuri del prodotto.
            La sezione infine si conclude con una riflessione finale sull'esperienza di stage svolta.
\end{description}
In merito alla stesura del testo, all'interno del presente documento sono state adottate le seguenti convenzioni tipografiche:
\begin{itemize}
      \item gli acronimi, le abbreviazioni e i termini ambigui o di uso non comune menzionati vengono definiti nel glossario;
      \item per la prima occorrenza dei termini riportati nel glossario viene utilizzata la seguente nomenclatura: \textit{parola}\glox\gloxspacing;
      \item i termini in lingua straniera o facenti parti del gergo tecnico sono evidenziati con il carattere \textit{corsivo}.
\end{itemize}
