\chapter{Conclusioni}
\label{chap:conclusioni}
Nella seguente ed ultima sezione vengono presentate le conclusioni del progetto di stage, valutando i risultati conseguiti, esponendo
le riflessioni finali e proponendo possibili sviluppi futuri.

\section{Consuntivo finale}
Nella seguente tabella vengono riportate le ore effettive svolte durante il progetto di stage, suddivise per attività e per periodo di svolgimento.

\begin{center}
    \rowcolors{1}{white}{tableGray}
    \begin{longtable}{|p{2.5cm}|p{2.5cm}|p{7.5cm}|}
        \hline
        \rowcolor{gray!30}
        \textbf{Ore}                       & \multicolumn{1}{|c|}{\textbf{Settimane}}         & \multicolumn{1}{|c|}{\textbf{Descrizione}}                                      \\
        \hline
        \endfirsthead
        \hline
        \multicolumn{3}{|c|}{{\tablename\ \thetable{} -- continua dalla pagina precedente}}                                                                                     \\
        \hline
        \rowcolor{gray!30}
        \textbf{Ore}                       & \multicolumn{1}{|c|}{\textbf{Settimane}}         & \multicolumn{1}{|c|}{\textbf{Descrizione}}                                      \\
        \endhead
        \hline
        \multicolumn{3}{|r|}{{Continua nella prossima pagina...}}                                                                                                               \\
        \hline
        \endfoot
        \hline
        \multicolumn{1}{|c|}{\textbf{304}} & \multicolumn{2}{|c|}{\textbf{Totale ore svolte}}                                                                                   \\
        \hline
        \endlastfoot
        \hline
        54                                 & 1, 2                                             & Formazione su tecnologie utilizzate, studio del progetto e del SDK preesistente \\
        \hline
        68                                 & 2, 3, 4                                          & Progettazione SDK: layer di rappresentazione - grafici                          \\
        \hline
        90                                 & 4, 5, 6                                          & Sviluppo SDK: layer di  rappresentazione - grafici                              \\
        \hline
        30                                 & 6                                                & Progettazione SDK: layer user interaction                                       \\
        \hline
        52                                 & 7, 8                                             & Refactor SDK: layer user interaction                                            \\
        \hline
        10                                 & 8                                                & Deployment dell'SDK prodotto                                                    \\
    \end{longtable}
    \captionof{table}{Tabella consuntivo finale}
    \label{tab:consuntivo_finale}
\end{center}


\section{Valutazione del progetto}
Il progetto di stage ha permesso di sviluppare un prodotto finale che rispondesse alle esigenze dell'azienda, soddisfacendo tutti requisiti individuati nel capito
(\S \href{chapter:analisi_requisiti}{Analisi dei requisiti}). \\
L'SDK \textit{dsdashboard2} è stato implementato con successo, introducendo nuove funzionalità e miglioramenti rispetto al precedente SDK \textit{dsdashboard},
raggiungendo dimensioni del \textit{bundle} minori (1.26 MB → 418.03 kB (gzip) rispetto ai 4.11 MB → 1.203 MB (gzip) iniziali) e una miglior leggibilità
e manutenibilità del codice. \\
In merito alle tecnologie utilizzate, gli strumenti e le librerie adottate si sono rivelate efficaci e funzionali per lo sviluppo del prodotto, permettendo di
implementare un prodotto di qualità. Fa eccezione la libreria \textit{Plotly.js}, il cui utilizzo è risultato vincolato dalle risposte fornite dalle \textit{API} del \textit{backend} per
motivi di retrocompatibilità, la quale ha richiesto adattamenti e modifiche sostanziali al codice per garantire un corretto ed efficace funzionamento dell'SDK. \\
Il progetto nel suo complesso costituisce un \textit{tool} grafico avanzato e robusto, pronto per essere utilizzato all'interno dei prodotti Datasoil S.r.l., sebbene sia
da sottolineare la forte dipendenze con le \textit{API} interrogate, le quali costituiscono il principale punto critico del prodotto, ostacolando uno sviluppo flessibile e indipendente.

\section{Possibili sviluppi futuri}
La progettazione dell'SDK \textit{dsdashboard2} è stata effettuata non solo con l'ottica di introdurre nuove funzionalità e miglioramenti rispetto al precedente
SDK \textit{dsdashboard}, ma anche con l'intenzione di migliorare la manutenibilità e agevolare l'estendibilità del prodotto. \\
Possibili sviluppi futuri della libreria potrebbero dunque includere l'introduzione di nuove componenti grafiche, a partire da tutti quei grafici resi disponibili
da \textit{Plotly.js} e non ancora implementati all'interno dell'SDK, come ad esempio i grafici a radar o i grafici a proiettile (\textit{bullet chart}). \\
Sebbene all'interno dell'SDK l'introduzione di nuove componenti non costituisca un problema, rimane da valutare le modifiche lato backend da introdurre all'interno
dei prodotti Datasoil S.r.l. per permettere il supporto di tali nuove widget: \textit{dsdashboard2} è infatti progettato per essere utilizzato all'interno
di un contesto applicativo in cui le dashboard vengono generate a partire da configurazioni fornite lato backend e non in un prodotto in cui le componenti
vengono montate direttamente nel codice frontend. \\
Altre possibili sviluppi futuri potrebbero consistere nell'introduzione della funzionalità di \textit{drag and drop} all'interno dell'SDK, permettendo all'utente
di spostare e ridimensionare le componenti all'interno della dashboard, in modo da poter personalizzare la schermata in base alle proprie esigenze specifiche. \\
Infine, un ulteriore feature che era stata implementata durante il periodo di stage ma che a causa delle dimensioni in termini di memoria delle librerie utilizzate
era stata in seguito rimossa, è la funzionalità di \textit{export} delle tabelle in formato \textit{PDF}: questa funzionalità potrebbe essere reintrodotta all'interno
dell'SDK con una implementazione più leggera e performante, permettendo all'utente di esportare i dati in un formato più leggibile e di facile consultazione rispetto
al limitato \textit{CSV}.

\section{Riflessioni finali}
Durante lo svolgimento dello stage ho avuto modo di apprendere nuove competenze e conoscenze, applicandomi per la prima volta in un contesto lavorativo reale. \\
Ho avuto modo di studiare e lavorare con tecnologie che non avevo mai avuto l'opportunità di approfondire, come ad esempio la libreria di componenti \textit{PrimeReact} e la
libreria grafica \textit{Plotly.js}, oltre che approfondire le mie conoscenze in ambito di sviluppo web con \textit{React} e \textit{TypeScript}, utilizzando la mia
creatività e le mie competenze per sviluppare un prodotto finale che rispondesse alle esigenze dell'azienda. \\
Ho avuto l'opportunità di lavorare all'interno di un team di sviluppo, confrontandomi con colleghi più esperti e apprendendo costantentemente da loro, oltre che collaborare
con il mio tutor aziendale, il quale mi ha guidato e supportato durante tutto il percorso di stage. \\
Durante questa esperienza, è stato quindi fondamentale il saper comunicare con i miei colleghi, affinando le mie capacità di comunicazione e collaborazione, prestando la massima
attenzione ai dettagli e consigli ricevuti, cercando di migliorare costantemente il mio \textit{way of working} e le mie competenze. \\
Il progetto mi ha permesso di acquisire una maggior consapevolezza delle mie capacità, mettendomi alla prova in un ambiente professionale e affrontando problematiche concrete
che richiedessero soluzioni immediate ma allo stesso tempo funzionali e ben strutturate: ciò che più emerge in rilievo da questa esperienza è infatti l'importanza delle soluzioni individuate,
valutando accuratamente i pro e i contro di ciascuna opzione possibile, in quanto ogni scelta effettuata avrà delle ripercussioni sul progetto e sul lavoro futuro di manutenzione da svolgere
sul prodotto sviluppato. \\
E' risultato dunque fondamentale il saper sviluppare una valutazione critica che permettesse di individuare le soluzioni migliori e non quelle più immediate, cercando di prevedere
possibili problematiche future. \\
In conclusione, valuto quindi positivamente l'esperienza di stage svoltasi presso l'azienda Datasoil S.r.l., in quanto mi ha permesso di crescere professionalmente e personalmente,
acquisendo e affinando competenze e conoscenze funzionali al mio percorso di studi e alla mia futura carriera lavorativa.