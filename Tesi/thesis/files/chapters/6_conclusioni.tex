\chapter{Conclusioni}
\label{chap:conclusioni}
Nella seguente ed ultima sezione vengono presentate le conclusioni del progetto di stage, valutando i risultati conseguiti, esponendo
le riflessioni finali e proponendo possibili sviluppi futuri.

\section{Consuntivo finale}
Nella seguente tabella vengono riportate le ore effettive svolte durante il progetto di stage, suddivise per attività e per periodo di svolgimento.

\begin{center}
    \rowcolors{1}{white}{tableGray}
    \begin{longtable}{|p{2.5cm}|p{2.5cm}|p{7.5cm}|}
        \hline
        \rowcolor{gray!30}
        \textbf{Ore}                       & \multicolumn{1}{|c|}{\textbf{Settimane}}         & \multicolumn{1}{|c|}{\textbf{Descrizione}}                                      \\
        \hline
        \endfirsthead
        \hline
        \multicolumn{3}{|c|}{{\tablename\ \thetable{} -- continua dalla pagina precedente}}                                                                                     \\
        \hline
        \rowcolor{gray!30}
        \textbf{Ore}                       & \multicolumn{1}{|c|}{\textbf{Settimane}}         & \multicolumn{1}{|c|}{\textbf{Descrizione}}                                      \\
        \endhead
        \hline
        \multicolumn{3}{|r|}{{Continua nella prossima pagina...}}                                                                                                               \\
        \hline
        \endfoot
        \hline
        \multicolumn{1}{|c|}{\textbf{304}} & \multicolumn{2}{|c|}{\textbf{Totale ore svolte}}                                                                                   \\
        \hline
        \endlastfoot
        \hline
        54                                 & 1, 2                                             & Formazione su tecnologie utilizzate, studio del progetto e del SDK preesistente \\
        \hline
        68                                 & 2, 3, 4                                          & Progettazione SDK: layer di rappresentazione - grafici                          \\
        \hline
        90                                 & 4, 5, 6                                          & Sviluppo SDK: layer di  rappresentazione - grafici                              \\
        \hline
        30                                 & 6                                                & Progettazione SDK: layer user interaction                                       \\
        \hline
        52                                 & 7, 8                                             & Refactor SDK: layer user interaction                                            \\
        \hline
        10                                 & 8                                                & Deployment dell'SDK prodotto                                                    \\
    \end{longtable}
    \captionof{table}{Tabella consuntivo finale}
    \label{tab:consuntivo_finale}
\end{center}


\section{Valutazione del progetto}

\section{Possibili sviluppi futuri}


\section{Riflessioni finali}
Durante lo svolgimento dello stage ho avuto modo di apprendere nuove competenze e conoscenze, applicandomi per la prima volta in un contesto lavorativo reale. \\
Ho avuto modo di studiare e lavorare con tecnologie che non avevo mai avuto l'opportunità di approfondire, come ad esempio la libreria di componenti \textit{PrimeReact} e la
libreria grafica \textit{Plotly.js}, oltre che approfondire le mie conoscenze in ambito di sviluppo web con \textit{React} e \textit{TypeScript}, utilizzando la mia
creatività e le mie competenze per sviluppare un prodotto finale che rispondesse alle esigenze dell'azienda. \\
Ho avuto l'opportunità di lavorare all'interno di un team di sviluppo, confrontandomi con colleghi più esperti e apprendendo costantentemente da loro, oltre che collaborare
con il mio tutor aziendale, il quale mi ha guidato e supportato durante tutto il percorso di stage. \\
Durante questa esperienza, è stato quindi fondamentale il saper comunicare con i miei colleghi, affinando le mie capacità di comunicazione e collaborazione, prestando la massima
attenzione ai dettagli e consigli ricevuti, cercando di migliorare costantemente il mio \textit{way of working} e le mie competenze. \\
Il progetto mi ha permesso di acquisire una maggior consapevolezza delle mie capacità, mettendomi alla prova in un ambiente professionale e affrontando problematiche concrete
che richiedessero soluzioni immediate ma allo stesso tempo funzinali e ben strutturate: ciò che più emerge in rilievo da questa esperienza è infatti l'importanza delle soluzioni individuate,
valutando accuratamente i pro e i contro di ciascuna opzione possibile, in quanto ogni scelta effettuata avrà delle ripercussioni sul progetto e sul lavoro futuro di manutenzione da svolgere
sul prodotto sviluppato. \\
E' risultato dunque fondamentale il saper sviluppare una valutazione critica che permettesse di individuare le soluzioni migliori e non quelle più immediate, cercando di prevedere
possibili problematiche future. \\
In conclusione, valuto quindi positivamente l'esperienza di stage svoltasi presso l'azienda Datasoil S.r.l., in quanto mi ha permesso di crescere professionalmente e personalmente,
acquisendo e affinando competenze e conoscenze funzionali al mio percorso di studi e alla mia futura carriera lavorativa.