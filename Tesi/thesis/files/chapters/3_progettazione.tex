\chapter{Progettazione}
\label{chap:progettazione}
Nel presente capitolo viene presentata l'attività di progettazione relativa al progetto di stage focalizzato sul refactoring e
sull'ottimizzazione dello SDK impiegato dall'azienda Datasoil S.r.l. per lo sviluppo di dashboard dinamiche.
Verrà analizzato l'approcio adottato per individuare le soluzioni volte al soddisfacimento dei requisiti individuati durante
la fase di analisi dei requisiti, proseguendo con la descrizione dell'architettura progettata per la realizzazione della libreria grafica
e gli strumenti utilizzati per la sua realizzazione, concludendo con la progettazione delle singole componenti individuate.

\section{Metodologia di Progettazione}
La metodologia di progettazione adottata durante lo svolgimento del progetto di stage ha seguito inizialmente un approccio
\textit{top-down}, che permettesse, attraverso uno studio generale della libreria preesistente e della sua architettura, di invididuare
il flusso di dati tra le varie componenti, fornendo una visione globale delle interazioni e delle dipendenze esistenti.
Successivamente, è stato adottato un approccio \textit{bottom-up}, analizzando le singole componenti e le loro funzionalità,
in modo da identificare le possibili ottimizzazioni e le modifiche necessarie per il miglioramento delle prestazioni, della leggibilità e
della manutenibilità del codice.\newline
L'attività di progettazione è stata condotta in collaborazione stretta con il tutor aziendale, il quale ha fornito supporto e orientamento
riguardo alle decisioni progettuali individuate.

\section{Design dell'Architettura}
L'architettura progettata per la realizzazione della libreria grafica si basa su un pattern comunemente utilizzato nello sviluppo di packages
di componenti frontend per \textit{React}, ovvero il pattern \textit{Component Library Architecture}: questa architettura si fonda sulla creazione di una libreria di
componenti riutilizzabili, modulari e indipendenti tra loro. \newline
Tale pattern permette di creare l'architettura ideale per librerie che offrono collezioni di componenti riutilizzabili, permettendo e agevolando
la condivisione e il riutilizzo di codice all'interno del progetto, garantendo una maggior manuntenibilità del codice. \newline
Il modello proposto prevede la seguente struttura, la quale si riflette nella definizione delle macro-directory all'interno del progetto:
\begin{itemize}
      \item \textbf{components}: directory contenente i componenti della libreria grafica, suddivisi in sottodirectory per categoria;
      \item \textbf{hooks}: directory contenente i custom hooks utilizzati all'interno della libreria grafica;
      \item \textbf{utils}: directory contenente le funzioni di utilità utilizzate all'interno della libreria grafica;
      \item \textbf{models}: directory contenente le definizioni dei tipi e interfacce utilizzate all'interno della libreria grafica;
      \item \textbf{index.ts}: file principale della libreria grafica, contenente l'esportazione di tutti i componenti e le funzioni utilizzate.
\end{itemize}

\section{Strumenti Utilizzati}
La seguente sezione fornisce gli strumenti utilizzati durante lo svolgimento dello stage per la realizzazione della libreria grafica.
Gli strumenti verranno presentati in ordine alfabetico secondo la seguente struttura:
\begin{itemize}
      \item \textbf{Nome strumento}: breve descrizione dello strumento e del suo utilizzo;
      \item \textbf{Versione}: versione utilizzata durante lo stage;
      \item \textbf{Link}: link di riferimento per ulteriori informazioni sullo strumento.
      \item \textbf{Descrizione}: breve descrizione dello strumento e delle sue funzionalità.
      \item \textbf{Vantaggi}: principali vantaggi derivanti dall'utilizzo dello strumento. \textit{(Opzionale)}
      \item \textbf{Svantaggi}: principali svantaggi derivanti dall'utilizzo dello strumento. \textit{(Opzionale)}
      \item \textbf{Alternative Esaminate}: alternative studiate e prese in considerazione durante la scelta dello strumento, con una breve considerazione
            su di esse e la motivazione per cui non sono state selezionate. \textit{(Opzionale)}
\end{itemize}

\subsection{Yarn}
\begin{itemize}
      \item \textbf{Nome strumento}: Yarn
      \item \textbf{Versione}: 1.22.22
      \item \textbf{Link}: \href{https://yarnpkg.com/}{yarnpkg.com}
      \item \textbf{Descrizione}: Yarn è un package manager per JavaScript, sviluppato da Facebook, Google e Tilde.
      \item \textbf{Vantaggi}: Yarn è più veloce di npm, ha un sistema di cache più efficiente e permette di installare pacchetti in modo deterministico.
      \item \textbf{Svantaggi}: Yarn è meno flessibile di npm e non supporta alcune funzionalità di npm.
      \item \textbf{Alternative Esaminate}:
            \begin{itemize}
                  \item \textit{npm}: npm è il package manager di default per Node.js, ma è più lento di Yarn e ha un sistema di cache meno efficiente.
            \end{itemize}
\end{itemize}

% \subsection{React}

\subsection{Plotly.js}
\begin{itemize}
\item \textbf{Nome strumento}: Plotly.js
\item \textbf{Versione}: custom-boundle: 2.33.0
\item \textbf{Link}: \href{https://plotly.com/javascript/}{plotly.com/javascript}
\item \textbf{Descrizione}: Plotly.js è una libreria JavaScript open-source, con supporto per TypeScript, utilizzata per la visualizazione di dati mediante grafici interattivi.
Costruito sopra \textit{D3.js}, Plotly.js offre un vasto panorama di grafici dinamici in formato SVG, altamente personalizzabili.
\item \textbf{Vantaggi}:
\begin{itemize}
      \item Plotly.js offre funzionalità di interattività avanzate, quali zoom, pan, selezione e salvataggio dei grafici;
      \item Le disponibilità di grafici offerte da Plotly.js sono molto ampie, permettendo di soddisfare la maggior parte delle esigenze
            all'interno della libreria grafica;
      \item Plotly.js è supportato da una vasta e attiva community, con ampie serie di esempi disponibili su \textit{CodePen} (una piattaforma di condivisione di codice
            online che permette di visualizzare e modificare codice HTML, CSS e JavaScript);
      \item Plotly.js offre la possibilità di generare custom-bundle personalizzati, permettendo di registrare le sole \textit{traces} che si vogliono utilizzare, riducendo
            così le dimensioni del pacchetto finale;
      \item Plotly.js era già precedentemente utilizzato all'interno della libreria preesistente, non necessitando così di modifiche lato backend negli editor
            utilizzati per la generazione delle risposte JSON da parte dei server Datasoil S.r.l. per la generazione delle dashboard;
      \item Plotly.js offre una funzionalità, \textit{Plotly.react}, che permette di aggiornare i grafici in modo efficiente, riducendo il tempo di rendering
            e migliorando le prestazioni complessive della libreria.
\end{itemize}
\item \textbf{Svantaggi}:
\begin{itemize}
      \item Plotly.js è una libreria molto pesante, dovuto anche dal fatto che è implementata sopra un wrapper proprietario di \textit{D3.js},
            che impedisce l'ottimizzazione di alcune dipendenze non necessarie in quanto non utilizzate;
      \item La documentazione ufficiale di Plotly.js è vaga e poco esaustiva;
      \item La versione ufficiale di Plotly.js presenta degli errori durante la registrazine delle \textit{traces}, impedendo di importare correttamente
            \tetxit{Plotly} all'interno dei moduli TypeScript, motivo per il quale è stata utilizzata una versione custom-boundle di Plotly.js.
\end{itemize}
\item \textbf{Alternative Esaminate}:
\begin{itemize}
      \item \textit{D3.js}: D3.js è una libreria JavaScript open-source utilizzata per la generazione di grafici dinamici e manipolazione dati, la quale dalla sua parte risulta però
            essere più complessa rispetto a Plotly.js, richiedendo una maggiore curva di apprendimento e una maggiore quantità di codice per la generazione di grafici.
      \item \textit{Chart.js}: Chart.js è una libreria JavaScript open-source utilizzata per la generazione di grafici dinamici, la quale risulta essere più leggera di Plotly.js,
            ma offre una minore quantità di grafici disponibili; il suo utilizzo avrebbe inoltre comportato la richiesta di modifiche lato backend.
\end{itemize}


\subsection{TypeScript}
\begin{itemize}
      \item \textbf{Nome strumento}: TypeScript
      \item \textbf{Versione}: 5.1.6
      \item \textbf{Link}: \href{https://www.typescriptlang.org/}{typescriptlang.org}
      \item \textbf{Descrizione}: TypeScript è un linguaggio di programmazione open-source sviluppato da Microsoft, che fa da super-set a JavaScript,
            permettendo di definire tipi ed interfacce per le variabili e i parametri delle funzioni, garantendo una maggiore sicurezza e affidabilità del codice.
\end{itemize}

\section{Progettazione delle componenti}
