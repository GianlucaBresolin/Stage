\chapter{Progettazione}
\label{chap:progettazione}
Nel presente capitolo viene presentata l'attività di progettazione relativa al progetto di stage focalizzato sul refactoring e
sull'ottimizzazione dello SDK impiegato dall'azienda Datasoil S.r.l. per lo sviluppo di dashboard dinamiche.
Verrà analizzato l'approcio adottato per individuare le soluzioni volte al soddisfacimento dei requisiti individuati durante
la fase di analisi dei requisiti, proseguendo con la descrizione dell'architettura progettata per la realizzazione della libreria grafica
e gli strumenti utilizzati per la sua realizzazione, concludendo con la progettazione delle singole componenti individuate.

\section{Metodologia di Progettazione}
La metodologia di progettazione adottata durante lo svolgimento del progetto di stage ha seguito inizialmente un approccio
\textit{top-down}, che permettesse, attraverso uno studio generale della libreria preesistente e della sua architettura, di invididuare
il flusso di dati tra le varie componenti, fornendo una visione globale delle interazioni e delle dipendenze esistenti.
Successivamente, è stato adottato un approccio \textit{bottom-up}, analizzando le singole componenti e le loro funzionalità,
in modo da identificare le possibili ottimizzazioni e le modifiche necessarie per il miglioramento delle prestazioni, della leggibilità e
della manutenibilità del codice.\newline
L'attività di progettazione è stata condotta in collaborazione stretta con il tutor aziendale, il quale ha fornito supporto e orientamento
riguardo alle decisioni progettuali individuate.

\section{Design dell'Architettura}
L'architettura progettata per la realizzazione della libreria grafica si basa su un pattern comunemente utilizzato nello sviluppo di packages
di componenti frontend per \textit{React}, ovvero il pattern \textit{Component Library Architecture}: questa architettura si fonda sulla creazione di una libreria di
componenti riutilizzabili, modulari e indipendenti tra loro. \newline
Tale pattern permette di creare l'architettura ideale per librerie che offrono collezioni di componenti riutilizzabili, permettendo e agevolando
la condivisione e il riutilizzo di codice all'interno del progetto, garantendo una maggior manuntenibilità del codice. \newline
Il modello proposto prevede la seguente struttura, la quale si riflette nella definizione delle macro-directory all'interno del progetto:
\begin{itemize}
      \item \textbf{components}: directory contenente i componenti della libreria grafica, suddivisi in sottodirectory per categoria;
      \item \textbf{hooks}: directory contenente i custom hooks utilizzati all'interno della libreria grafica;
      \item \textbf{utils}: directory contenente le funzioni di utilità utilizzate all'interno della libreria grafica;
      \item \textbf{models}: directory contenente le definizioni dei tipi e interfacce utilizzate all'interno della libreria grafica;
      \item \textbf{index.ts}: file principale della libreria grafica, contenente l'esportazione di tutti i componenti e le funzioni utilizzate.
\end{itemize}
L'entry point della libreria grafica è il file \textit{index.ts}, il quale si occupa di esportare il \textit{Renderer}, il componente principale
della liberia che si occupa di renderizzare il corretto componente a seconda del tipo di visualizzazione richiesto: tali informazioni necessarie
vengono passate come props dalle componenti del modulo \textit{dashboard} che utilizzano il \textit{Renderer}, riducendo così la dipendenza tra le
componenti di librerie differenti, agevolando la manutenibilità e la scalabilità del codice.


\section{Progettazione dei Tipi e delle Interfacce}
Nella seguente sezione verranno presentati i tipi e le interfacce definiti durante l'attività di progettazione, associati ad una descrizione dettagliata
delle informazioni che rappresentano. \newline
I tipi e le interfacce verranno presentati in ordine alfabetico.

\subsection{RendererProps}
Interfaccia che definisce i tipi dei dati passati alla componenti \textit{Renderer, Counter, Chart} e \textit{Table}.
\begin{listing}[H]
      \begin{minted}{typescript}
      export interface RendererProps {
      type: string;
      visualizationName: string;
      data: {
      [x: string]: any;
      };
      options: VisualizationOptions;
      }
      \end{minted}
      \caption{Definizione dell'interfaccia RendererProps}
      \label{listing:rendererProps}
\end{listing}
\begin{itemize}
      \item \textbf{type}: stringa che definisce il tipo componente da renderizzazre. Può assumere i seguenti valori: \textit{COUNTER}, \textit{CHART} e \textit{TABLE};
      \item \textbf{visualizationName}: stringa che definisce il titolo di visualizzazione della componente;
      \item \textbf{data}: oggetto che contiene i dati da visualizzare all'interno della componente;
      \item \textbf{options}: oggetto che contiene le opzioni di visualizzazione della componente di tipo \textit{VisualizationOptions}.
\end{itemize}


\section{Progettazione delle componenti}
Nella seguente sezione verranno presentate le componenti individuate durante l'attività di progettazione, con una descrizione dettagliata delle
funzionalità offerte e delle possibili ottimizzazioni individuate. \newline
Le componenti verranno presentate in ordine alfabetico.

\subsection{Counter}
La componente Counter prevede la visualizzazione di un contatore, il cui valore viene definito a partire dai dati passati come props tra le varie componenti e
forniti dalla risposta in formato JSON ricevuta dai server Datasoil. \newline
La componente permette inoltre di confrontare il valore attuale del contatore con un valore target (opzionale) da raggiungere, anch'esso ricavato
dalla risposta in formato JSON ricevuta dal server. \newline
I props della componente Counter sono definiti dall'interfaccia \textit{RendererProps}, su cui il componente ne effettua il picking delle informazioni
\textit{data, options} e \textit{visualizationName}. \newline
Il valore del contatore può essere formattato in base alle preferenze definite lato backend, presentando le seguenti opzioni:
\begin{itemize}
      \item Prefisso: stringa da aggiungere prima del valore del contatore;
      \item Sufisso: stringa da aggiungere dopo il valore del contatore;
      \item Numero di cifre decimali: numero di cifre decimali da visualizzare nel valore del contatore;
      \item Separatore delle migliaia: carattere da utilizzare come separatore delle migliaia;
      \item Separatore decimale: carattere da utilizzare come separatore decimale.
\end{itemize}
La progettazione della presente componente prevede l'utilizzo di hooks per la gestione del suo stato interno e per il calcolo delle dimensioni
del contatore, necessarie per permettere la visualizzazione responsiva della componente: l'utilizzo di tali hooks permette di garantire
un'implementazione efficiente e performante, riducendo il numero di renderizzazioni e computazioni non necessarie.
