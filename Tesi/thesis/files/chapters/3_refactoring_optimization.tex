\chapter{Fondamenti di Refactoring e Ottimizzazione}
\label{chap:fondamenti-refactoring-ottimizzazione}

\section{Definizione di Refactoring e Ottimizzazione}

\subsection{Refactoring}
Il refactoring del software è una attività di manutenzione che prevede la ristrutturazione del codice sorgente senza modificarne 
il comportamento esterno. L'obiettivo che si pone tale attività consiste nel migliorare la leggibilità e la manutenibilità del codice,
riducendo la complessità e rendendo più semplice l'aggiunta di nuove funzionalità. \newline
I vantaggi che si possono ottenere dal refactoring sono molteplici e impliciti nella qualità del codice prodotto, permettendo di ridurre il 
tempo necessario per la correzione di bug e l'introduzione di nuove funzionalità, grazie all'agevolazione della comprensione del codice.
% TODO: aggiungere riferimento ISO/IEC . Esempio JS -> TS

\subsection{Ottimizzazione}
L'ottimizzazione del software è un'attività di manutenzione che prevede la modifica del codice sorgente: in quanto l'ambito delle ottimizzazioni 
risulta essere vasto e con aspetti fortemente legati al contesto di attuazione, in questa tesi verranno presentate le sole ottimizzazioni operate
durante lo svolgimento dell'attività di stage. \newline
% sorgente per migliorarne le prestazioni, 
% l'utilizzo delle risorse, la scalabilità e l'efficienza del codice stesso.


\newpage