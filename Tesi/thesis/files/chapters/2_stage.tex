\chapter{Lo stage}
\label{chap:descrizione-stage}
Nel seguente capito si descriverà in dettaglio la proposta di stage accettata, indicandonene gli obiettivi, la pianificazione
delle attività, i prodotti attesi, gli strumenti e le tecnologie utilizzate, terminando con le motivazioni che hanno portato 
alla scelta di tale proposta.


\section{Introduzione al progetto}
Lo 

\section{Analisi preventiva dei rischi}

Durante la fase di analisi iniziale sono stati individuati alcuni possibili rischi a cui si potrà andare incontro.
Si è quindi proceduto a elaborare delle possibili soluzioni per far fronte a tali rischi.

\begin{risk}{Performance del simulatore hardware}
    \riskdescription{le performance del simulatore hardware e la comunicazione con questo potrebbero risultare lenti o non abbastanza buoni da causare il fallimento dei test}
    \risksolution{coinvolgimento del responsabile a capo del progetto relativo il simulatore hardware}
    \label{risk:hardware-simulator} 
\end{risk}

\section{Requisiti e obiettivi}

\begin{center}
    \rowcolors{1}{}{tableGray}
    \begin{longtable}{|p{2.25cm}|p{7.75cm}|p{2.25cm}|}
    \hline
    \multicolumn{1}{|c|}{\textbf{A}} & \multicolumn{1}{c|}{\textbf{B}}\\ 
    \hline 
    \endfirsthead
    \rowcolor{white}
    \multicolumn{3}{c}{{\bfseries \tablename\ \thetable{} -- Continuo della tabella}}\\
    \hline
    \multicolumn{1}{|c|}{\textbf{A}} & \multicolumn{1}{c|}{B}\\ \hline 
    \endhead
    \hline
    \rowcolor{white}
    \multicolumn{3}{|r|}{{Continua nella prossima pagina...}}\\
    \hline
    \endfoot
    \endlastfoot 
    
    AA & BB \\
    \hline
    AA & BB \\
    \hline
    AA & BB \\
    \hline
    AA & BB \\
    \hline
    \hiderowcolors
    \caption{Lorem.}
    \label{tab:requisiti_obbiettivi}
    \end{longtable}
\end{center}

\section{Pianificazione}
\begin{figure}[H]
    \centering
    \includegraphics[alt={Testo alternativo dell'immagine}, width=0.5\columnwidth]{img/pk_estate.jpeg}
    \caption{Caption}
    \label{fig:pk_estate_2}
\end{figure}
\lipsum[1]

\begin{enumerate}
    \item[\textbf{F}:] Funzionale.
    \item[\textbf{Q}:] Qualitativo.
    \item[\textbf{V}:] Di vincolo.
    \item[\textbf{N}:] Obbligatorio (necessario).
    \item[\textbf{D}:] Desiderabile.
    \item[\textbf{Z}:] Opzionale.
\end{enumerate}

Nelle tabelle \ref{tab:requisiti_funzionali}, \ref{tab:requisiti_qualitativi} e \ref{tab:requisiti_vincolo} sono riassunti i requisiti e il loro tracciamento con gli use case delineati in fase di analisi.

\section{Tabelle dei requisiti}
\begin{center}
    \rowcolors{1}{}{tableGray}
    \begin{longtable}{|p{2.25cm}|p{7.75cm}|p{2.25cm}|}
    \hline
    %\rowcolor{hyperColor!5}
    \multicolumn{1}{|c|}{\textbf{Requisito}} & \multicolumn{1}{c|}{\textbf{Descrizione}} & \multicolumn{1}{c|}{\textbf{Use Case}}\\
    \hline 
    \endfirsthead
    \rowcolor{white}
    \multicolumn{3}{c}{{\bfseries \tablename\ \thetable{} -- Continuo della tabella}}\\
    \hline
    %\rowcolor{hyperColor!5}
    \multicolumn{1}{|c|}{\textbf{Requisito}} & \multicolumn{1}{c|}{\textbf{Descrizione}} & \multicolumn{1}{c|}{\textbf{Use Case}}\\
    \hline 
    \endhead
    \hline
    \rowcolor{white}
    \multicolumn{3}{|r|}{{Continua nella prossima pagina...}}\\
    \hline
    \endfoot
    \endlastfoot
    
    RFN-1 & L’interfaccia permette di configurare il tipo di sonde del test & UC1 \\
    \hline
    \hiderowcolors
    \caption{Tabella del tracciamento dei requisiti funzionali.}
    \label{tab:requisiti_funzionali}
    \end{longtable}
\end{center}

\begin{center}
    \rowcolors{1}{}{tableGray}
    \begin{longtable}{|p{2.25cm}|p{7.75cm}|p{2.25cm}|}
    \hline
    %\rowcolor{hyperColor!5}
    \multicolumn{1}{|c|}{\textbf{Requisito}} & \multicolumn{1}{c|}{\textbf{Descrizione}} & \multicolumn{1}{c|}{\textbf{Use Case}}\\
    \hline 
    \endfirsthead
    \rowcolor{white}
    \multicolumn{3}{c}{{\bfseries \tablename\ \thetable{} -- Continuo della tabella}}\\
    \hline
    %\rowcolor{hyperColor!5}
    \multicolumn{1}{|c|}{\textbf{Requisito}} & \multicolumn{1}{c|}{\textbf{Descrizione}} & \multicolumn{1}{c|}{\textbf{Use Case}}\\
    \hline 
    \endhead
    \hline
    \rowcolor{white}
    \multicolumn{3}{|r|}{{Continua nella prossima pagina...}}\\
    \hline
    %\caption{Tabella del tracciamento dei requisiti qualitativi.}
    \endfoot
    \endlastfoot
    
    RQD-1n & Le prestazioni del simulatore hardware deve garantire la giusta esecuzione dei test e non la generazione di falsi negativi & - \\
    \hline
    RQD-1n & Le prestazioni del simulatore hardware deve garantire la giusta esecuzione dei test e non la generazione di falsi negativi & - \\
    \hline
    RQD-1n & Le prestazioni del simulatore hardware deve garantire la giusta esecuzione dei test e non la generazione di falsi negativi & - \\
    \hline
    RQD-1n & Le prestazioni del simulatore hardware deve garantire la giusta esecuzione dei test e non la generazione di falsi negativi & - \\
    \hline
    RQD-1n & Le prestazioni del simulatore hardware deve garantire la giusta esecuzione dei test e non la generazione di falsi negativi & - \\
    \hline
    RQD-1n & Le prestazioni del simulatore hardware deve garantire la giusta esecuzione dei test e non la generazione di falsi negativi & - \\
    \hline
    \hiderowcolors
    \caption{Tabella del tracciamento dei requisiti qualitativi.}
    \label{tab:requisiti_qualitativi}
    \end{longtable}
\end{center}

\begin{center}
    \rowcolors{1}{}{tableGray}
    \begin{longtable}{|p{2.25cm}|p{7.75cm}|p{2.25cm}|}
    \hline
    %\rowcolor{hyperColor!5}
    \multicolumn{1}{|c|}{\textbf{Requisito}} & \multicolumn{1}{c|}{\textbf{Descrizione}} & \multicolumn{1}{c|}{\textbf{Use Case}}\\
    \hline 
    \endfirsthead
    \rowcolor{white}
    \multicolumn{3}{c}{{\bfseries \tablename\ \thetable{} -- Continuo della tabella}}\\
    \hline
    %\rowcolor{hyperColor!5}
    \multicolumn{1}{|c|}{\textbf{Requisito}} & \multicolumn{1}{c|}{\textbf{Descrizione}} & \multicolumn{1}{c|}{\textbf{Use Case}}\\
    \hline 
    \endhead
    \hline
    \rowcolor{white}
    \multicolumn{3}{|r|}{{Continua nella prossima pagina...}}\\
    \hline
    \endfoot
    \endlastfoot
    
    RVO-1 & La libreria per l'esecuzione dei test automatici deve essere riutilizzabile & - \\
    \hline
    \hiderowcolors
    \caption{Tabella del tracciamento dei requisiti di vincolo.}
    \label{tab:requisiti_vincolo}
    \end{longtable}
\end{center}

\subsection{subsection}
\lipsum[1]

\subsubsection{subsubsection}
\lipsum[1]

\paragraph{paragraph}
\lipsum[1]

\newpage