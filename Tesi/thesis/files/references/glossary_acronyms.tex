% Acronyms

% Glossary

\newglossaryentrywithacronym{SDK}{SDK}{Software Development Kit}{
    Un \textit{Software Development Kit (SDK)} è un insieme di strumenti di sviluppo software riuniti in un unico pacchetto installabile.
    Gli SDK sono progettati per semplificare il processo di sviluppo di applicazioni, fornendo librerie, strumenti di sviluppo, documentazione e esempi di codice.
}

\newglossaryentrywithacronym{FCP}{FCP}{First Contentful Paint}{
    Il \textit{First Contentful Paint (FCP)} è una metrica che misura il tempo dal momento in cui una pagina inizia il caricamento fino al momento in cui qualsiasi parte
    del contenuto della pagina viene resa visibile sullo schermo.
}

\newglossaryentrywithacronym{SVG}{SVG}{Scalable Vector Graphics}{
    L'\textit{SVG (Scalable Vector Graphics)} è un formato di file basato su \textit{XML} per descrivere immagini vettoriali bidimensionali.
    Questo formato permette la rappresentazione di grafica vettoriale con alta precisione, consentendo lo zoom e il ridimensionamento senza perdita di qualità.
    Gli SVG sono utilizzati ampiamente nel \textit{web design} e nelle applicazioni grafiche grazie alla loro scalabilità e al supporto per interattività e animazioni tramite script e fogli di stile.
}

\newglossaryentrywithacronym{API}{API}{Application Programming Interface}{
    Un'\textit{API (Application Programming Interface)} è un insieme di definizioni e protocolli che permettono a diversi software di comunicare tra loro.
    Le API forniscono un'interfaccia standardizzata che consente agli sviluppatori di accedere a funzionalità e servizi di un altro software, sistema operativo, libreria o
    framework senza conoscere i dettagli della loro implementazione.
}

\newglossaryentrywithacronym{TTI}{TTI}{Time to Interactive}{
    Il Time to Interactive (TTI) è una metrica che misura il tempo necessario affinché una pagina web diventi completamente interattiva, cioè quando è possibile
    interagire con tutti gli elementi principali della pagina senza ritardi significativi.
}

\newglossaryentry{open-source}{
    name={Open-Source},
    sort=open-source,
    description={Il termine open-source si riferisce a un tipo di software il cui codice sorgente è disponibile per l'uso, la modifica e la distribuzione da parte di chiunque. Questo modello promuove
            la collaborazione e la trasparenza nello sviluppo software.}
}

\newglossaryentry{refactoring}{
    name={Refactoring},
    sort=refactoring,
    description={Il refactoring è il processo di ristrutturazione del codice sorgente di un programma senza modificarne il comportamento esterno.
            L'obiettivo è migliorare la leggibilità, la manutenibilità e la riduzione della complessità del codice.}
}

\newglossaryentry{package}{
    name={Package},
    sort=package,
    description={Un package è un insieme di moduli o classi che vengono raggruppati insieme e distribuiti come una singola unità.
            I pacchetti facilitano la gestione e la distribuzione del software, includendo tutte le dipendenze necessarie.}
}

\newglossaryentry{bundle}{
    name={Bundle},
    sort=bundle,
    description={Un bundle è un pacchetto che contiene diversi file e risorse che vengono aggregati insieme per essere distribuiti come un'unica unità.
            In ambito web, un bundle può includere file JavaScript, CSS e immagini.}
}

\newglossaryentry{backend}{
    name={Backend},
    sort=backend,
    description={Il backend si riferisce alla parte server di un'applicazione, che gestisce la logica, le operazioni di database, l'autenticazione degli utenti e
            altre funzionalità che non sono visibili direttamente dall'utente finale.}
}

\newglossaryentry{responsive}{
    name={Responsive},
    sort=responsive,
    description={Il termine responsive si riferisce a un design che è in grado di adattarsi a diverse dimensioni dello schermo e dispositivi,
            offrendo un'esperienza utente ottimale, indipendentemente dal dispositivo utilizzato.}
}

\newglossaryentry{rendering}{
    name={Rendering},
    sort=rendering,
    description={Il rendering è il processo di generazione di un'immagine o di una visualizzazione a partire da un modello o da un insieme di dati.
            In ambito web, il rendering si riferisce alla trasformazione del codice HTML, CSS e JavaScript in una pagina web visibile e interattiva. Il rendering può essere eseguito lato client, nel browser,
            o lato server, prima che la pagina venga inviata al client.}
}

\newglossaryentry{runtime}{
    name={Runtime},
    sort=runtime,
    description={Il runtime si riferisce al periodo di tempo durante il quale un programma è in esecuzione. Il termine viene anche utilizzato per indicare un ambiente o
            una libreria che supporta l'esecuzione di un programma, fornendo servizi come la gestione della memoria e l'esecuzione del codice.}
}

\newglossaryentry{super-set}{
    name={Super-Set},
    sort=super-set,
    description={Un super-set è un insieme che contiene tutti gli elementi di un altro insieme, detto sottoinsieme. Nel contesto della programmazione, un linguaggio o un framework
            può essere considerato un super-set se include tutte le funzionalità di un altro linguaggio o framework, aggiungendo ulteriori caratteristiche e capacità.}
}

\newglossaryentry{peerdependency}{
    name={Peer Dependency},
    sort=peerdependency,
    description={Una \textit{peer dependency} è una dipendenza di un modulo che non viene installata automaticamente, ma che deve essere fornita dall'ambiente in cui il modulo stesso è eseguito.
            Questo tipo di dipendenza è utilizzato per garantire che i moduli condividano una singola istanza di una libreria comune, evitando problemi di incompatibilità e ridondanza.}
}

\newglossaryentry{bundler}{
    name={Bundler},
    sort=bundler,
    description={Un \textit{bundler} è uno strumento utilizzato nello sviluppo web per combinare diversi file e risorse, come \textit{JavaScript}, \textit{CSS} e immagini, in un unico
            file o in un insieme di file più ridotto. Questo processo migliora l'efficienza del caricamento delle pagine web riducendo il numero di richieste HTTP necessarie.}
}

\newglossaryentry{tree-shaking}{
    name={Tree-Shaking},
    sort=tree-shaking,
    description={Il \textit{tree-shaking} è una tecnica di ottimizzazione del codice utilizzata dai \textit{bundler} per rimuovere codice inutilizzato dai file JavaScript. Questo processo
            analizza le dipendenze del codice per identificare e eliminare le parti che non sono effettivamente utilizzate, riducendo la dimensione finale del \textit{bundle} e migliorando le
            prestazioni delle applicazioni web.}
}