% Acronyms

% Glossary

\newglossaryentrywithacronym{SDK}{SDK}{Software Development Kit}{
    Un \textit{Software Development Kit (SDK)} è un insieme di strumenti di sviluppo software riuniti in un unico pacchetto installabile.
    Gli SDK sono progettati per semplificare il processo di sviluppo di applicazioni, fornendo librerie, strumenti di sviluppo, documentazione e esempi di codice
}

\newglossaryentrywithacronym{FCP}{FCP}{First Contentful Paint}{
    Il \textit{First Contentful Paint (FCP)} è una metrica che misura il tempo dal momento in cui una pagina inizia il caricamento fino al momento in cui qualsiasi parte
    del contenuto della pagina viene resa visibile sullo schermo
}

\newglossaryentrywithacronym{SVG}{SVG}{Scalable Vector Graphics}{
    L'\textit{SVG (Scalable Vector Graphics)} è un formato di file basato su \textit{XML} per descrivere immagini vettoriali bidimensionali.
    Questo formato permette la rappresentazione di grafica vettoriale con alta precisione, consentendo lo zoom e il ridimensionamento senza perdita di qualità.
    Gli SVG sono utilizzati ampiamente nel \textit{web design} e nelle applicazioni grafiche grazie alla loro scalabilità e al supporto per interattività e animazioni tramite script e fogli di stile
}

\newglossaryentrywithacronym{API}{API}{Application Programming Interface}{
    Un'\textit{API (Application Programming Interface)} è un insieme di definizioni e protocolli che permettono a diversi software di comunicare tra loro.
    Le API forniscono un'interfaccia standardizzata che consente agli sviluppatori di accedere a funzionalità e servizi di un altro software, sistema operativo, libreria o
    framework senza conoscere i dettagli della loro implementazione
}

\newglossaryentrywithacronym{TTI}{TTI}{Time to Interactive}{
    Il \textit{Time to Interactive (TTI)} è una metrica che misura il tempo necessario affinché una pagina web diventi completamente interattiva, cioè quando è possibile
    interagire con tutti gli elementi principali della pagina senza ritardi significativi
}

\newglossaryentry{open-source}{
    name={Open-Source},
    text={open-source}
    sort=open-source,
    description={Il termine \textit{open-source} si riferisce a un tipo di software il cui codice sorgente è disponibile per l'uso, la modifica e la distribuzione da parte di chiunque.
            Questo modello promuove la collaborazione e la trasparenza nello sviluppo software}
}

\newglossaryentry{refactoring}{
    name={Refactoring},
    text={refactoring}
    sort=refactoring,
    description={Il refactoring è il processo di ristrutturazione del codice sorgente di un programma senza modificarne il comportamento esterno.
            L'obiettivo è migliorare la leggibilità, la manutenibilità e la riduzione della complessità del codice}
}

\newglossaryentry{package}{
    name={Package},
    text={package}
    sort=package,
    description={Un \textit{package} è un insieme di moduli o classi che vengono raggruppati insieme e distribuiti come una singola unità.
            I pacchetti facilitano la gestione e la distribuzione del software, includendo tutte le dipendenze necessarie}
}

\newglossaryentry{bundle}{
    name={Bundle},
    text={bundle}
    sort=bundle,
    description={Un \textit{bundle} è un pacchetto che contiene diversi file e risorse che vengono aggregati insieme per essere distribuiti come un'unica unità.
            In ambito web, un \textit{bundle} può includere file \textit{JavaScript}, \textit{CSS} e immagini}
}

\newglossaryentry{backend}{
    name={Backend},
    text={backend}
    sort=backend,
    description={Il \textit{backend} si riferisce alla parte server di un'applicazione, che gestisce la logica, le operazioni di database, l'autenticazione degli utenti e
            altre funzionalità che non sono visibili direttamente dall'utente finale}
}

\newglossaryentry{responsive}{
    name={Responsive},
    text={}
    sort=responsive,
    description={Il termine \textit{responsive} si riferisce a un design che è in grado di adattarsi a diverse dimensioni dello schermo e dispositivi,
            offrendo un'esperienza utente ottimale, indipendentemente dal dispositivo utilizzato}
}

\newglossaryentry{rendering}{
    name={Rendering},
    text={rendering}
    sort=rendering,
    description={Il rendering è il processo di generazione di un'immagine o di una visualizzazione a partire da un modello o da un insieme di dati.
            In ambito web, il rendering si riferisce alla trasformazione del codice \textit{HTML, CSS} e \textit{JavaScript} in una pagina web visibile e interattiva.
            Il rendering può essere eseguito lato \textit{client}, nel \textit{browser},
            o lato \textit{server}, prima che la pagina venga inviata al \textit{client}}
}

\newglossaryentry{runtime}{
    name={Runtime},
    text={runtime}
    sort=runtime,
    description={Il \textit{runtime} si riferisce al periodo di tempo durante il quale un programma è in esecuzione. Il termine viene anche utilizzato per indicare un ambiente o
            una libreria che supporta l'esecuzione di un programma, fornendo servizi come la gestione della memoria e l'esecuzione del codice}
}

\newglossaryentry{super-set}{
    name={Super-Set},
    text={super-set}
    sort=super-set,
    description={Un super-set è un insieme che contiene tutti gli elementi di un altro insieme, detto sottoinsieme. Nel contesto della programmazione, un linguaggio o un \textit{framework}
            può essere considerato un \textit{super-set} se include tutte le funzionalità di un altro linguaggio o \textit{framework}, aggiungendo ulteriori caratteristiche e capacità.}
}

\newglossaryentry{peerdependency}{
    name={Peer Dependency},
    text={peer-dependency}
    sort=peerdependency,
    description={Una \textit{peer dependency} è una dipendenza di un modulo che non viene installata automaticamente, ma che deve essere fornita dall'ambiente in cui il modulo stesso è eseguito.
            Questo tipo di dipendenza è utilizzato per garantire che i moduli condividano una singola istanza di una libreria comune, evitando problemi di incompatibilità e ridondanza.}
}

\newglossaryentry{bundler}{
    name={Bundler},
    text={bundler}
    sort=bundler,
    description={Un \textit{bundler} è uno strumento utilizzato nello sviluppo web per combinare diversi file e risorse, come \textit{JavaScript}, \textit{CSS} e immagini, in un unico
            file o in un insieme di file più ridotto. Questo processo migliora l'efficienza del caricamento delle pagine web riducendo il numero di richieste HTTP necessarie}
}

\newglossaryentry{tree-shaking}{
    name={Tree-Shaking},
    text={tree-shaking}
    sort=tree-shaking,
    description={Il \textit{tree-shaking} è una tecnica di ottimizzazione del codice utilizzata dai \textit{bundler} per rimuovere codice inutilizzato dai file JavaScript. Questo processo
            analizza le dipendenze del codice per identificare e eliminare le parti che non sono effettivamente utilizzate, riducendo la dimensione finale del \textit{bundle} e migliorando le
            prestazioni delle applicazioni web}
}

\newglossaryentry{memoizzato}{
    name={Memoizzato},
    text={memoizzato}
    sort=memoizzato,
    description={Il termine \textit{memoizzato} si riferisce a una tecnica di ottimizzazione che memorizza il risultato di una funzione o di un calcolo in modo da poterlo riutilizzare
            in seguito senza doverlo ricalcolare. Questo approccio migliora le prestazioni dell'applicazione riducendo il tempo di esecuzione e l'utilizzo delle risorse}
}

\newglossaryentry{frontend}{
    name={Frontend},
    text={frontend}
    sort=frontend,
    description={Il \textit{frontend} si riferisce alla parte visibile di un'applicazione, che interagisce direttamente con l'utente finale. Questa parte dell'applicazione gestisce l'interfaccia
            utente, la presentazione dei dati e le interazioni con l'utente}
}

\newglossaryentry{repository}{
    name={Repository},
    text={repository}
    sort=repository,
    description={Un \textit{repository} è un archivio centralizzato e strutturato per la conservazione, la gestione e il controllo di versioni di codice sorgente, documentazione e altri file correlati
            a un progetto software. I repository possono essere ospitati su server locali o remoti e sono spesso gestiti tramite sistemi di controllo versione, consentendo agli sviluppatori di collaborare,
            tracciare modifiche, gestire diramazioni (\textit{branch}) e fusioni (\textit{merge}), mantenendo una cronologia accurata dello sviluppo del software}
}

\newglossaryentry{framework}{
    name={Framework},
    text={framework}
    sort=framework,
    description={Un \textit{framework} è un insieme di librerie, strumenti e linee guida che forniscono una struttura e un'infrastruttura per lo sviluppo di applicazioni software.
            I \textit{framework} semplificano il processo di sviluppo fornendo funzionalità comuni, standardizzando le pratiche di programmazione e riducendo la complessità del codice}
}

\newglossaryentry{memory leak}{
    name={Memory Leak},
    text={memory leak}
    sort=memoryleak,
    description={Un \textit{memory leak} (perdita di memoria) è una condizione anomala in cui un programma software, durante la sua esecuzione, non rilascia correttamente la memoria che ha allocato.
            Questo comportamento porta ad un utilizzo inefficiente della memoria disponibile, riducendone progressivamente la quantità fino a potenzialmente esaurire tutte le risorse di memoria del sistema.
            I \textit{memory leak} sono spesso causati da errori di programmazione, come la mancata deallocazione di memoria dinamica o riferimenti pendenti a oggetti non più utilizzati}
}

\newglossaryentry{mock}{
    name={Mock},
    text={mock}
    sort=mock,
    description={Un \textit{mock} è un oggetto simulato o fittizio che viene utilizzato al posto di un oggetto o una funzione reale durante i test software. I \textit{mock} vengono utilizzati per simulare il comportamento
            di un oggetto o funzione reale, consentendo ai test di verificare il funzionamento del codice in modo controllato e prevedibile, limitando le responsabilità testate alle sole logiche specifiche}
}

\newglossaryentry{munging}{
    name={Munging},
    text={munging}
    sort=munging,
    description={Il \textit{munging} è una tecnica di trasformazione dei dati che modifica o oscura i dati in modo da renderli irriconoscibili o incomprensibili per chi non è autorizzato a visualizzarli.
            Il \textit{munging} viene spesso utilizzato per proteggere i dati sensibili o per nascondere informazioni personali, come indirizzi email o numeri di telefono, sostituendo i caratteri con simboli o codici.
            Questa tecnica viene inoltre utilizzata per ottimizzare il codice sorgente, riducendo la dimensione dei file}
}